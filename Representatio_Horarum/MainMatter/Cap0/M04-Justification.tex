Los problemas del lenguaje de señas, como se indicó en el apartado anterior, no solo generan barreras y rompen la comunicación, sino que también limitan el acceso a la información y a las oportunidades para las comunidades que tienen esta discapacidad. Es por esto que se considera que una herramienta que pueda comprender y, con el paso del tiempo y su desarrollo, ayudar a unificar las variantes del lenguaje de señas, como una solución de alto impacto. Al poder reducir costos en traducciones, personalizar la atención, entre otros beneficios.\vspace{0.5cm}\par

Con esas ideas en mente, se es consciente de la gran amplitud y magnitud de dicho desafío, sin olvidar las limitaciones de recursos que rodean a muchos proyectos académicos de pregrado, este proyecto de investigación no pretende desarrollar una herramienta de unificación final. Es por esto que su impacto se busca desde determinados puntos, con la intención final de poder sentar las bases y generar nuevo conocimiento de técnicas innovadoras, para la traducción parcial de palabras en lenguaje de señas, señalando a la comunidad investigativa, el gran potencial que se oculta detrás de estas herramientas.\vspace{0.5cm}\par

El primer punto es la evaluación de viabilidad de las técnicas que se seleccionaron, se busca analizar la capacidad de una arquitectura específica (Autoencoder 3D-CNN + GRU Bidireccional) para extraer y estructurar representaciones de palabras en lenguaje de señas por medio de videos. Donde el objetivo es determinar si este enfoque, con una nueva e innovadora función de pérdida, muestra un potencial real para identificar patrones semánticos comunes entre diferentes lenguajes.\vspace{0.5cm}\par

Posteriormente, se planea el establecimiento de una línea base para estas técnicas como un punto de partida, al tener ciertos recursos computacionales explícitamente definidos y delimitados, no es esencial que el rendimiento del modelo sea el mejor. Este será un dato empírico y muy importante que podrá informar a la comunidad científica sobre los requisitos de hardware y datos necesarios para que esta técnica sea exitosa y planteada a mayor escala.\vspace{0.5cm}\par

En otra instancia, se planea la generación de una hoja de ruta para futuros trabajos, por medio del análisis de los resultados, el planteamiento de hipótesis de como se podría potenciar el modelo con otros recursos y especialmente de las visualizaciones del espacio latente (PCA y UMAP), donde estas visualizaciones permitirán formular recomendaciones concretas. Independientemente del rendimiento numérico del modelo, los hallazgos permitirán responder preguntas como: ¿Qué tan separadas o cercanas quedan las señas de diferentes idiomas? ¿La arquitectura tiende a agruparlas por etiqueta o por idioma? En otras palabras, el rechazar o confirmar la hipótesis bajo estas restricciones realmente significa ofrecer un resultado bastante valioso que podrá guiar a otros futuros investigadores sobre qué caminos tomar y cuáles evitar.\vspace{0.5cm}\par

Por otro lado, este proyecto de investigación, también busca animar a los estudiantes de la Universidad Sergio Arboleda, para demostrarles que no se necesita de una solución definitiva, los mejores recursos computacionales o invertir mucho dinero, para hacer aportes valiosos que puedan contribuir con la construcción de un mundo mejor. Si no más bien, de una investigación rigurosa, honesta y fundamental, que por medio de la excelencia académica que caracteriza a los Sergistas, se puedan hacer aportes más significativos, sentando las piedras angulares como un puente hacia un futuro donde grandes soluciones, iniciando con este tipo de aportes, permitan que todos tengan las mismas oportunidades así se tenga una discapacidad auditiva.