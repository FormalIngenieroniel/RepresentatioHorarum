% Página de agradecimientos. En ella el (los) autor (es) expresa (n) el reconocimiento hacia las personas y entidades que asesoraron técnicamente, suministraron datos, financiaron total o parcialmente la investigación o contribuyeron significativamente al desarrollo del tema. Es opcional y debe contener, además de la nota correspondiente, los nombres de las personas con sus respectivos cargos y los nombres completos de las instituciones y su aporte al trabajo.

\section*{Agradecimientos}

En primer lugar, me gustaría expresar mi agradecimiento a la generosidad de la Universidad Sergio Arboleda, que me brindó la oportunidad de cursar toda mi carrera con una beca que materializó mis sueños y metas. Además, agradezco por concederme las diferentes condecoraciones Honores Rodrigo Noguera que porto con orgullo y me animan a dar mi mejor esfuerzo. De igual manera, agradezco la oportunidad de permitirme expandir mis horizontes con un increíble intercambio con la Escuela de Ingeniería Julio Garavito. También quiero expresar mi agradecimiento a Juan Sebastián Malagón, que gracias a su apoyo inicial, pude desarrollar un punto de vista innovador al problema planteado en este proyecto. Por otro lado, a mi asesor Juan Pablo Ospina, por su valiosa ayuda y orientación que, a pesar de no estar presente en todo el desarrollo del proyecto, lo acogió con mucho cariño como si fuera propio. En última instancia, pero no menos importante, agradezco a mi familia y amigos que siempre me brindaron su apoyo incondicional en todo momento que lo necesité. Quiero hacer una mención especial a mi hermana Alejandra Bernal, quien dedicó considerable parte de su tiempo para ayudarme con la producción escrita de este documento y ser mi apoyo emocional en momentos complicados.


%  Se debe dejar la nueva página para guardar el orden de las seciones
\newpage
