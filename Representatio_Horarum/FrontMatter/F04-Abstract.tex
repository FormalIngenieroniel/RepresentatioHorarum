% Presentación abreviada y precisa del contenido de un documento, sin agregar interpretación o crítica y se recomienda un exceda una página.
\section*{Resumen}

El resumen es una presentación abreviada y precisa del contenido de un documento, sin agregar interpretación o crítica. Para documentos extensos como informes, tesis y trabajos de grado, no debe exceder de 500 palabras, y debe ser lo suficientemente breve para que no ocupe más de una página. Se recomienda que este resumen sea anal\'{\i}tico, es decir, que sea completo, con informaci\'{o}n cuantitativa y cualitativa, generalmente incluyendo los siguientes aspectos: objetivos, dise\~{n}o, lugar y circunstancias, pacientes (u objetivo del estudio), intervenci\'{o}n, mediciones y principales resultados, y conclusiones. Al final del resumen se deben usar palabras claves tomadas del texto (m\'{\i}nimo 3 y m\'{a}ximo 10 palabras), las cuales permiten la recuperaci\'{o}n de la informaci\'{o}n.\vspace{0.5cm}\par

% debe incluir una lista de palabras clave 
\textbf{Palabras clave}: Lenguaje de señas, Espacio latente, Autoencoders, Modelo reconstructivo, Autoencoder variacional, Aprendizaje contrastivo.

% Incluir un resumen en otro idioma de preferencia inglés
\newpage
\section*{Abstract}

% debe incluir una lista de palabras clave en el otro idioma
\textbf{Keywords}: Sign language, Latent space, Autoencoders, Reconstructive model, Variational autoencoder, Contrastive learning.

