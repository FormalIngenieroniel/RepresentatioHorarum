%        1         2         3         4         5         6         7         8         9         0         1         2
%23456789012345678901234567890123456789012345678901234567890123456789012345678901234567890123456789012345678901234567890
\documentclass[12pt]{MIA-USA}

\usepackage{layout}
\usepackage{subcaption}
\usepackage{paralist}
\usepackage{breakcites}

\usepackage[T1]{fontenc}
\usepackage{textcomp}
\usepackage{float}
\usepackage{longtable}
\usepackage{array}

\usepackage{lmodern}

\usepackage[breaklinks=true]{hyperref}
\hypersetup{
	pdftitle={Documento de trabajo Final}, 
	pdfauthor={D.Bernal}, 
	pdfsubject={}, 
	pdfcreator={Daniel Bernal with Latex}, 
	colorlinks=true,
	linkcolor=GrisUno,
	citecolor=AzulInstitucional,
	filecolor=AzulClaro,
	urlcolor=AzulInstitucional,
	linktoc=all
}

\graphicspath{{Images/}}


\title{Análisis de Técnica Reconstructiva y Contrastiva para la Traducción de Palabras en Lenguajes de Señas mediante Espacios Latentes con Datos y Recursos Limitados}
\author{Daniel Camilo Bernal Ternera}
\documenttype{Documento Final}

\advisor{Nombres y apellidos Asesor}

% Si hay un coasesor 
% \coadvisor{Nombres y apellidos Co-Asesor}


% Aqui comienza el documento
% - -- --- ----- ------- ------- ----- --- -- -
\pgfplotsset{compat=1.18}
\begin{document}
    % \layout{}
	
	\maketitle
	% Incluye la pagina de acptación (Opcional)
	\input{FrontMatter/F01-Approval}
	% Incluye la dedicatoria (Opcional)
	%  Nota mediante la cual el autor ofrece su trabajo, en forma especial, a personas o entidades. Su presentación es opcional y debe conservar los márgenes
\subsection*{Dedicatoria}
\vspace*{1.5cm}
{\hfill
    \begin{minipage}{0.5\textwidth}
        \noindent
         A mi tierra, a la fuerza de mi raíz,

        al profundo e inmenso cielo azul del llano colombiano,
        donde pude encomendar mis ideas a un gavilán pescador, para que las dejara volar con la libertad del viento, 
        
        y a la selva, donde la colosal ceiba, como regio soporte, permita florecer de forma inconmensurable los posibles proyectos que puedan surgir de este estudio.
    
    \end{minipage}
}
\newpage
	% Incluye los agradecimientos (Opcional)
	% Página de agradecimientos. En ella el (los) autor (es) expresa (n) el reconocimiento hacia las personas y entidades que asesoraron técnicamente, suministraron datos, financiaron total o parcialmente la investigación o contribuyeron significativamente al desarrollo del tema. Es opcional y debe contener, además de la nota correspondiente, los nombres de las personas con sus respectivos cargos y los nombres completos de las instituciones y su aporte al trabajo.

\section*{Agradecimientos}

En primer lugar, me gustaría expresar mi agradecimiento a la generosidad de la Universidad Sergio Arboleda, que me brindó la oportunidad de cursar toda mi carrera con una beca que materializó mis sueños y metas. Además, agradezco por concederme las diferentes condecoraciones Honores Rodrigo Noguera que porto con orgullo y me animan a dar mi mejor esfuerzo. De igual manera, agradezco la oportunidad de permitirme expandir mis horizontes con un increíble intercambio con la Escuela de Ingeniería Julio Garavito. También quiero expresar mi agradecimiento a Juan Sebastián Malagón, que gracias a su apoyo inicial, pude desarrollar un punto de vista innovador al problema planteado en este proyecto. Por otro lado, a mi asesor Juan Pablo Ospina, por su valiosa ayuda y orientación que, a pesar de no estar presente en todo el desarrollo del proyecto, lo acogió con mucho cariño como si fuera propio. En última instancia, pero no menos importante, agradezco a mi familia y amigos que siempre me brindaron su apoyo incondicional en todo momento que lo necesité. Quiero hacer una mención especial a mi hermana Alejandra Bernal, quien dedicó considerable parte de su tiempo para ayudarme con la producción escrita de este documento y ser mi apoyo emocional en momentos complicados.


%  Se debe dejar la nueva página para guardar el orden de las seciones
\newpage

	% Resumen
    
    % Presentación abreviada y precisa del contenido de un documento, sin agregar interpretación o crítica y se recomienda un exceda una página.
\section*{Resumen}

El resumen es una presentación abreviada y precisa del contenido de un documento, sin agregar interpretación o crítica. Para documentos extensos como informes, tesis y trabajos de grado, no debe exceder de 500 palabras, y debe ser lo suficientemente breve para que no ocupe más de una página. Se recomienda que este resumen sea anal\'{\i}tico, es decir, que sea completo, con informaci\'{o}n cuantitativa y cualitativa, generalmente incluyendo los siguientes aspectos: objetivos, dise\~{n}o, lugar y circunstancias, pacientes (u objetivo del estudio), intervenci\'{o}n, mediciones y principales resultados, y conclusiones. Al final del resumen se deben usar palabras claves tomadas del texto (m\'{\i}nimo 3 y m\'{a}ximo 10 palabras), las cuales permiten la recuperaci\'{o}n de la informaci\'{o}n.\vspace{0.5cm}\par

% debe incluir una lista de palabras clave 
\textbf{Palabras clave}: Lenguaje de señas, Espacio latente, Autoencoders, Modelo reconstructivo, Autoencoder variacional, Aprendizaje contrastivo.

% Incluir un resumen en otro idioma de preferencia inglés
\newpage
\section*{Abstract}

% debe incluir una lista de palabras clave en el otro idioma
\textbf{Keywords}: Sign language, Latent space, Autoencoders, Reconstructive model, Variational autoencoder, Contrastive learning.


    \let\cleardoublepage\clearpage
    
    % Crea la tabla de contenidos a partir de la estructura del documento
    \tableofcontents
    % Crea la lista de figuras (Opcional)
    \listoffigures
    % Crea la lista de tablas (Opcional)
    \listoftables
    % Incluye la lista de simbolos (Opcional)
    %\input{FrontMatter/F05-Symbols}
    \let\cleardoublepage\clearpage
    % Glosario (Opcional)
    %\input{FrontMatter/F06-Glossary}
    \let\cleardoublepage\clearpage
    
    % Introduction
    \chapter{Introducción}
    %% En ella, el autor presenta y señala la importancia, el origen (los antecedentes teóricos y prácticos), los objetivos, los alcances, las limitaciones, la metodología empleada, el significado que el estudio tiene en el avance del campo respectivo y su aplicación en el área investigada.

De acuerdo con la NTC1486, la introducción es un espacio dónde el autor presenta y señala la importancia, el origen (los antecedentes teóricos y prácticos), los objetivos, los alcances, las limitaciones, la metodología empleada, el significado que el estudio tiene en el avance del campo respectivo y su aplicación en el área investigada. No debe confundirse con el resumen, ni contener un recuento detallado de la teoría, el método o los resultados, como tampoco anticipar las conclusiones y recomendaciones, y se recomienda que la introducci\'{o}n tenga una extensi\'{o}n de m\'{\i}nimo 2 p\'{a}ginas y m\'{a}ximo de 4 p\'{a}ginas.\vspace{0.5cm}\par

Para el desarrollo del documento utilizando la plantilla en \LaTeX se recomienda la guía \textit{the not so short guide to \LaTeX} que brinda una introducci\'on breve pero muy completa. En donde se presentan entro otras cosas los comandos y ambeintes para el trabajo con gr\'aficas y tablas, como las que se muestran a continuaci\'on:

\begin{figure}[htbp!]
    \caption{Muestra de inclusio\'on de un elemento gr\'afico}
    \centering
    \includegraphics[width=0.8\textwidth]{Images/ImagenYoda.jpg}
    \label{fig:yoda}
\end{figure}

As\'i la Figura~\ref{fig:yoda} muestra una imagen que se encuentra en el directorio images, mientras la tabla~\ref{tab:tablasEx}, muestra dos ejemplos de información tabular con combinaci\'on de columnas, y combinaci\'on de filas.

\begin{table}[htbp!]
    \centering
    \caption{Ejemplo de tabla con multi columnas (arriba) y multi filas(abajo)}
    \begin{tabular}{ |p{3cm}||p{3cm}|p{3cm}|p{3cm}|  }
        \hline
        \multicolumn{4}{|c|}{Country List} \\
        \hline
        Country Name or Area Name& ISO ALPHA 2 Code &ISO ALPHA 3 Code&ISO numeric Code\\
        \hline
        Afghanistan   & AF    &AFG&   004\\
        Aland Islands&   AX  & ALA   &248\\
        Albania &AL & ALB&  008\\
        Algeria    &DZ & DZA&  012\\
        American Samoa&   AS  & ASM&016\\
        Andorra& AD  & AND   &020\\
        Angola& AO  & AGO&024\\
        \hline
    \end{tabular}\\
    \vspace{0.5cm}
    \begin{tabular}{ |c|c|c|c| } 
        \hline
        col1 & col2 & col3 \\
        \hline
        \multirow{3}{4em}{Multiple row} & cell2 & cell3 \\ 
        & cell5 & cell6 \\ 
        & cell8 & cell9 \\ 
        \hline
    \end{tabular}

    \label{tab:tablasEx}
\end{table}

Adem\'as se recomienda consultar los manuales que aparecen en al secci\'on de aprendizaje de overleaf, por ejemplo el de \href{https://www.overleaf.com/learn/latex/Tables}{tablas}.

La presente plantilla tiene en cuenta aspectos importantes de la Norma T\'{e}cnica Colombiana - NTC 1486. Las m\'{a}rgenes, numeraci\'{o}n, tama\~{n}o de las fuentes y dem\'{a}s aspectos de formato, deben ser conservada de acuerdo con esta plantilla, la cual esta dise\~{n}ada para imprimir por lado y lado en hojas tama\~{n}o carta. Se sugiere que los encabezados cambien seg\'{u}n la secci\'{o}n/cap\'{i}tulo del documento.\vspace{0.5cm}\par

La redacci\'{o}n debe ser impersonal y gen\'{e}rica. La numeraci\'{o}n de las hojas sugiere que las p\'{a}ginas preliminares se realicen en n\'{u}meros romanos en may\'{u}scula y las dem\'{a}s en n\'{u}meros ar\'{a}bigos, en forma consecutiva a partir de la introducci\'{o}n que comenzar\'{a} con el n\'{u}mero 1. La cubierta y la portada no se numeran pero si se cuentan como p\'{a}ginas.\vspace{0.5cm}\par

Para trabajos muy extensos se recomienda publicar m\'{a}s de un volumen. Se debe tener en cuenta que algunas facultades tienen reglamentada la extensi\'{o}n m\'{a}xima de las tesis  o trabajo de investigaci\'{o}n; en caso que no sea as\'{\i}, se sugiere que el documento no supere 120 p\'{a}ginas.\vspace{0.5cm}\par

No se debe utilizar numeraci\'{o}n compuesta como 13A, 14B \'{o} 17 bis, entre otros, que indican superposici\'{o}n de texto en el documento. Para resaltar, puede usarse letra cursiva o negrilla. Los t\'{e}rminos de otras lenguas que aparezcan dentro del texto se escriben en cursiva.\vspace{0.5cm}\par

El contenido de este documento esta basado principalmente en la NTC1486 y la plantilla de Tesis de maestr\'ia y doctorado de la Universidad Nacional de colombia.
    \let\cleardoublepage\clearpage

    % Problema de investigacion
    \chapter{Problema de investigación}
    
\section*{Panorama del lenguaje de señas}

Mediante una investigación, se pudo evidenciar que el lenguaje de señas abarca un campo de estudio de gran relevancia y tamaño que ha sido abordado a lo largo de los años desde múltiples disciplinas, las cuales incluyen la lingüística, la sociología, la antropología y las ciencias de la computación, entre otras más. Esta aproximación multidisciplinaria va acorde a su naturaleza como un sistema lingüístico bastante completo y complejo, que va más allá de la percepción que usualmente se tiene de ser únicamente una serie de gestos manuales y sistemáticos\autocite[p.~216]{Probierz2023}. Esta profundidad surge en parte por la diversidad y contexto en el que se desarrolla \enquote{There are many different sign languages in the world, each with its own unique characteristics. Therefore, translating from a national language into a sign language requires not only knowledge of the sign language, but also an understanding of the culture and social context of the people who use it}\autocite[p.~213]{Probierz2023}. Además, como sugiere Probierz, como sistema lingüístico natural, el lenguaje de señas posee su propia gramática, vocabulario y estructura sintáctica, lo que requiere un enfoque interdisciplinario que integre la lingüística con el análisis cultural para poder entender toda su complejidad\autocite[p.~213]{Probierz2023}.\vspace{0.5cm}\par

No obstante, esta complejidad aumenta bastante al considerar que, a pesar de lo que las personas piensan comúnmente, no existe un lenguaje de señas universal. Tan solo en Estados Unidos, de los 30 millones de personas con pérdida auditiva, únicamente 6 millones utilizan el ASL, y entre esos, solo 2 millones son usuarios activos.\autocite[p.~851]{Adler2025} A nivel mundial, la situación se vuelve más complicada y fragmentada, teniendo más de 200 lenguajes de señas distintos. Esta diversidad, si bien es un reflejo de la riqueza cultural, representa un gran obstáculo significativo para la comunicación global. Además, la falta de recursos accesibles se convierte en una barrera sistemática que contribuye a que las personas con discapacidades queden atrasadas respecto a sus pares sin discapacidades en la búsqueda de una carrera profesional, como ocurre en STEMM\autocite[p.~851]{Adler2025}.\vspace{0.5cm}\par

Estableciendo que el lenguaje es un constructo profundo, con múltiples capas en su origen y por ende en su significado, donde el componente cultural desempeña un papel fundamental como uno de los pilares centrales en su constitución. Se puede analizar que en el caso de las lenguas de señas, estas no solo funcionan como un sistema de comunicación, sino que también encarnan una identidad y una cultura propias, enriquecidas por diferentes elementos históricos, sociales y educativos que complementan y construyen su complejidad\autocite[p.~166]{Bedoin2024}. Además, para entender el lenguaje de señas, también es importante analizar la sordera desde sus diferentes componentes \enquote{Deafness as a historical and social phenomenon is analysed through an interdisciplinary approach, including Deaf history, Sign language, Deaf culture, Deaf identity, Deaf education, etc.}\autocite[p.~163]{Bedoin2024}. Con relación a lo anterior, el lenguaje de señas se conecta y puede abordar desde diferentes puntos de vista que lo componen para desentrañar su complejidad, como cualquier otra forma de comunicación entre humanos, van más allá de la sola funcionalidad de comunicar para ser en un pilar de la identidad cultural de las comunidades sordas a través del mundo\autocite[p.~163]{Bedoin2024}.\vspace{0.5cm}\par

Siguiendo ese orden de ideas, las lenguas de señas no son muy diferentes de las orales, porque ambas están ligadas a variaciones dialécticas que son un reflejo de la diversidad en la cultura de las diferentes regiones de los hablantes. Cuando se habla de variaciones, se hace referencia a acentos, regionalismos y otras expresiones lingüísticas que enriquecen la diversidad del lenguaje\autocite[p.~162]{Bedoin2024}. Con lo anterior claro, ya no es descabellado pensar que el Lenguaje de Señas Colombiano (LSC) presenta diferencias entre regiones, como entre la región del pacífico y de la capital, siguiendo la lógica del español que se habla allí, donde se pueden notar grandes diferencias en los acentos de los habitantes de estas áreas. Este concepto antes descrito se conoce como la diversidad dialectal en las lenguas de señas, este se alinea con el reconocimiento de la multilingualidad en los distintos estudios dirigidos a personas con discapacidad auditiva, en los cuales se ha podido observar que está ocurriendo un cambio dentro el estudio de múltiples lenguas, los cuales tienen en cuenta las lenguas de señas heredadas\autocite[p.~166]{Bedoin2024}. Este cambio se ha podido observar en diferentes estudios \enquote{Scholars often conducted their studies on a single pair of languages (e.g. English and ASL). The interest in bimodal bilingualism has recently shifted towards multilingualism (Swanwick 2016), including heritage spoken and sign languages}\autocite[p.~163]{Bedoin2024}. Lo que implica que cada vez es más fácil, al tener más estudios en esta área y enfoque, sustentar que las lenguas de señas no son homogéneas, sino que, por el contrario, estas tienen ciertas variaciones que, como se establece con anterioridad, son un reflejo de la riqueza lingüística y cultural de las comunidades con alguna discapacidad de tipo auditivo. Es por eso que también se pueden plantear posibles conflictos relacionados con problemas sociales actuales, como lo puede ser el incremento de la migración, donde las lenguas heredadas pueden jugar un papel esencial, al significar choques culturales entre la lengua establecida en el territorio y la que ha tenido que desplazarse e integrarse al mismo.\vspace{0.5cm}\par

Por lo cual, cuando se empieza a ver las lenguas de señas desde otra perspectiva y no solo como herramientas comunicativas, se puede ver que constituyen un componente muy importante de la identidad cultural de las diferentes comunidades con discapacidad al rededor del mundo. Entonces, cuando se tiene esta identidad, que va más allá de las limitaciones físicas y se construye a partir de una comunidad que históricamente se ha desarrollado como ninguna otra por sus diferentes condiciones de vida y contextos, se puede ver que en realidad abarca un área inmensamente diversa de experiencias culturales y lingüísticas. Como indica Bedoin \enquote{Deaf people have long been considered members of a unique and uniform community – with a Sign Language, a Deaf culture and a Deaf identity. This leads to the affirmation of a homogeneous or even a universal Deaf culture from which a strong Deaf identity plays a positive role in the recognition of Deaf communities around the world}\autocite[p.~163]{Bedoin2024}.

\section*{El problema con el lenguaje de señas}

No obstante, a pesar de todas las investigaciones que se han realizado desde diferentes perspectivas, aún no se ha llegado a una solución efectiva del problema, esto solo genera una mayor falta de sensibilización y el surgimiento de etiquetas equivocadas que se le asignan tanto al lenguaje como a las personas involucradas siendo consecuencias mayormente significativas para las PDHL (People Disabling Hearing Loss), especialmente en el acceso a servicios de salud, lo cual es contradictorio, pues debería ser este campo el que tuviera este problema cubierto en mayor medida.\vspace{0.5cm}\par

La falta de conciencia sobre la perdida de audición puede resultar en experiencias negativas que afectan la independencia, la confianza y el bienestar psicológico de las personas que poseen esta discapacidad\autocite[p.~7]{Parmar2025}. Así mismo, es importante destacar cómo la falta de ciertos ajustes al sistema y la necesidad constante de justificarse por sí mismas generan frustración y agotamiento emocional en las personas con esta discapacidad. \enquote{Respondents in this study detailed their experiences during NHS clinical consultations, highlighting how the lack of deaf awareness has adversely impacted their access to services, independence, confidence, and psychological well-being. Additionally, they reported feeling a perceived responsibility to self-advocate to secure a better standard of care.}\autocite[p.~14]{Parmar2025}\vspace{0.5cm}\par

Así mismo, estos problemas se identifican en el área académica y profesional, incluso en las interacciones informales, o en espacios profesionales para fomentar el networking y el aprendizaje colaborativo, donde frecuentemente excluyen a las personas con discapacidad, ya que las ayudas visuales propuestas como las auto-capturas o la escritura resultan insuficientes ante la rapidez y complejidad de estas conversaciones\autocite[p.~852]{Adler2025}. De igual manera, un científico discapacitado puede encontrarse constantemente en una dinámica de "ponerse al día", en la cual, cuando logra entender toda la información y des atrasarse, sus compañeros de trabajo con la capacidad de escuchar ya han avanzado bastante en la discusión, empeorando una brecha de conocimiento que afecta directamente su competitividad, en un escenario real que puede ser el obtener financiación en sus investigaciones, especialmente si deciden ser independientes\autocite[p.~852]{Adler2025}. De igual manera, la falta de intérpretes de ASL, y posiblemente cualquier lenguaje de señas, con conocimientos técnicos en áreas especializadas hace mucho peor esta situación, donde se sufre el riesgo posible de caer en malas interpretaciones o traducciones que pueden ser críticas en un contexto científico.\autocite[p.~853]{Adler2025}.\vspace{0.5cm}\par

Igualmente, las personas con discapacidad mencionaron con urgencia que tienen sentimientos de ansiedad, humillación y miedo a no poder transmitir información vital debido a barreras del lenguaje, lo que en los peores escenarios lleva a las personas evitar la atención médica. \enquote{The way I have been treated whenever I have needed to attend either a hospital or doctor’s appointment makes me scared to go on my own and I tend to avoid contacting the health services even when it’s likely I need them.} \autocite[p.~11]{Parmar2025}\vspace{0.5cm}\par

Teniendo en cuenta las experiencias de las personas que tienen esta discapacidad, que muestran cómo los malentendidos, en su forma de comunicarse usando el lenguaje de señas, no solo impulsan el uso de etiquetas erróneas, sino que también, como se aborda en el estudio citado, a potenciar desigualdades en la atención médica, afectando la calidad de vida de las PDHL. Además de revisar la importancia cultural y significativa que puede tener algo tan profundo a la comunidad como lo es el lenguaje. Se decide abordar el problema de la barrera del idioma en el lenguaje de señas, principalmente por lo mencionado anteriormente, entre otros motivos que se abordan en la justificación y marco teórico. Con la intención de poder aportar a futuras investigaciones que a través de ciertas técnicas de inteligencia artificial puedan llegar a crear soluciones efectivas que impulse la unificación del lenguaje de señas a nivel global.\vspace{0.5cm}\par

\section*{Indagación de estudios previos con el lenguaje de señas}

Ahora bien, con un problema definido que se quiere solucionar, se realiza una investigación al estado del arte relacionado con la integración de tecnologías avanzadas, como la inteligencia artificial, el aprendizaje automático y el procesamiento del lenguaje natural con el lenguaje de señas para poder encontrar un campo que no haya sido trabajo de manera exhaustiva, identificando lo que se realizó y cuál es la importancia que cada estudio puede aportar. Es importante precisar que no se mencionaran todos los estudios consultados en este apartado, pues además de ser bastante extenso y perder la cohesión entre los argumentos, solo se quiere mencionar a los principales estudios que más adelante moldearían el curso de la investigación y el cómo se llegó a una pregunta de investigación bien delimitada, los demás estudios serán referenciados en el marco teórico.\vspace{0.5cm}\par

Al revisar el estudio \enquote{A Survey on Chinese Sign Language Recognition: From Traditional Methods to Artificial Intelligence}\autocite[p.~1-40]{Jiang2024} podemos ver que esencialmente se trata de una revisión exhaustiva de las metodologías y tecnologías empleadas en el reconocimiento de la Lengua de Señas China (CSL) a lo largo de los últimos 20 años. Lo realiza haciendo especial énfasis en la evolución de los métodos, desde enfoques tradicionales hasta innovaciones basadas en inteligencia artificial.\vspace{0.5cm}\par

Inicialmente, durante la realización del estudio se utilizaron Modelos Ocultos de Markov (HMM) y Máquinas de Vectores de Soporte (SVM), las cuales se centraban en la clasificación de gestos a partir de determinadas características extraídas, así como la estrategia Dynamic Time Warping (DTW), la cual permitía alinear secuencias de gestos con variaciones en la velocidad. Sin embargo, en la última década, la investigación cambio su punto de vista, ahora hacia Redes Neuronales Profundas (DNN), que han demostrado ser muy importantes para mejorar la precisión y robustez del reconocimiento de gestos, por otro lado, la implementación de modelos híbridos que combinan diferentes arquitecturas para optimizar el rendimiento también han demostrado ser de gran utilidad.\vspace{0.5cm}\par

Además, el estudio también abarca la exploración de la integración de múltiples modalidades, lo anterior quiere decir que, busca fusionar características tanto de gestos como con expresiones faciales y con movimientos de labios, lo que demostró ser crucial para una interpretación más precisa y fiel del lenguaje de señas. Teniendo todo lo anterior en cuenta, los objetivos de esta investigación no solo van hasta mejorar la precisión y robustez de los sistemas de reconocimiento, sino que también desarrollar conjuntos de datos más grandes y estandarizados para que se pueda facilitar el entrenamiento de modelos, así como promover aplicaciones prácticas que hagan accesible esta tecnología al público general. No obstante, a pesar de los avances significativos que se mencionan, el documento también registra que hay desafíos que no se han podido superar aún, como lo es la fusión de características de lengua de señas continuas y la coordinación de gestos con expresiones faciales, incluso señalando la necesidad de mejorar la robustez y el rendimiento de este tipo de algoritmos en tiempo real. En conclusión, se puede prever que el continuo desarrollo de nuevas tecnologías y la integración de diferentes campos científicos impulsarán aún más de forma positiva el reconocimiento de la lengua de señas china, con un énfasis determinado en modelos híbridos y diferentes técnicas de aprendizaje profundo, lo que promete generar un futuro mucho más accesible y efectivo para las todas aquellas personas con discapacidades auditivas.\vspace{0.5cm}\par

Por otro lado, al revisar el estudio \enquote{A Systematic Review of Hand Gesture Recognition: An Update From 2018 to 2024}\autocite[p.~1-35]{Hashi2024} se puede ver que este presenta una revisión sistemática de la literatura sobre el reconocimiento de lenguaje de señas, haciendo énfasis especial en los métodos de visión, sensores y enfoques híbridos que se han utilizado entre 2018 y 2024.\vspace{0.5cm}\par

Cuando se revisa en profundidad, este muestra que la metodología del estudio se basa en un proceso estructurado que incluye primeramente la identificación del dominio de investigación, luego la selección de estudios a través de bases de datos como IEEE Xplore, ScienceDirect, Scopus y Web of Science, y por último la aplicación de criterios de inclusión y exclusión para filtrar un total de 1,316 artículos, de los cuales solo se seleccionaron 256 para un análisis exhaustivo. Los objetivos principales del estudio se centran en evaluar la representación de lenguaje de señas, la adquisición de datos y la precisión de los métodos de reconocimiento. Así mismo, se concluye que la precisión del reconocimiento puede variar en gran medida, alcanzando de esta manera entre 64\% y 98\% en casos donde se puede conocer la identidad del intérprete, y entre 52\% y 98\% en situaciones donde no es relevante, con un promedio de 87.9\% y 79\% respectivamente.\vspace{0.5cm}\par

Al revisar lo anterior con detenimiento, se pueden identificar desafíos en la caracterización de gestos continuos y de igual manera se destaca la necesidad de mejorar la viabilidad práctica de los sistemas de reconocimiento de gestos basados en visión, así como resaltar la importancia de integrar otros enfoques interdisciplinarios que consideren aspectos de interacción que estén enfocados en la interacción humano-computadora sumada a la complejidad del lenguaje de señas, que al igual que el estudio anterior, se concluye que comprende no solo movimientos de las manos, sino que también expresiones faciales y lenguaje corporal.\vspace{0.5cm}\par

El documento de la investigación menciona varios métodos utilizados en el reconocimiento de gestos, específicamente en el contexto del reconocimiento de lenguaje de señas. Estos métodos se pueden clasificar en tres categorías principales, siendo los primeros los métodos Basados en Visión. Estos métodos se caracterizan por utilizar información visual para reconocer gestos, esto quiere decir que, incluyen varias etapas, como la recopilación de datos, el preprocesamiento, la segmentación, la extracción de características y la clasificación. Estos se enfocan de manera principal en el análisis de imágenes o secuencias de video para identificar gestos estáticos y dinámicos.\vspace{0.5cm}\par

En otra instancia, están los métodos Basados en Sensores, los cuales se enfocan en utilizar sensores, como indica su nombre, que usualmente están integrados en guantes o dispositivos portátiles, que son utilizados para capturar datos sobre los movimientos de las manos. Estos sensores son capaces de medir parámetros como lo son la flexión, la orientación y la rotación de la mano. Esto hace que este método sea menos susceptible a las condiciones que se relacionan al entorno, lo que permite una captura de datos más precisa, aunque puede ser percibido como incómodo debido a la necesidad de usar múltiples dispositivos.\vspace{0.5cm}\par

Por último, se tiene a los métodos Híbridos, estos métodos se caracterizan por combinar técnicas de visión además de sensores para aprovechar las ventajas de ambos enfoques. Al integrar datos visuales y de sensores, se busca mejorar la precisión y la robustez del reconocimiento de gestos. Es importante recalcar que el documento también destaca la importancia de emplear algoritmos de aprendizaje profundo, como lo pueden ser las redes neuronales convolucionales (CNN), para mejorar la eficacia del reconocimiento de gestos, así como la necesidad de abordar aspectos no manuales del lenguaje de señas, como las expresiones faciales y el lenguaje corporal, lo cual puede significar en la omisión de datos que son cruciales para una interpretación fiel del significado en el lenguaje de señas.\vspace{0.5cm}\par

Siguiendo con la idea del uso de sensores, se analiza el estudio \enquote{American Sign Language Recognition and Translation Using Perception Neuron Wearable Inertial Motion Capture System}\autocite[p.~1-15]{Gu2024} el cual trata del reconocimiento y la traducción de la Lengua de Señas Americana (ASL) a través de un sistema de captura de movimiento inercial conocido como \enquote{Perception Neuron}.\vspace{0.5cm}\par

A lo largo del estudio se recopila un conjunto de datos en los cuales se incluyen 300 oraciones comúnmente usadas en ASL, compuestas por 455 gestos diferentes, a los que se les asignan dos tipos de etiquetas. La primera de ellas se basa en la gramática de la lengua de señas, mientras que la otra se asigna desacuerdo a la gramática del lenguaje hablado. Así mismo, se desarrollan dos modelos de procesamiento de lenguaje natural (NLP) para el reconocimiento de secuencias y la traducción de extremo a extremo.\vspace{0.5cm}\par

La metodología de los procesos antes descritos van desde la segmentación manual de palabras para su validación, es decir, se seleccionan 20 palabras para evaluar la precisión del clasificador, que combina un extractor de características CNN y una capa de clasificación totalmente conectada, donde los resultados muestran una alta precisión en la clasificación con un promedio de alrededor del 88\%.\vspace{0.5cm}\par

A pesar de que se identifican confusiones entre palabras con gestos similares, las conclusiones del estudio muestran que, aunque el modelo de reconocimiento logra una alta precisión, se encontró que la traducción de extremo a extremo presenta una mayor cantidad de errores debido a la falta de un conocimiento gramatical. Aparte de eso, se observa que la validación a nivel de oración se vuelve más difícil con un vocabulario más extenso, lo que termina reduciendo la tasa de precisión. Dentro del estudio se subraya de gran manera la importancia de las diferencias individuales dentro de cada uno de los gestos y la necesidad de mejorar la comprensión gramatical para optimizar la traducción de ASL.\vspace{0.5cm}\par

Para tener un enfoque diverso en la metodología para hallar una solución, se revisa el estudio \enquote{Sign language recognition using the fusion of image and hand landmarks through multi-headed convolutional neural network}\autocite[p.~1-11]{Pathan2023} el cual se centra en la fusión de técnicas de procesamiento de imágenes y aprendizaje profundo, más específicamente a través de la utilización de una red neuronal convolucional multi-cabeza (CNN). Para lograrlo se hace uso de un conjunto de datos de imágenes denominado \enquote{Finger Spelling, A} para entrenar el modelo, esto con el objetivo de mejorar la precisión en la detección de gestos de la lengua de señas americana (ASL).\vspace{0.5cm}\par

Las metodologías incluyen la extracción de características utilizando transformadas de características invariantes a escala como lo es (SIFT) y la clasificación mediante el uso de redes neuronales, así también como la implementación de técnicas de detección de bordes como lo pueden ser Canny y ORB. Así mismo, como se ve en otros estudios, se aplican redes neuronales convolucionales 3D (3DRCNN) para capturar tanto la información espacial como temporal de los gestos.\vspace{0.5cm}\par

Para finalizar, este estudio concluye en que la combinación del procesamiento de imágenes tradicional sumado con la extracción de puntos de referencia de la mano más el uso de CNN multi-cabeza puede permitir una tasa de detección mucho mejor, logrando unos resultados positivos incluso en condiciones donde hay mucho ruido y variabilidad en las imágenes. A través de este enfoque no solo se mejora la precisión del reconocimiento, sino que también reduce la necesidad de hardware costoso, lo que lo hace accesible.\vspace{0.5cm}\par

Para analizar en mejor medida como se pueden usar tanto videos como imágenes, se introduce \enquote{Sign Language Recognition System for Communicating to People with Disabilities}\autocite[p.~13-20]{Obi2023} Este estudio se caracteriza por usar una metodología que se basa en redes neuronales convolucionales (CNN) que para poder facilitar la comunicación entre personas con discapacidades auditivas y la sociedad en general, realiza una recolección de un conjunto de datos de señas en lenguaje americano (ASL) de Kaggle, que incluye 24 clases de gestos, aplicando un filtro de desenfoque gaussiano para mejorar la calidad de las imágenes.\vspace{0.5cm}\par

Posteriormente, para poder lograr el objetivo, se implementan diversas técnicas de procesamiento de imágenes, como lo es la detección de contornos, la extracción de características KAZE y la conversión de espacios de color, para poder preparar los datos para su respectiva clasificación. Esta clasificación se realiza utilizando algoritmos como el de vecinos más cercanos y, principalmente, el de CNN, que ha demostrado ser el método más preciso, siendo capas de alcanzar una precisión del 100\% en imágenes y del 73\% en videos.\vspace{0.5cm}\par

En conclusión, lo valioso de este sistema es que está diseñado para recibir entradas de video en tiempo real, procesar las imágenes y convertir los gestos en texto que puede leer cualquier persona, lo cual permite una interacción más fluida y efectiva. También se destaca la importancia de este tipo de tecnologías en la mejora de la comunicación para aquellas personas con discapacidad, subrayando la necesidad de herramientas accesibles y precisas que faciliten la interacción en situaciones cotidianas y críticas.\vspace{0.5cm}\par

Para profundizar aún más sobre los videos y su relación con los autoencoders, se trae a colación el documento \enquote{VTAN: A Novel Video Transformer Attention-Based Network for Dynamic Sign Language Recognition}\autocite[p.~2793-2812]{Deng2024} el cual se caracteriza por utilizar un modelo innovador para el reconocimiento de lenguaje de señas de una manera dinámica, el modelo combina un autoencoder convolucional (CAE) y un transformador que se enfoca en atención suave.\vspace{0.5cm}\par

Ahora bien, en este documento se abordan dos problemas muy importantes, siendo el primero la redundancia de fotogramas importantes en videos de lenguaje de señas y el segundo, la necesidad de enfocarse principalmente en las regiones de las manos, las cuales se considera que son muy importantes para la traducción de gestos. Para lograrlo se usa un módulo de agregación de características visuales, a través de un CAE para extraer fotogramas clave mediante clustering K-means, reduciendo de esta manera los datos redundantes y mejorando de manera considerable la eficiencia computacional. Por otro lado, este mismo módulo de mejora de características espacio-temporales (STHE) emplea un transformador para poder priorizar las características de las manos, capturando de esta manera dinámicas existentes entre el espacio y su temporalidad.\vspace{0.5cm}\par

Para probar lo anterior, los experimentos se realizaron en los conjuntos de datos AUTSL y SLR500, los cuales muestran mejoras significativas en precisión, alcanzando la cifra de 93.6\% y 91.3\%, respectivamente, destacando la efectividad que tiene VTAN frente a otros métodos previos.\vspace{0.5cm}\par

Por último, para finalizar esta breve compilación de los documentos que principalmente moldearon la idea principal, se tiene el trabajo \enquote{G2P-DDM: Generating Sign Pose Sequence from Gloss Sequence with Discrete Diffusion Model}\autocite[p.~6234-6242]{Xie2024} el cual propone un modelo innovador, este modelo se enfoca en la producción de lenguaje de señas (SLP), centrándose en la transformación de secuencias de etiquetas, es decir anotaciones o etiquetas textuales de los signos, en secuencias de poses (G2P) En pocas palabras, una traducción de texto a señas.\vspace{0.5cm}\par

Para poder lograrlo, el estudio utiliza un enfoque que consta de dos etapas, la primera de estas se basa en que el modelo llamado \enquote{Pose-VQVAE} convierte secuencias continuas de poses en códigos latentes discretos, el modelo realiza esta tarea dividiendo el esqueleto humano en tres partes principales siendo estas el cuerpo, la mano derecha y la mano izquierda, empleando múltiples estructuras en códigos para mejorar la reconstrucción. Mientras que en la segunda etapa, la cual se llama \enquote{G2P-DDM}, se basa principalmente en un modelo de difusión discreta, es decir, el modelo genera secuencias de poses a partir de las diferentes etiquetas usando la herramienta CodeUnet. Posteriormente, se utiliza un algoritmo de clustering secuencial-KNN el cual se encarga de predecir longitudes variables en las secuencias.\vspace{0.5cm}\par

En otras palabras, este trabajo centra su investigación en la producción automática de lenguaje de señas al discretizar el espacio de poses y emplear modelos de difusión, para luego abordar desafíos como lo pueden ser la variabilidad en la longitud de las secuencias y la complejidad de los gestos dinámicos. A través del uso de representaciones latentes discretas, las cuales permiten explorar el espacio latente para tareas como la generación de datos sintéticos hasta la interpolación de gestos.

\section*{Identificación de los conceptos principales del problema}

Teniendo en cuenta los estudios previos que se seleccionaron previamente, se realiza un recuento secuencial de como cada uno influencio y moldeo el planteamiento del problema. Bajo ese orden de ideas, se comienza con la necesidad de realizar un avance en un campo, de preferencia uno aún no investigado en gran medida, para poder contribuir en el mismo realizando una investigación que sirva de base.\vspace{0.5cm}\par

El primer estudio \enquote{A Survey on Chinese Sign Language Recognition: From Traditional Methods to Artificial Intelligence}\autocite[p.~1-40]{Jiang2024}, el cual mostró una revisión exhaustiva durante dos décadas de la evolución en el reconocimiento de la Lengua de Señas China (CSL), con la cual se destacó la transición de métodos tradicionales, hacia enfoques más recientes basados en Redes Neuronales Profundas (DNN). Con este enfoque se llegó a una conclusión muy importante, la cual fue resaltar la importancia en capturar la información completa del intérprete, es decir, gestos manuales, expresiones faciales y movimientos corporales, en lugar de limitarse solamente a las manos. Además de mostrar los beneficios de centrarse en la búsqueda de una solución  avanzada, alejándose de enfoques tradicionales y justificando la evolución hacia técnicas de aprendizaje profundo. Por estos motivos se comenzó a indagar en la posibilidad de emplear un autoencoder capaz de procesar toda la información que se le es transmitida, asegurando de esta manera una representación más completa de los datos.\vspace{0.5cm}\par

Por otro lado, el segundo documento, \enquote{A Systematic Review of Hand Gesture Recognition: An Update From 2018 to 2024}\autocite[p.~1-35]{Hashi2024}, fundamento el uso de la herramienta que se planteó en el anterior documento al revisar 1,300 estudios, de los cuales la mayoría perteneció a métodos de visión, sensores e híbridos. Esto quiere decir que los autoencoders no han sido explorados de manera sustancial en comparación con otras técnicas. Esta investigación también llegó a la conclusión que es importante considerar elementos aparte de las manos para el reconocimiento del lenguaje de señas, como las expresiones faciales y otro lenguaje corporal, para llegar a una interpretación precisa. Por lo tanto, lo encontrado solo reforzó aún más la elección de un autoencoder que capture el contexto completo.\vspace{0.5cm}\par

Por el contrario, el tercer estudio, \enquote{American Sign Language Recognition and Translation Using Perception Neuron Wearable Inertial Motion Capture System}\autocite[p.~1-15]{Gu2024}, ayudo a delimitar la idea, pues con su investigación, respecto a la evaluación del uso de sensores inerciales para el reconocimiento de la Lengua de Señas Americana (ASL), expuso las limitaciones de los métodos basados en sensores. Siendo estas su gran costo, la complejidad de implementación y la incomodidad de los usuarios al llevar puestos diferentes dispositivos portátiles. Con lo anterior claro, se pudo descartar esta aproximación para poder llegar a soluciones más accesibles y escalables. Sin embargo, el estudio también resaltó la importancia de distinguir gestos similares, lo que inspiró la idea teórica de utilizar un espacio latente para diferenciarlos según sus distancias representacionales. Además, esta idea es muy compatible con las el funcionamiento de los autoencoders los cuales son capaces de generar representaciones compactas y significativas, sentando las bases para su posible uso no solo en reconocimiento, sino que también en una potencial traducción.\vspace{0.5cm}\par

Teniendo la idea un poco más clara, se pudo explorar una alternativa para poder hacer funcionar lo anterior, así es como en el cuarto estudio, \enquote{Sign language recognition using the fusion of image and hand landmarks through multi-headed convolutional neural network}\autocite[p.~1-11]{Pathan2023}, se demostró el potencial del uso de técnicas que involucran redes 3D convolucionales (3DRCNN) para capturar información espacial y temporal, lo cual sugirió la posibilidad de desarrollar arquitecturas que sean capaces de detectar la temporalidad de los gestos, lo cual es un aspecto muy importante del lenguaje de señas.\vspace{0.5cm}\par

Ahora bien, para sustentar la utilización de los datos, el quinto estudio, \enquote{Sign Language Recognition System for Communicating to People with Disabilities}\autocite[p.~13-20]{Obi2023}, presento una gran idea en el uso de CNN para procesar secuencias de video de ASL, lo cual se puede unir con el anterior estudio demostrando que es posible realizar este proceso, en este caso particular se logró realizar con técnicas de preprocesamiento como detección de contornos y extracción de características. Este documento, además de aportar un ejemplo concreto de éxito en el procesamiento de video, reforzando la confianza en las capacidades de los autoencoders para este propósito, también revelo pasos importantes dentro del preprocesamiento de los videos para tener un mayor margen de éxito.\vspace{0.5cm}\par

De igual manera, para sentar las últimas bases de las generalidades del problema, se revisó el sexto estudio, \enquote{VTAN: A Novel Video Transformer Attention-Based Network for Dynamic Sign Language Recognition}\autocite[p.~2793-2812]{Deng2024}, que introdujo un modelo que ya se centraba completamente en un autoencoder convolucional (CAE) con un transformador, lo cual deja en evidencia su posible potencial que se estableció teóricamente con los primeros estudios analizados. Además, este enfoque abordó un problema clave cuando se trata con videos, siendo este mismo, la redundancia de frames en videos de lenguaje de señas para seleccionar características relevantes y reducir datos innecesarios.\vspace{0.5cm}\par

En última instancia, el séptimo documento, \enquote{G2P-DDM: Generating Sign Pose Sequence from Gloss Sequence with Discrete Diffusion Model}\autocite[p.~6234-6242]{Xie2024}, aunque tiene una aproximación muy diferente, sirvió de inspiración para constatar el cómo sería evaluado el proyecto, al proporcionar un punto teórico que refuerza lo visto en previos estudios y es la implementación e importancia de representaciones latentes discretas dentro de un espacio para realizar traducciones. Esta idea fue clave para definir el enfoque final, pues sirve como puente para sustentar el concepto de utilizar autoencoders no solo para identificar patrones, sino también para traducir palabras en lenguajes de señas a través de un espacio latente. Esta traducción se basa en el comportamiento de diferentes lenguajes de señas en un mismo espacio latente donde se teoriza que luego de un proceso determinado, es posible que se agrupen las mismas palabras de distintos lenguajes, indicando que existe alguna relación que pueda permitir la asociación de etiquetas al ingresar una nueva palabra de un lenguaje que no ha sido entrenado, esto al ubicarse cerca de sus palabras equivalentes en otros idiomas.

\section*{Delimitación del problema}

Dejando en claro como surge la base del problema a través de ciertos documentos claves, ahora se realiza una acotación para poder enfocar el proyecto de investigación a alcances realistas y obtener resultados determinados, para hacer recomendaciones certeras sobre el cómo se puede enfocar una investigación de este tipo en un campo en concreto. Es por esto que se opta por hacer la mención de \enquote{recursos determinados} como una delimitación en la investigación, esto con el fin de responder a una decisión consciente del alcance del proyecto, basada en las herramientas y capacidades accesibles en el contexto de un estudiante de pregrado de la Universidad Sergio Arboleda, sin incurrir en costos monetarios significativos.\vspace{0.5cm}\par

Teniendo lo anterior en cuenta primero se realiza una búsqueda preliminar de los datos a utilizar, esta búsqueda es muy importante, pues desde la teoría y la practica se ha establecido como un principio fundamental que es muy valioso en el aprendizaje automático el señalar que la calidad y el rendimiento de un modelo de IA son directamente proporcionales a la calidad y representatividad de sus datos de entrenamiento, este principio es conocido como \enquote{Garbage In, Garbage Out}\autocite{GIGOEBSCO}.\vspace{0.5cm}\par

Por eso, se establecen los siguientes parámetros de búsqueda basándose en los estudios previamente mencionados, comenzando por la característica que el dataset este compuesto por videos de intérpretes realizando palabras continuas, no por deletreo de señas. Se toma esta decisión para poder detectar la temporalidad de la secuencia que compone a una palabra, como se indica en el estudio \enquote{VTAN: A Novel Video Transformer Attention-Based Network for Dynamic Sign Language Recognition}\autocite[p.~2793-2812]{Deng2024} Además, este parámetro se respalda con los estudios anteriores que denotan la importancia de buscar toda la información que se omite cuando solo se concentra en una seña en particular, al solo tener un deletreo no se hace uso de otras partes del cuerpo que pueden dar información importante adicional.\vspace{0.5cm}\par

Consecuentemente con lo anterior, se busca que los videos contengan vistas completas del cuerpo para capturar toda la información posible, de igual manera que haya cierta diversidad de intérpretes y de videos por palabra es importante, como se menciona en \enquote{A Systematic Review of Hand Gesture Recognition: An Update From 2018 to 2024}\autocite[p.~1-35]{Hashi2024} \enquote{A critical need emerges for the development of more comprehensive and diverse databases to facilitate a more thorough investigation of SLR systems} donde luego se justifica que al cumplir esta diversidad se lograría que el modelo generalizara de mejor manera.\vspace{0.5cm}\par

Siguiendo con los requerimientos, unos muy importantes son los que están relacionados con su calidad, es decir, fondo limpio, condiciones buenas de iluminación, anotaciones y etiquetas consistentes y tamaños normalizados. Esto es muy importante para simplificar pasos de preprocesamiento y limpieza de datos como los que se mencionan en la sección de preprocesamiento de imágenes en el estudio \enquote{A Systematic Review of Hand Gesture Recognition: An Update From 2018 to 2024}\autocite[p.~1-35]{Hashi2024} donde se repasa desde métodos para la Eliminación de objetos no deseados en el fondo hasta la configuración del tamaño de los videos.\vspace{0.5cm}\par

Por último, para lograr una traducción parcial consistente de palabras por medio de sus representaciones latentes se necesita que se tengan las mismas etiquetas, esto implica que cuando se menciona anteriormente la necesidad de tener variedad en las palabras, es importante que se logre sin caer en regionalismos para tener una buena equivalencia entre diferentes idiomas. Consecuentemente, se necesita tener las mismas palabras en diferentes idiomas para lograr dicha equivalencia.\vspace{0.5cm}\par

Después de una búsqueda exhaustiva tanto en los documentos como en la red, se llegó a la conclusión que cuando se delimita la búsqueda según los anteriores parámetros establecidos existe la escasez de datasets públicos, a gran escala y de alta calidad. Los recursos que existen actualmente tiene la particularidad de no cumplir frecuentemente con las características necesarias para un entrenamiento ideal, lo cual no resulta ser una sorpresa, pues ya se han mencionado en este documento, con varios estudios, que a partir de este problema se enfocaron en la creación de datasets de calidad, como lo es el ejemplo de \enquote{A Survey on Chinese Sign Language Recognition: From Traditional Methods to Artificial Intelligence}\autocite[p.~1-40]{Jiang2024}, \enquote{A Systematic Review of Hand Gesture Recognition: An Update From 2018 to 2024}\autocite[p.~1-35]{Hashi2024} y \enquote{VTAN: A Novel Video Transformer Attention-Based Network for Dynamic Sign Language Recognition}\autocite[p.~2793-2812]{Deng2024} el cual se caracteriza por utilizar un modelo innovador para el reconocimiento de lenguaje de señas de una manera dinámica, por mencionar unos pocos.\vspace{0.5cm}\par

A pesar de esto, se pudieron identificar tres datasets que cumplen con la mayoría de lo anteriormente descrito, sin embargo, tienen ciertas limitaciones. El primer dataset seleccionado consiste en videos de palabras del lenguaje de señas americano, este dataset cuenta con alrededor de 2000 palabras, siendo el dataset más grande que hay hasta la fecha de su creación del lenguaje americano, la versión que se emplea del dataset es la 03\autocite{WLASL2020}.\vspace{0.5cm}\par

El segundo dataset que se eligió fue de lenguaje de señas en indio, se escogió este porque aparte de cumplir con la mayoría de requisitos antes descritos, tiene 4292 videos de diferentes palabras, siendo estás muy variadas\autocite{Sridhar2020}.\vspace{0.5cm}\par

El último dataset que se decidió utilizar consiste en 20400 videos grabados por 194 diferentes intérpretes, el problema principal con este dataset es que no están grabados en espacios uniformes y su captura fue con diferentes cámaras\autocite{Kapitanov2023}.\vspace{0.5cm}\par

Se decidió utilizar solo estos tres datasets, pues son los que más videos etiquetados contienen, además de estar abiertos al público general, permitiendo una descarga accesible. También son los que presentan la mayor calidad en la presentación de sus datos, haciéndolos la mejor opción. Sin embargo, las etiquetas y la manera de como los datos están estructurados son totalmente diferentes entre sí, lo que complica el preprocesamiento y unificación de un formato de los datos, por las razones anteriores, se decidió no incluir más datasets, los diferentes costos que representan en tiempo de investigación el incluir más conjuntos de datos que no están estructurados de la misma manera, no es viable. Además, el experimento funciona teóricamente de buena manera con solo dos lenguajes de señas.\vspace{0.5cm}\par

Los recursos que se utilizaron para este proyecto de investigación fueron brindados por la Universidad gracias a las cuentas de prueba académicas de AWS academy learner lab, se contaron con 50 dólares con los cuales se pudo acceder a diferentes cuadernos del servicio Amazon SageMaker AI, que sé alojaron en la nube para ejecutar todo el código. Las mejores instancias a las que se pudieron acceder son las ml.c5.xlarge, estas cuentan con optimización para computación, no son de inicio rapido, 4 vCPU y 8GiB de memoria. No se contó con memoria GPU para la realización del proyecto, por lo cual no se pudo realizar diferentes estrategias de cómputo con CUDA o aceleración con gráficas\autocite{AWSSageMaker2022}. Es importante recalcar que cada sesión del laboratorio tiene un máximo de 4 horas para trabajar antes de que se reinicie, una vez ocurre esto toca volver a cargar todos los recursos, pues se borra toda la información que tenía el kernel, además de tener que encender todos los servicios que se estaban utilizando antes del apagado.\vspace{0.5cm}\par

La última delimitación que se realiza antes de tener el problema completamente formulado es la teórica, se toma la decisión de enfocar el trabajo de investigación en una técnica de aprendizaje autosupervisado (Self-Supervised Learning) para aprender representaciones de video. Se piensa abordar esta estrategia con un modelo que se compone de un autoencoder basado en convoluciones 3D (3D-CNN) para extraer características espaciotemporales, y una Red Neuronal Recurrente (GRU) Bidireccional para modelar las secuencias temporales. Posteriormente, para estructurar el espacio latente se hace uso de una función de pérdida compuesta que incluye la reconstrucción (MSE), múltiples pérdidas triplet para la estructura temporal y una pérdida KL. Por último, se ve el resultado de la estructuración de las representaciones por medio de PCA y UMAP.\vspace{0.5cm}\par

La decisión de utilizar un enfoque de aprendizaje autosupervisado nace del estudio \enquote{Advancing video self-supervised learning via image foundation models}\autocite{Wu2025} el cual señala que hay una manera de realizar de manera eficiente el aprendizaje de un modelo cuando no se dispone de datos etiquetados a nivel de frame. Además, presenta un concepto que se utilizara en el proyecto de investigación, las tareas de pretexto, según la anterior investigación, estas son las que se centran en crear desafíos que permitan a los modelos aprender relaciones espaciales y temporales en los datos sin etiquetas.\vspace{0.5cm}\par

Así mismo la arquitectura 3d se justifica al leer el documento \enquote{Batch feature standardization network with triplet loss for weakly-supervised video anomaly detection}\autocite{Yi2022} en el cual se expone en la sección 2.1 la idea de que los autoencoders, especialmente los 3D, son muy buenos cuando se emplean para aprender representaciones en videos, siendo capaces de reconstruir, con un buen rendimiento, eventos normales y potencialmente anomalías debido a su capacidad de generalización en los patrones espacio-temporales que capturan.\vspace{0.5cm}\par

En acompañamiento a esta arquitectura se establece que la mejor opción sería incluir una capa GRU Bidireccional, esto gracias a las afirmaciones de la investigación \enquote{Entwicklung und Evaluation eines Deep-Learning Algorithmus für die Worterkennung aus Lippenbewegungen für die deutsche Sprache}\autocite{Pham2022} en la cual se establece que, una Red Neuronal Recurrente (GRU) Bidireccional para modelar las secuencias temporales, es una arquitectura más que adecuada para tareas de reconocimiento de acciones y lenguaje de señas en formato de video. Esto lo explica con diferentes aproximaciones, mencionando que las CNN 3D extraen información espacial y temporal de los videos, mientras que las GRU modelan las secuencias temporales, haciéndolo un modelo hibrido que es capas de captar tanto información espacial como temporal en videos.\vspace{0.5cm}\par

En cuanto a las funciones de perdida, se escoge la aproximación antes descrita en el primer párrafo, basándose en el documento \enquote{ATCM-Net: A deep learning method for phase unwrapping based on perception optimization and learning enhancement}\autocite{Xu2025} en donde se describe la formulación de una función de pérdida compuesta y cómo esta combina diferentes componentes para mejorar el entrenamiento y la capacidad del modelo. Al realizar una buena combinación se permite enfocar el aprendizaje en múltiples objetivos, siguiendo prácticas recomendadas para mejorar la precisión y la robustez del modelo. Sin embargo, el método que se expone es uno limitado, es por eso que se decidió adoptar el concepto de la función compuesta, pero empleando triplet loss, se tomó esta decisión luego de analizar el documento \enquote{A novel triplet loss architecture with visual explanation for detecting the unwanted rotation of bolts in safety-critical environments}\autocite{Bolton2025} Donde se expone que el triplet loss está diseñado para ser robusto frente a variaciones en las condiciones de captura. Esto es muy bueno porque permite que el modelo se vuelva invariante a ciertos tipos de ruido, lo que es crítico en el contexto del proyecto de investigación donde, como se explicó anteriormente, los datasets no son los mejores y las imágenes pueden no ser perfectas.\vspace{0.5cm}\par

De igual manera, para poder garantizar que el modelo pueda tener una sólida representación óptima de un dato, se integra el uso combinado de la pérdida de Divergencia de Kullback-Leibler (KL) y el Error Cuadrático Medio (MSE) respaldándose en el documento \enquote{TB-Net: Intra- and inter-video correlation learning for continuous sign language recognition}\autocite{Liu2024} el cual muestra que reconocen la relevancia del principio del IB para el aprendizaje de representaciones. Para ahondar más en este concepto y como afecta al aprendizaje se refirió a la bibliografía de este documento para encontrar que en \enquote{PAC-BAYES INFORMATION BOTTLENECK}\autocite{Wang2022} se evalúa la diferencia entre la distribución de las representaciones latentes aprendidas y una distribución previa simple, por medio de la pérdida de divergencia KL y MSE. Donde se busca minimizar esta divergencia para que de esta forma el espacio latente sea estructurado y evitar que el modelo memorice los datos, lo que se alinea con el objetivo de mantener una representación de mínima complejidad.\vspace{0.5cm}\par

Por último, se decide emplear las herramientas UMAP y TSNE para la visualización de los espacios latentes, donde luego de consultar \enquote{Comparison of dimensionality reduction techniques for the visualisation of chemical space in organometallic catalysis}\autocite{Villares2024} se puede definir que para tener un panorama más completo de los datos, se usarán las dos herramientas donde TSNE puede ser utilizada para obtener una representación inicial de los datos, mientras que UMAP puede afinar esa representación, proporcionando una mejor separación y visualización.

\section*{Definición final del problema}

Luego de la revisión documental completa, que se plantea anteriormente como registro de los antecedentes de la creación y diseño de la solución al problema de investigación, finalmente se llega a su definición completa que se usara para el resto del proyecto de investigación, la cual considerando todo lo anteriormente dicho, contempla que este proyecto de investigación se enfoca en sentar las bases para un contexto debidamente limitado, el análisis de la representación de palabras individuales en un espacio latente a través de unas técnicas y arquitecturas específicas. Donde, principalmente, se busca aportar conocimiento sobre el uso de autoencoders con estrategias específicas para subsanar el problema que se encontró previamente dentro del campo del lenguaje de señas. En otras palabras, el problema de investigación se centra en el comportamiento y la organización de las palabras dentro de un determinado espacio vectorial de baja dimensionalidad, para facilitar su visualización, con el fin de plantear recomendaciones a trabajos futuros que puedan resultar en una posible herramienta que sea capas de unificar el lenguaje de señas. Teniendo en cuenta esto, surge la siguiente pregunta de investigación \enquote{¿En qué medida puede un autoencoder a través de determinadas técnicas, entrenado con tres lenguajes de señas y evaluado con recursos determinados, identificar patrones útiles para representar palabras en un espacio latente?}.
    \let\cleardoublepage\clearpage

    % Justificación
    \chapter{Justificación}
    Los problemas del lenguaje de señas, como se indicó en el apartado anterior, no solo generan barreras y rompen la comunicación, sino que también limitan el acceso a la información y a las oportunidades para las comunidades que tienen esta discapacidad. Es por esto que se considera que una herramienta que pueda comprender y, con el paso del tiempo y su desarrollo, ayudar a unificar las variantes del lenguaje de señas, como una solución de alto impacto. Al poder reducir costos en traducciones, personalizar la atención, entre otros beneficios.\vspace{0.5cm}\par

Con esas ideas en mente, se es consciente de la gran amplitud y magnitud de dicho desafío, sin olvidar las limitaciones de recursos que rodean a muchos proyectos académicos de pregrado, este proyecto de investigación no pretende desarrollar una herramienta de unificación final. Es por esto que su impacto se busca desde determinados puntos, con la intención final de poder sentar las bases y generar nuevo conocimiento de técnicas innovadoras, para la traducción parcial de palabras en lenguaje de señas, señalando a la comunidad investigativa, el gran potencial que se oculta detrás de estas herramientas.\vspace{0.5cm}\par

El primer punto es la evaluación de viabilidad de las técnicas que se seleccionaron, se busca analizar la capacidad de una arquitectura específica (Autoencoder 3D-CNN + GRU Bidireccional) para extraer y estructurar representaciones de palabras en lenguaje de señas por medio de videos. Donde el objetivo es determinar si este enfoque, con una nueva e innovadora función de pérdida, muestra un potencial real para identificar patrones semánticos comunes entre diferentes lenguajes.\vspace{0.5cm}\par

Posteriormente, se planea el establecimiento de una línea base para estas técnicas como un punto de partida, al tener ciertos recursos computacionales explícitamente definidos y delimitados, no es esencial que el rendimiento del modelo sea el mejor. Este será un dato empírico y muy importante que podrá informar a la comunidad científica sobre los requisitos de hardware y datos necesarios para que esta técnica sea exitosa y planteada a mayor escala.\vspace{0.5cm}\par

En otra instancia, se planea la generación de una hoja de ruta para futuros trabajos, por medio del análisis de los resultados, el planteamiento de hipótesis de como se podría potenciar el modelo con otros recursos y especialmente de las visualizaciones del espacio latente (PCA y UMAP), donde estas visualizaciones permitirán formular recomendaciones concretas. Independientemente del rendimiento numérico del modelo, los hallazgos permitirán responder preguntas como: ¿Qué tan separadas o cercanas quedan las señas de diferentes idiomas? ¿La arquitectura tiende a agruparlas por etiqueta o por idioma? En otras palabras, el rechazar o confirmar la hipótesis bajo estas restricciones realmente significa ofrecer un resultado bastante valioso que podrá guiar a otros futuros investigadores sobre qué caminos tomar y cuáles evitar.\vspace{0.5cm}\par

Por otro lado, este proyecto de investigación, también busca animar a los estudiantes de la Universidad Sergio Arboleda, para demostrarles que no se necesita de una solución definitiva, los mejores recursos computacionales o invertir mucho dinero, para hacer aportes valiosos que puedan contribuir con la construcción de un mundo mejor. Si no más bien, de una investigación rigurosa, honesta y fundamental, que por medio de la excelencia académica que caracteriza a los Sergistas, se puedan hacer aportes más significativos, sentando las piedras angulares como un puente hacia un futuro donde grandes soluciones, iniciando con este tipo de aportes, permitan que todos tengan las mismas oportunidades así se tenga una discapacidad auditiva.
    \let\cleardoublepage\clearpage
    
    % Objetivos
    \chapter{Objetivos}
    % Incluye los objetivos general y específicos del proyecto. Debe procurar seguir una metodología de formulación de objetivos, por ejemplo S.M.A.R.T.

\section{Objetivo general}
\begin{itemize}
    \item  Analizar, mediante una técnica reconstructiva y contrastiva, patrones en el espacio latente de autoencoders entrenados con lenguajes de señas para la Traducción de palabras
\end{itemize}

\section{Objetivos específicos}
\begin{itemize}
    %\item Determinar la aplicabilidad de la técnica seleccionada con base a la revisión bibliográfica.
    \item Establecer e implementar los pasos adecuados de preprocesamiento y organización a los datos para garantizar su calidad, coherencia y compatibilidad con la técnica seleccionada.
    \item Aplicar los datos obtenidos a la técnica de manera sistemática con respaldo en investigaciones previas para la obtención de datos.
    \item Evaluar los resultados obtenidos para poder establecer una base técnica para futuras investigaciones en representaciones latentes de señas y su posible equivalencia interlingüística.
\end{itemize}
    \let\cleardoublepage\clearpage

    % Trabajos previos
    \chapter{Marco Teórico}
    

\section{Marco Histórico}

\subsection{Historia y Reconocimiento del Lenguaje de Señas}

Cuando se hace un recuento histórico, se puede evidenciar que antes del siglo XVII, hay pocos registros escritos sobre el lenguaje de señas o la educación de personas sordas. Es por esto que es difícil hacer un recuento histórico de fechas que daten antes de esto, sin embargo, es importante anotar que las señas como una forma de comunicación que surge primero que la hablada en la historia de la humanidad. Siguiendo con la documentación histórica del lenguaje de señas, usualmente se reconocen figuras determinadas que demostraron el potencial intelectual de las personas sordas. Como por ejemplo, en Francia, una de las figuras más antiguas e importantes es Etienne de Fay (1669 - 1750). El cual fue sordo de nacimiento, además de ser educado en la abadía de Amiens, donde no solo aprendió a leer y escribir, sino que también fue capas de convertirse en un muy buen arquitecto, escultor y maestro para otros niños sordos\autocite[p.~13-15]{Fischer1993}. De Fay era capas de comunicarse con buenas habilidades a través del lenguaje de señas y también de educar a sus estudiantes utilizando tanto las señas como la escritura, demostrando que la educación podía llevar a la autonomía y la responsabilidad, es considerado el primer profesor para personas con discapacidad auditiva\autocite[p.~18]{Fischer1993}. Además, se considera que su vida representa una gran diferencia para la historia de los sordos en Francia, incluso siendo anterior a otra de las figuras más conocidas en esta parte de la historia, siendo L'Epée\autocite[p.~14]{Fischer1993}. El cual planteaba métodos diferentes de enseñanza en estas comunidades en particular. Por otro lado, la comunidad sorda siempre ha sufrido de ser marginada en varios aspectos desde periodos mucho más antiguos de lo que datan los registros y esta época no es la excepción, la mayoría de las personas sordas vivían en el aislamiento donde casi siempre eran confundidas con personas sin capacidad intelectual, y su acceso a la educación era prácticamente inexistente, a menos que pertenecieran a familias con muchos recursos que pudieran permitirse buenos tutores, marcando la separación entre la elite y las masas\autocite[p.~26]{Fischer1993}.\vspace{0.5cm}\par

Luego se realiza un salto en la historia a principios del siglo XIX, donde se puede evidenciar que el lenguaje de señas americano (ASL) está intrínsecamente ligado a la herencia francesa que se revisó anteriormente. Porque en 1817, gracias a la colaboración del reverendo estadounidense Thomas Hopkins Gallaudet y de Laurent Clerc, un muy buen profesor sordo del instituto de París, se pudo fundar el Asilo Americano para la Educación de Sordomudos (ahora conocida como la Escuela Americana para Sordos o ASD) en Hartford, Connecticut. El profesor Clerc no solo fue cofundador, sino que también el principal profesor, importando el Lenguaje de Señas Francés (LSF) al continente americano\autocite[p.~X]{Shaw2015}.\vspace{0.5cm}\par

El hecho de traer este lenguaje a un nuevo lugar provoco un cambio cultural en la ASD, donde el LSF de Clerc se logró mezclar con diversas formas de comunicación que ya utilizaban los estudiantes allí, quienes provenían de diferentes partes del país. Así mismo, entre estas formas de comunicación se encontraba el lenguaje de señas que ya existía en la isla de Martha's Vineyard. En este lugar, el lenguaje de señas se volvió una necesidad por una alta incidencia de sordera hereditaria que origino un lenguaje de señas utilizado tanto por personas sordas como oyentes. La combinación de estas diferentes corrientes, con sus bases predominantes en el LSF, dio origen a lo que hoy se conoce como ASL. Desde Hartford, el modelo educativo y el nuevo lenguaje de señas se expandieron rápidamente por todo Estados Unidos, donde los graduados de la ASD fundaban nuevas escuelas residenciales, asegurando de esta manera una notable uniformidad del ASL en las primeras generaciones de los hablantes\autocite[p.~X-XI]{Shaw2015}.\vspace{0.5cm}\par

Posteriormente, hubo un problema dentro de la comunidad hablante de señas debido al congreso de milán y la resistencia americana. Lo que sucedió, consistió en que Hacia finales del siglo XIX, la educación de la gente sorda se vio envuelta en un complicado debate filosófico que se originó entre los defensores del método manual, es decir los que usaban las señas, y los del método oral, que peleaban por enseñar a los sordos a hablar y leer los labios. Este problema alcanzó su punto más alto en el Segundo Congreso Internacional sobre la Educación de las personas sordas, el cual tuvo su origen en Milán en 1880. De esta manera, en el congreso, la mayoría de educadores, que cabe aclarar no poseían ninguna discapacidad, votó a favor de la prohibición del lenguaje de señas en la educación, lo que ocasiono la prevalencia del método oral sobre toda Europa\autocite[p.~XI]{Shaw2015}.\vspace{0.5cm}\par

Este suceso ocurrido en el edicto de Milán tuvo consecuencias devastadoras en Europa, llevando a la supresión casi en su totalidad del lenguaje de señas en las escuelas, además de la marginación de los profesores sordos y los estudiantes que usaran las señas, donde los vigilaban constantemente para evitar que lo hicieran, esto fue muy malo, pues era la manera en la cual podían enseñar y comunicarse. En ese momento, en Francia, la comunidad sorda tuvo que ver con desesperación cómo su lengua era borrada de las aulas durante décadas\autocite[p.~XI]{Shaw2015}. Sin embargo, en Estados Unidos la reacción fue bastante diferente. Aunque el oralismo ganó mucho terreno, la comunidad sorda de América, gracias a que estaba fortalecida por varias escuelas que enseñaban las señas por varios años, además de tener muchos exalumnos educados y organizaciones como la Asociación Nacional de Sordos (NAD) que se dedicaban a las señas, pudo crear una muy buena resistencia. Durante esta época de resistencia, en los próximos años 1904 al 1910, la NAD genero varias películas que mostraban la indignación de las personas ante los "falsos profetas" que impartían el oralismo, unificando a la comunidad sorda de América.\autocite[p.~XIII]{Shaw2015}. Por otro lado, factores como la distancia cultural a Europa, además de la autonomía que tenían las escuelas estatales, sin olvidar el apoyo de grupos religiosos que veían en las señas un medio eficaz para la evangelización permitieron que el ASL sobreviviera, aunque algunas veces de forma clandestina dentro de las escuelas que oficialmente adoptaban el oralismo. Pero prevaleciendo el hecho que en America aumento el uso de las señas comparando con Francia, esto se puede ver también en la manera en la que cambiaron los lenguajes a través de la historia\autocite[p.~XIV]{Shaw2015}.\vspace{0.5cm}\par

A pesar de este problema, a principios del siglo XX, en plena pelea que estaba en contra del oralismo, la comunidad sorda estadounidense empezó medidas bastante buenas para preservar su lengua. Como se mencionó antes, la NAD, bajo el liderazgo de su presidente George Veditz, emprendió un proyecto pionero, el cual consistía en grabar una serie de películas que luego se popularizarían entre los años 1910 y 1920 para documentar y preservar "la belleza y la gracia del lenguaje de señas" para las generaciones futuras\autocite[p.~3]{Supalla2015}. Estas películas no solo son un testimonio de la resistencia de la comunidad, sino también una gran oportunidad para admirar y apreciar la evolución del ASL, esto porque se dice porque en estas aparecen "maestros de las señas" de diferentes generaciones, mostrando las variaciones diacrónicas y sincrónicas del lenguaje a lo largo de los años. Por ejemplo, se puede observar cómo señas que originalmente eran compuestos de dos movimientos, como la seña de PADRE (una combinación de "hombre" y "engendrar"), se fueron reduciendo y simplificando con el tiempo hasta llegar finalmente a la forma moderna de un solo toque en la frente.\autocite[p.~4-5]{Supalla2015}\vspace{0.5cm}\par

Es en este mismo momento en que se logran publicar los primeros diccionarios de ASL, como los de J. Schuyler Long (1910) y Daniel D. Higgins (1923), con el objetivo de llegar a estandarizar y preservar la "pureza original" de la lengua frente a la creciente amenaza del oralismo. Este periodo, aunque difícil y conocido como la época oscura, demostró la resiliencia de la comunidad sorda y su profundo arraigo con su identidad lingüística y cultural.\autocite[p.~XIV]{Shaw2015}.\vspace{0.5cm}\par

Así mismo, esta resiliencia dio frutos porque pesar de la opresión, la comunidad sorda logro resistir. Es importante señalar, que el punto de inflexión más significativo del siglo XX llegó desde el punto de vista académico. Esto gracias a William C. Stokoe, el cual fue un lingüista estadounidense que en 1960, fue el primero en demostrar científicamente que el ASL no era algo simple y superficial o un código gestual, sino más bien una lengua verdadera y compleja, con su propia fonología, morfología y sintaxis\autocite[p.~XIV]{Shaw2015}. Con este análisis lingüístico estableció la riqueza del ASL e impulso en diferentes círculos académicos su reintroducción en la educación y sentó las bases para su reconocimiento académico.\vspace{0.5cm}\par

Gracias al trabajo de Stokoe, junto con el impulso que tuvo el movimiento por los derechos civiles en Estados Unidos, liderado por Martin Luther King, genero un cambio de percepción, donde Las personas sordas comenzaron a verse a sí mismas como una minoría lingüística y cultural haciendo que su lucha por los derechos se intensificara\autocite[p.~517]{Fischer1993}. Por otro lado, en Francia, este surgimiento de espíritu de lucha se conoció como el Réveil sourd (El despertar sordo), el cual consistió en un movimiento para reafirmar el Lenguaje de Señas Francés (LSF) después de un siglo de represión desde el Congreso de Milán\autocite[p.~XII]{Shaw2015}.\vspace{0.5cm}\par

Sin embargo, también estaban ocurriendo otros movimientos a nivel internacional, el movimiento por el reconocimiento de los lenguajes de señas ganó impulso en otros lados del mundo. Por ejemplo, en Alemania, un logro bastante grande fue en el Congreso de Hamburgo de 1985, donde se declaró por primera vez que el Lenguaje de Señas Alemán era un sistema lingüístico independiente y completo. Además, este evento fue un antes y después en la historia de la comunidad sorda alemana y significo después en la fundación del Centro de Lenguaje de Señas Alemán en Hamburgo en 1987. Por otro lado, también significo que la lucha política se intensificara, dando como resultado a los líderes sordos comenzando a exigir una participación activa en la educación de los niños sordos\autocite[p.~188-189]{Fischer1993}.\vspace{0.5cm}\par

Por último, la década de 1980 se conoció por culminar con dos eventos muy importantes. El primero de ellos, En 1988, fue que el Parlamento Europeo adoptó una resolución comunicando a los países miembros a reconocer legalmente sus respectivos lenguajes de señas nacionales\autocite[p.~204]{Fischer1993}. Y la segunda fue la victoria en la universidad Gallaudet en Rusia, donde los estudiantes protestaron con éxito para que se nombrara al primer presidente sordo de la universidad, este hecho se supo mucho después, luego de 75 años, por culpa de la dificultad que tenía ese país en dicha época, donde la información era bastante limitada y lo que le pasaba a la comunidad sorda de Rusia estaba totalmente aislado\autocite[p.~195]{Fischer1993}. De estos hechos a la actualidad, la lucha por los derechos de las personas sordas sigue en pie, donde a través de la tecnología se ha tratado de dar una solución, sin embargo, aún no se ha encontrado una respuesta definitiva como se puede apreciar en el siguiente apartado con la revisión histórica de la detección de señas.\vspace{0.5cm}\par

\subsection{Evolución de la Tecnología de Reconocimiento del Lenguaje de Señas}

Cuando se  analiza la evolución de la tecnología se pueden reconocer diferentes etapas,  que se diferencian por los avances en inteligencia artificial y visión por computador. Cuando se habla de la primera década de los últimos 20 años, se ve que en el campo del reconocimiento del lenguaje de señas, y también el del Lenguaje de Señas Chino (CSLR), se centró en su mayoría en aplicaciones y sistemas basados en sensores. Además, durante este período de experimentación, los investigadores tenían de métodos que requerían contacto físico o dispositivos que eran muy especializados. Donde la recolección de datos se realizaba a través de guantes de datos (data gloves) y sensores de movimiento como Kinect y Leap Motion\autocite[p.~3]{Jiang2024}.\vspace{0.5cm}\par

Siguiendo este orden de ideas, las tecnologías de clasificación que se veían de manera más común en esta etapa eran los Modelos Ocultos de Márkov (HMM), las Máquinas de Vectores de Soporte (SVM) y el Almacenamiento de Contraste Dinámico (DTW)\autocite[p.~3]{Jiang2024}. Donde estos métodos clásicos se aplicaban principalmente al reconocimiento de señas aisladas o al deletreo manual (dactilología), la particularidad de estos gestos es que se realizan de uno en uno en un entorno controlado, lo cual representaría un futuro problema donde hasta los métodos actuales se esfuerzan por poder reconocer los gestos en espacios continuos\autocite[p.~23]{Hashi2024}. No obstante, esta primera generación de sistemas presentaba limitaciones mu grandes, las cuales eran, el costo del reconocimiento, donde este era relativamente alto y la precisión era comparativamente baja. Sin olvidar que, la dependencia de tener sensores hacía que los sistemas fueran casi siempre invasivos y poco prácticos para el uso del día a día\autocite[p.~3]{Jiang2024}.\vspace{0.5cm}\par

Por otro lado, se provocó un cambio en las dinámicas, permitiendo que surgiera en mayor medida las técnicas por visión por computador y de aprendizaje profundo. Este cambio se originó en la última década, impulsado por el rápido desarrollo de la visión por computadora y las tecnologías de inteligencia artificial\autocite[p.~2]{Jiang2024}. Este período tiene la peculiaridad de haber marcado un cambio de paradigma desde los métodos basados en sensores hacia técnicas basadas en la visión, que utilizan cámaras para capturar el movimiento de las manos y el cuerpo sin necesidad de contacto para medir las distancias de los dedos con respecto a la posición de la mano\autocite[p.~3]{Jiang2024}.\vspace{0.5cm}\par

Es en esta  transición que ocurre algo importante, que sería la publicación del trabajo de Su et al. en 2016, quienes propusieron un método no visual para el reconocimiento del lenguaje de señas basado en acelerometría (ACC) y electromiografía de superficie (sEMG), utilizando un algoritmo de Random Forest que alcanzó una notable precisión del 98.25\% en la clasificación de 121 palabras del CSL\autocite[p.~12]{Jiang2024}. Casi al mismo tiempo de este hecho, el potencial del aprendizaje profundo se hizo mucho más evidente, gracias a que en 2017, Yang et al. utilizaron una Red Neuronal Convolucional (CNN) que junto a una segmentación de manos para verificar 40 palabras del vocabulario de señas, pudo lograr una tasa de reconocimiento del 99.00\%\autocite[p.~13]{Jiang2024}.\vspace{0.5cm}\par

Consecuentemente con lo anterior, estos avances demostraron que los enfoques basados en visión y aprendizaje profundo no solo podían igualar, sino que también superar la precisión de los anteriores sistemas basados en sensores, dándole paso a una nueva era de investigación. La introducción de arquitecturas innovadoras como el modelo Transformer por Vaswani et al. en 2017 y BERT por Devlin et al. en 2018 ayudaron a sentar las bases para modelos de lenguaje y reconocimiento mucho más potentes y contextuales.\autocite[p.~19]{Jiang2024}.\vspace{0.5cm}\par

Ahora bien, con las bases sentadas, se puede ver que desde 2018 hasta el presente, la investigación en SLR (reconocimiento de lenguaje de señas) ha entrado en una fase que se puede considerar de perfeccionamiento, con un enfoque particular en la aplicación de técnicas de aprendizaje profundo para abordar los aspectos más complejos del lenguaje de señas\autocite[p.~3]{Hashi2024}. Cuando se menciona el uso de la tecnología CNN, se evidencia que se ha generalizado desde 2019, y además ha sido complementado por arquitecturas más avanzadas como las 3D-CNN, Redes Neuronales Recurrentes (entre otras de sus variantes como LSTM), y los modelos Transformer, lo cuales son más adecuados para el reconocimiento de lenguaje de señas continuo debido a su capacidad para procesar la dinámica temporal y la información contextual\autocite[p.~28]{Jiang2024}.\vspace{0.5cm}\par

Así mismo, los enfoques de investigación actuales se centran en superar determinados desafíos clave. Como por ejemplo. uno de ellos es lograr un reconocimiento de lenguaje de señas que pueda ser independiente del intérprete, donde los marcos de aprendizaje profundo pudieron mostrar resultados buenos\autocite[p.~3]{Hashi2024}. De igual manera, se está reconociendo cada vez más la importancia de los componentes no manuales del lenguaje de señas, es decir, las expresiones faciales y la postura corporal, además de trabajar en la integración de estos elementos en las tecnologías para mejorar la precisión y robustez de los sistemas\autocite[p.~3]{Hashi2024}. Gracias a esto, se pudo potenciar a 2021 las investigaciones que destacan la integración de redes neuronales y de sensores equipables, como por ejemplo el uso de guantes de datos para el reconocimiento en tiempo real de movimientos dinámicos de los dedos\autocite[p.~11]{Hashi2024}.\vspace{0.5cm}\par

No obstante, a pesar de los avances que se han hecho hasta el día de hoy, todavía permanecen obstáculos significativos. Como por ejemplo, siendo este de los más importantes, la limitación de las bases de datos existentes, que casi siempre son pequeñas y carecen de diversidad, conteniendo solo alfabetos, números o un conjunto reducido de palabras\autocite[p.~1]{Hashi2024}. Gracias a la falta de datasets a gran escala y que sean diversos, especialmente enfocados para el lenguaje de señas continuo, es decir, en formato de videos o secuencias, dificulta el desarrollo de sistemas que puedan ser verdaderamente prácticos y, más aún, robustos\autocite[p.~23]{Hashi2024}. De igual manera, otros desafíos incluyen la necesidad de desarrollar mejores métodos híbridos de extracción de características para poder reducir la dimensionalidad de los datos brutos y la también capacidad que tienen los sistemas para poder manejar condiciones enfocadas al mundo real, como por ejemplo lo pueden ser oclusiones (cuando un objeto bloquea la vista de la mano) y variaciones en la iluminación o escena.\autocite[p.~5]{Hashi2024}

\section{Marco Referencial}

\subsection{Panorama general del reconocimiento de lenguaje de señas}

Como se ha podido evidenciar, la investigación en SLR (reconocimiento de lenguaje de señas) ha ido evolucionando de manera significa, pasando de métodos tradicionales a enfoques avanzados basados en inteligencia artificial, lo que muestra un progreso avanzado en la interacción entre el humano y el computador. Como se pudo apreciar en el apartado anterior con la evolución del campo, se puede segmentar en tres áreas principales de investigación que representan diferentes niveles de complejidad y desarrollo, siendo estas el reconocimiento de señas aisladas, el reconocimiento de señas continuas y por último la traducción del lenguaje de señas.\vspace{0.5cm}\par

En primera instancia, al evaluar las señas aisladas, se ve que esta es la categoría más fundamental del SLR y se enfoca meramente en la identificación de señas individuales que son ejecutadas de manera planeada y con pausas claras entre ellas. Cuando se aborda esta aproximación, se necesita que cada video o secuencia de datos contenga una única seña que debe ser clasificada en una categoría que ya está definida. Por esto es que este tipo de reconocimiento se considera análogo al reconocimiento de gestos estáticos, donde la información sobre la manera en que se presenta la mano es de suma importancia, o en su defecto a los gestos dinámicos simples cuyo movimiento sigue un patrón bien definido y establecido\autocite[p.~2]{Hashi2024}. Como se pudo ver con anterioridad, a lo largo de los años, esta tarea se pudo lograr con éxito usando métodos de aprendizaje automático tradicionales como los Modelos Ocultos de Márkov (HMM), las Máquinas de Vectores de Soporte (SVM) y la Deformación Dinámica del Tiempo (DTW)\autocite[p.~1]{Jiang2024}. Hay que tener en cuenta, a pesar de que el reconocimiento de señas aisladas pudo alcanzar altos niveles de precisión en vocabularios controlados, a la hora de la verdad, su aplicación es muy limitada en la comunicación natural y fluida, porque las personas hablantes no suelen hacer pausas entre cada seña y mucho menos deletrearlas en una conversación normal.\vspace{0.5cm}\par

Por otro lado, cuando se revisa el reconocimiento de señas continuas (CSLR), se puede ver en la revisión documental que representa un salto significativo en complejidad y representa el enfoque de la mayoría de las investigaciones actuales. Cuando se habla de este enfoque, se hace referencia a poder identificar y transcribir una secuencia de señas a partir de un video sin tener que realizarle algún corte, por este motivo, no hay delimitadores marcados entre una seña y la siguiente\autocite[p.~2]{Khan2025}. Por lo anterior, se considera que esta tarea es mucho más complicada y también se pueden identificar dos desafíos principales, siendo el primero de estos, la segmentación temporal, es difícil determinar solo usando código dónde termina una seña y dónde comienza la siguiente en parte por los movimientos de transición que no hacen el cambio de manera evidente\autocite[p.~3]{Khan2025}. Por otro lado, el segundo, es el efecto de co-articulación, donde la apariencia y ejecución de una seña se ven influenciadas por las señas que se hicieron antes y las que se harán después, esto añade una capa extra de variabilidad y complejidad\autocite[p.~14]{Khan2025}. Con esto en mente, para poder abordar estos desafíos, se ha identificado la utilización de los enfoques que están fundamentados en aprendizaje profundo, que utilizan arquitecturas como las Redes Neuronales Convolucionales (CNN) para la extracción de características espaciales y Redes Neuronales Recurrentes (RNN) o Transformers para el modelado temporal\autocite[p.~9]{Khan2025}\autocite[p.~11]{Khan2025}.\vspace{0.5cm}\par

Ahora bien, cuando se habla de la última categoría, que corresponde a la Traducción del Lenguaje de Señas, se dice que es la más complicada, porque diferencia del CSLR, que solo busca transcribir la secuencia de señas, la SLT tiene como objetivo generar una oración gramaticalmente correcta además de tener sentido dentro del contexto de un lenguaje hablado. Teniendo en cuenta lo anterior, se puede deducir que este proceso no solo requiere el reconocimiento preciso de las señas, sino también un profundo entendimiento de la gramática, la sintaxis y la estructura lingüística del lenguaje de señas de origen, así como del lenguaje hablado al que se quiera hacer la traducción. Como se ha podido evidenciar en la investigación, traducir de un lenguaje hablado a uno de señas no solo exige conocimiento del lenguaje, sino también una comprensión de la cultura y el contexto social de sus usuarios. Por esto mismo es que la SLT debe manejar tanto el hecho de que el orden de las señas no siempre corresponde directamente con el orden de las palabras en la oración traducida, como con el hecho de incorporar información crucial transmitida a través de todo el cuerpo del intérprete, como las expresiones faciales y el movimiento del cuerpo, que son parte integral del significado en el lenguaje de señas. En esta área, los casi todos los trabajos de la actualidad, plantean lo anterior como investigaciones futuras o como pasos que ya se han realizado para poder llegar a este ideal, como lo puede ser el reconocimiento de la importancia de capturar todo el intérprete en lugar de solo las manos o la integración de nuevas tecnológicas para encontrar la manera de capturar la temporalidad.\autocite[p.~2]{Akarsh2024}\autocite[p.~23]{Hashi2024}\autocite[p.~11]{Khan2025}.\vspace{0.5cm}\par

Ahora bien, con el panorama general analizado, se puede ubicar este proyecto de investigación dentro de la categoría de Reconocimiento de Señas Continuas (CSLR), con una posible extensión hacia los fundamentos de la Traducción del Lenguaje de Señas (SLT). Esto se puede decir por qué al plantear la naturaleza del lenguaje de señas como un \enquote{sistema lingüístico bastante completo y complejo} que posee \enquote{su propia gramática, vocabulario y estructura sintáctica}, este trabajo de investigación reconoce desde el principio los desafíos que van más allá de la simple clasificación de gestos\autocite[p.~3]{Khan2025}. Además, la investigación se alinea con la corriente principal del CSLR, que se basa en buscar el desarrollo de sistemas capaces de interpretar secuencias de señas en un flujo continuo con videos, enfrentando directamente los problemas de segmentación y co-articulación que caracterizan a esta área\autocite[p.~14]{Khan2025}. Sin embargo, al enfocarse en la estructura del lenguaje, el proyecto sienta las bases para una futura transición hacia la SLT, usando nuevas técnicas en un área que requiere más atención para poder desentrañar el potencial de las herramientas avanzadas de inteligencia artificial.

\subsection{Investigaciones particulares relevantes}

Como se pudo ver en la definición del problema, se presentaron los trabajos más influyentes que inspiraron y moldearon el desarrollo y elección de la técnica que se plantea en esta investigación. No obstante, también hubo otros estudios que de igual manera se enfocan en áreas de interés más particulares para el estudio que sirvieron para probar diferentes teorías más adelante en los análisis de resultados. Esta revisión se inicia con las primeras y más comunes arquitecturas particulares en el reconocimiento de lenguaje de señas y tareas relacionadas, como lo puede ser por ejemplo la lectura de labios, estas se basan en la combinación de Redes Neuronales Convolucionales (CNN) para la extracción de características espaciales y Redes Neuronales Recurrentes (RNN) para el modelado de la dependencia temporal.\vspace{0.5cm}\par

Un ejemplo claro de esta aproximación es el trabajo de Dey et al., quienes proponen una red para el reconocimiento de gestos de preguntas "Wh" en lenguaje de señas americano. En ese estudio, se optó por emplear una arquitectura de Red Convolucional 3D para poder capturar las características espaciales y temporales de bajo nivel directamente de los fotogramas de un video.\autocite[p.~2920]{Dey2024} Posteriormente, se emplea una BiLSTM (Long Short-Term Memory Bidireccional) la cual procesa estas características para entender la secuencia del gesto en ambos sentidos, es decir, hacia adelante y hacia atrás. De igual manera, se incorpora un mecanismo de atención multi-cabeza (Multi-head Attention) el cual permite al modelo establecer la importancia de diferentes partes de la secuencia, mejorando de esta forma, la capacidad de capturar características complejas y relevantes para un reconocimiento mucho más preciso\autocite[p.~2921]{Dey2024}.\vspace{0.5cm}\par

Por otro lado, en otro dominio, siendo más específico en el de la lectura de labios, con el estudio de Inamdar et al. se propone un modelo que también combina convoluciones 3D y una LSTM. Donde se aprovechaba este enfoque híbrido de las CNN 3D para aprender representaciones de video y las LSTM para modelar las secuencias temporales de los movimientos de los labios\autocite[p.~1]{Inamdar2023}. En otra mano, el estudio de Innocente et al. tiene la similitud de seguir esta misma línea para desarrollar un sistema de lectura de labios en italiano con el fin de asistir a pacientes con patologías en las cuerdas vocales\autocite[p.~1]{Innocente2025}. Sin embargo, su modelo consta de una CNN espacio-temporal, la cual se enfoca en un vocabulario específico y cuidadosamente seleccionado para cubrir necesidades de comunicación esenciales y de emergencia, demostrando la aplicabilidad de estas arquitecturas en diferentes entornos con éxito. Además, este estudio también resalta un desafío muy importante en el reconocimiento visual del habla, siendo este la ambigüedad visual donde movimientos de labios similares pueden corresponder a fonemas diferentes (homofenos)\autocite[p.~1]{Innocente2025}.\vspace{0.5cm}\par

También se han analizado enfoques más complejos que buscan superar las limitaciones de los modelos tradicionales con el fin de mejorar la calidad de las representaciones de video. Este cambio de perspectiva se puede ver en la propuesta de Geng et al., quienes abordan el Reconocimiento Continuo de Lenguaje de Señas (CSLR) no como un problema de alineamiento, sino como una tarea de generación de video a texto, teniendo en cuenta que el alineamiento se refiere al categorizar ciertos conjuntos de frames en señas específicas\autocite[p.~1]{Geng2025}. Argumentan que el alineamiento entre los fotogramas del video y las glosas (unidades léxicas del lenguaje de señas) es inherentemente propenso a errores debido a la naturaleza débilmente supervisada de los datos. Para evitar este paso, utilizan un modelo de difusión para generar la secuencia de glosas a partir de las características visuales extraídas. Este enfoque generativo, combinado con aprendizaje contrastivo y mecanismos de atención cruzada (cross-attention), permite al modelo aprender una relación más directa y robusta entre el video y el texto, obteniendo resultados competitivos\autocite[p.~3]{Geng2025}.\vspace{0.5cm}\par

Finalmente, cuando se evalúa el aprendizaje auto-supervisado, se presenta como una solución prometedora para que ya no sea drástica la dependencia a los datos etiquetados de manera masiva. Esto se puede ver con el estudio de Dave et al. el cual se enfoca en desarrollar un TCLR (Temporal Contrastive Learning for Video Representation), a través de un marco de aprendizaje contrastivo diseñado específicamente para datos de video\autocite[p.~1]{Dave2022}. A diferencia de los métodos anteriores, el de TCLR se conoce por introducir explícitamente pérdidas que hacen que el modelo pueda discriminar no solo entre diferentes videos, sino también entre clips no superpuestos dentro del mismo video. Además, esto hace que las características aprendidas sean diversas a lo largo de la dimensión temporal\autocite[p.~6]{Dave2022}. Como resultado, se tienen mejores representaciones de video que impulsan significativamente el rendimiento en tareas posteriores como el reconocimiento de acciones, incluso con etiquetas limitadas. Este enfoque es particularmente importante para el lenguaje de señas, donde se ve que la dinámica temporal y el poder diferenciar entre señas que son visualmente similares es fundamental\autocite[p.~7]{Dave2022}.

\subsection{Research Gap dentro del campo}

Luego de realizar la investigacion del campo, se puede evidenciar que a pesar de los avances significativos en el reconocimiento del lenguaje de señas (SLR), aún hay brechas críticas que no permiten el desarrollo de soluciones robustas y generalizables. Lo anterior se puede sustentar en que una de las principales carencias identificadas es el enfoque común en las tareas de reconocimiento en lugar de representación semántica, lo que se quiere decir con esto es que la mayoría de los trabajos revisados, desde modelos híbridos como C3D-BiLSTM con atención hasta enfoques generativos basados en difusión, están optimizados para la transcripción continua y una posible traducción en el futuro, pero no para el análisis existente de las relaciones semánticas entre señas\autocite[p.~2921]{Dey2024}\autocite[p.~3]{Geng2025}. Lo que se ha evidenciado también es que estas representaciones internas son utilizadas en pro del entrenamiento y no como objetivo principal, es decir, están diseñadas para maximizar la precisión en clasificación, no para organizar el espacio latente según similitudes. Por otro lado, lo que se quiere lograr con este proyecto, se centra explícitamente en la estructuración semántica de palabras en un espacio de baja dimensionalidad, un cambio de paradigma desde la traducción hacia la representación, creando nuevos posibles campos y aproximaciones que se pueden estudiar en el futuro.\vspace{0.5cm}\par

Además, otra brecha detectada es la limitación de los trabajos a contextos monolingües. Es decir, los estudios como los de Jiang (CSL) o Gu (ASL) operan exclusivamente sobre datasets específicos (RWTH-PHOENIX-Weather 2014T, AQSVd), ignorando la diversidad dialectal y la falta de estándares universales en lenguajes de señas al rededor del mundo\autocite[p.~1-40]{Jiang2024}\autocite[p.~1-15]{Gu2024}. Así mismo, como señala Bedoin, los lenguajes de señas varían regionalmente, y su tratamiento como sistemas homogéneos limita su aplicabilidad real a diferentes contextos\autocite[p.~166]{Bedoin2024}. Por eso mismo es que este proyecto aborda este vacío al explorar un espacio latente compartido para tres lenguajes (ASL, indio y ruso), donde señas equivalentes puedan agruparse semánticamente, sentando bases para una futura unificación.\vspace{0.5cm}\par

Siguiendo con lo detectado durante el análisis, resulta llamativo encontrar campos de otras disciplinas que han implementado con éxito soluciones de inteligencia artificial que pueden ser muy valiosas y aún no se han probado en el de lenguaje de señas. Como por ejemplo, la lectura de labios o la rehabilitación motriz donde se han desarrollado técnicas innovadoras que podrían adaptarse, como por ejemplo el trabajo de Inamdar et al. el cual combina CNN 3D y LSTM para modelar movimientos labiales en pacientes con patologías vocales\autocite[p.~1]{Inamdar2023}, mientras que, por otro lado, Innocente et al. emplean arquitecturas espacio-temporales con capas bidireccionales para vocabularios médicos esenciales\autocite[p.~1]{Innocente2025}. A lo que se quiere llegar es que estas aproximaciones centradas en capturar dinámicas temporales y ambigüedades visuales (homofenismos), tienen la particularidad de ser directamente aplicables al SLR, donde la co-articulación y las expresiones faciales son obstáculos que han sido detectados en múltiples escenarios.\vspace{0.5cm}\par

Por último, se detectó un vacío arquitectónico y metodológico en técnicas determinadas. Por esto es que el núcleo de esta investigación reside en una combinación de técnicas que no ha sido explorada anteriormente, esta se compone en primer medida de un Autoencoder Convolucional 3D, el cual es ideal para comprimir videos de señas en representaciones latentes, preservando información espacio-temporal. El cual es evaluado por una función de pérdida compuesta, que se compone de MSE, el cual garantiza reconstrucción confiable de las señas. Que está acompañada de una función Triplet Loss, que su funcionamiento es estructurar el espacio latente mediante distancias semánticas (anclas, positivos y negativos). Acompañada también por una divergencia KL, la cual se encarga de regularizar la distribución latente para evitar sobreajuste y permitir interpolación entre señas. Se dice que no ha sido explorada, porque esta arquitectura difiere de enfoques como TCLR (aprendizaje contrastivo temporal), que diferencian clips pero no organizan el espacio semánticamente\autocite[p.~6]{Dave2022}. Además, de integrar investigaciones con otros objetivos como de modelos médicos, como puede ser la lectura de labios, o de producción de señas, como el modelo G2P-DDM para generación de poses\autocite[p.~6234]{Xie2024}, pero organizada bajo un único objetivo, el cual es poder mapear señas multilingües en un espacio estructurado.\vspace{0.5cm}\par

En conclusión, se pudo identificar que las brechas observadas revelan que el SLR está lejos de ser perfeccionado o estar en una etapa final. Esto por los diferentes enfoques centrados en representación, no solo reconocimiento. Además de no tener modelos multilingües destacables que capturen diversidad dialectal. Sin mencionar que faltan arquitecturas híbridas que combinen técnicas de aprendizaje autosupervisado y regularización semántica bajo funciones de perdida determinadas. Sin duda alguna, este proyecto llena un vacío crítico al proponer un marco para analizar y visualizar relaciones entre señas de distintos lenguajes, utilizando recursos limitados pero estratégicos (datasets ASL, indio y ruso). Donde los resultados podrían guiar futuras investigaciones hacia sistemas de traducción universal más inclusivos, alineados con las necesidades reales de las comunidades sordas\autocite[p.~851]{Adler2025}.\vspace{0.5cm}\par

\section{Marco Conceptual}

\subsection{Fundamentos Lingüísticos y Culturales del Lenguaje de Señas}

En esta subsección se aborda la naturaleza del lenguaje de señas visto como un sistema lingüístico complejo y además el pilar de una cultura compleja.

\begin{itemize}

\item Lenguaje de Señas como Sistema Lingüístico.\vspace{0.5cm}\par

Es un sistema de comunicación natural, completo y complejo, utilizado por las comunidades sordas. No obstante, no consiste en una simple serie de gestos, sino que más bien en poseer todos los componentes de una lengua, siendo estos la gramática, sintaxis y un léxico rico. Además, su estudio requiere un enfoque interdisciplinario que integre la lingüística con el análisis cultural para su mejor comprensión.

\item Parámetros Constituyentes.\vspace{0.5cm}\par

Son los cinco componentes fundamentales que constituyen una seña individual. Es decir, las características que dan información respecto a ella, que cuando se da una alteración de cualquiera de estos parámetros, se puede cambiar por completo el significado de la seña.

\textbf{\small Configuración Manual: }La forma específica que adopta una mano al realizar una seña, como por ejemplo, el puño cerrado o los dedos extendidos.

\textbf{\small Ubicación: }El lugar en el cuerpo o, en su defecto, en el espacio donde se ejecuta la seña, como por ejemplo en la frente o en el pecho.

\textbf{\small Movimiento: }La acción o trayectoria que realizan las manos o partes del cuerpo durante la ejecución de la seña, como lo pueden ser el movimiento circular o la línea recta.

\textbf{\small Orientación de la Palma: }La dirección hacia la cual apunta la palma o los dedos de la mano, como por ejemplo, hacia arriba, hacia abajo o incluso hacia el intérprete.

\textbf{\small Componentes no Manuales: }Incluyen expresiones faciales, como lo puede ser una ceja levantada para una pregunta, movimientos de la cabeza, la boca o el torso. Se evidencia que estos componentes son muy importantes, ya que son capaces de cumplir un papel importante en la interpretación, pudiendo añadir matices emocionales o hasta intensificar el significado.

\item Diversidad Lingüística del Lenguaje de Señas.\vspace{0.5cm}\par

\textbf{\small Mito del Lenguaje de Señas Universal: }Se refiere a la creencia errónea que se tiene de que existe un único lenguaje de señas para todas las personas sordas del mundo. De hecho, en realidad, existen cientos de lenguajes de señas distintos, siendo más de 200 documentados como se ha visto en la investigación, donde cada uno cuenta con su propia historia y evolución, como por ejemplo la evolución histórica que se estudió del Lenguaje de Señas Americano con relación al Lenguaje de Señas Francés.

\textbf{\small Dialectos y Regionalismos: }Al igual que los idiomas hablados, los lenguajes de señas presentan variaciones en su dialecto. Esto quiere decir que en una misma seña se puede tener formas ligeramente diferentes o pueden existir señas distintas y que estas tengan un mismo concepto dependiendo de la región geográfica, como por ejemplo las diferencias en el lenguaje entre Bogotá y la costa.

\item Cultura Sorda\vspace{0.5cm}\par

Esta es el conjunto de creencias, comportamientos, arte, tradiciones literarias, historia y valores compartidos por las comunidades de personas que se han visto afectadas por una discapacidad auditiva. Esto quiere decir que el lenguaje de señas no es solo una herramienta de comunicación, sino el pilar central que porta y preserva esta identidad cultural.

\item Bilingüismo Bimodal y Multilingüismo\vspace{0.5cm}\par

\textbf{\small Bilingüismo Bimodal: }Se refiere al dominio y uso de un lenguaje de señas, es decir, en la modalidad de gestos, y un lenguaje hablado o escrito, es decir, en modalidad oral o auditiva, por parte de un usuario.

\textbf{\small Multilingüismo en Comunidades Sordas: }Esto quiere decir que se reconoce, particularmente en contextos multiculturales o en casos como los migratorios, donde las personas sordas pueden llegar a dominar múltiples lenguajes de señas o, en su defecto, combinaciones de estos con varias lenguas orales.

\end{itemize}

\subsection{Perspectivas sobre Discapacidad, Accesibilidad e Inclusión}

En esta subsección se enmarca el proyecto dentro de un contexto social, destacando de esta manera las barreras que enfrentan las personas sordas y también la importancia de las soluciones tecnológicas.

\begin{itemize}

\item Modelo Social de la Discapacidad\vspace{0.5cm}\par

Es una perspectiva que entiende lo que es la discapacidad no como una deficiencia inherente al individuo, sino más bien como el resultado de las barreras sociales, culturales, actitudinales y físicas que pueden impedir la participación plena de las personas, por ejemplo, la exclusión de una persona sorda no se debe a su sordera, sino más bien a la falta de intérpretes, de tecnologías accesibles o de conciencia social.

\item Accesibilidad y Diseño Universal\vspace{0.5cm}\par

\textbf{\small Accesibilidad: }Se refiere al diseño de productos, servicios o entornos para que puedan ser utilizados por el mayor número posible de personas, independientemente de sus capacidades. Sin embargo, en el contexto tecnológico, esto implica crear herramientas que eliminen las barreras de comunicación.

\textbf{\small Diseño Universal: }Va un paso más allá de la accesibilidad, donde realmente se enfoca en la creación de soluciones que atiendan a la diversidad humana desde su creación, sin la necesidad de adaptaciones posteriores, como lo puede ser una app que está diseñada desde el inicio con opciones de visualización bastante flexibles que no requieran de más actualizaciones.

\end{itemize}

\subsection{Barreras de Accesibilidad e Impacto de la Falta de Conciencia}

Esta subsección menciona la falta de recursos accesibles y de conciencia sobre la cultura sorda, lo cual genera barreras sistemáticas que impactan negativamente la vida de las personas sordas en ámbitos académicos, profesionales y de salud. Esto significa en su exclusión, falta de independencia, y puede generar sentimientos de frustración, ansiedad y agotamiento emocional.

\subsection{Contexto Histórico de la Educación de Sordos}

Esta subsección se encarga de añadir una perspectiva histórica muy importante para entender los conflictos y valores dentro de la comunidad sorda.

\begin{itemize}

\item Oralismo vs. Método Manual\vspace{0.5cm}\par

\textbf{\small Oralismo: }Es un enfoque educativo, históricamente dominante tras el Congreso de Milán de 1880, el cual prioriza la enseñanza del habla y la lectura de labios, frecuentemente prohibiendo el uso del lenguaje de señas.

\textbf{\small Método Manual: }Es el que defiende el uso del lenguaje de señas y lo establece como el método principal y más natural para la educación de las personas con alguna discapacidad.

\item Resiliencia Comunitaria\vspace{0.5cm}\par

Se refiere a las estrategias de las comunidades sordas para preservar su lengua y cultura frente a la opresión, como la ocurrida frente al oralismo, como se puede apreciar en las películas de la National Association of the Deaf en EE.UU., que se encargaron de documentar el ASL para futuras generaciones.

\end{itemize}

\section{Marco Teórico}

\subsection{Teoría del Aprendizaje por Representación (Pilar Central)}
 
Cuando se habla del aprendizaje por representación, se hace referencia a que es el paradigma fundamental de esta investigación. Esto se dice porque su objetivo no es solo predecir, sino más bien aprender transformaciones de los datos que extraigan información útil para la realización de tareas. Cuando se toma esta aproximación, ya no se enfoca en operar sobre los píxeles brutos de un video, es decir, que el modelo aprende una nueva representación en un espacio vectorial de menor dimensionalidad, conocido como espacio latente.\vspace{0.5cm}\par

En esta teoría se establece la piedra angular del proyecto, ya que el problema se enfoca explícitamente en \enquote{el comportamiento y la organización de las palabras dentro de un determinado espacio vectorial de baja dimensionalidad}. Asi mismo, la elección de un autoencoder como núcleo de la arquitectura es una consecuencia directamente a esta teoría, se toma esta decisión también teniendo en cuanta que el modelo se entrena no para clasificar, sino más bien en comprimir una seña en una representación latente y luego reconstruirla. Por ende, la calidad y estructura de esta representación es el verdadero objetivo, pues se teoriza que en este espacio las relaciones semánticas entre señas, incluso de diferentes idiomas, se harán evidentes.

\subsection{El Principio del "Information Bottleneck" (IB) y la Teoría de Modelos Generativos}

En un principio, el principio del "Information Bottleneck" (IB) muestra que una representación ideal debe ser un "cuello de botella" para la información, esto quiere decir que se debe comprimir al máximo la entrada reteniendo solo la información que es relevante. Cuando se transporta al contexto del autoencoder, esto se puede traducir en un equilibrio entre compresión, una representación simple, y preservación de la información, una reconstrucción precisa. Por lo tanto, este proyecto integra el IB a través de la función de pérdida compuesta, es decir que el Error Cuadrático Medio (MSE) asegura la fidelidad de la reconstrucción, mientras que la Divergencia de Kullback-Leibler (KL) la fuerza a que las representaciones latentes sigan una distribución simple, actuando como un regularizador que estructura el espacio y evita la memorización.\vspace{0.5cm}\par

Por otro lado, la teoría de los Modelos Generativos, muestra que al utilizar un Autoencoder Variacional, el modelo no es solo de compresión, sino más bien generativo. Esto significa que no solo aprende a codificar y decodificar, sino que también aprende la distribución de probabilidad subyacente de los datos. Con esta perspectiva se tiene un enfoque más poderoso, ya que este implica que el modelo entiende todo lo que compone de una seña, pudiendo teóricamente generar nuevas instancias de señas visualmente coherentes. Además, esta capacidad generativa sirve como una validación para mostrar que el espacio latente ha sido capas de capturar la estructura fundamental de los datos.

\subsection{Teoría del Aprendizaje por Transferencia (Transfer Learning)}

Cuando se habla de esta teoría, se dice que es una adición fundamental para poder justificar el objetivo de la unificación global. Esto se puede afirmar, porque el Transfer Learning se enfoca en aprovechar el conocimiento adquirido en una tarea o dominio para mejorar el rendimiento en otro.\vspace{0.5cm}\par

Con lo anterior claro, se puede traer a colación el problema de la escasez de datasets etiquetados a gran escala, siendo esta una limitación crítica en el campo. Por otro lado, no es viable crear un dataset masivo para cada uno de los 200+ lenguajes de señas, por este motivo, la teoría del aprendizaje por transferencia, es ideal al entrenar el autoencoder en un lenguaje de señas de gran escala usando aprendizaje auto-supervisado. Donde la importancia reside en esta fase, en la cual el modelo aprenderá características universales sobre el movimiento humano y la gesticulación.

\subsection{Teorías de Aprendizaje Auto-Supervisado (SSL) y Métrico (Metric Learning)}

Cuando se habla de estas dos teorías, se puede ver que ambas definen la estrategia de entrenamiento para aprender representaciones semánticas sin necesidad de etiquetas masivas.\vspace{0.5cm}\par

Con más detalle, se ve que el aprendizaje Auto-Supervisado (SSL) es una respuesta estratégica a la escasez de datos etiquetados a nivel de frame. Donde el propio dato de entrada proporciona la supervisión, en cambio, la elección de la reconstrucción de video como tarea de pretexto es una implementación directa de SSL. Al utilizarlo, esto permite que el modelo aprenda características espaciotemporales de manera autónoma.\vspace{0.5cm}\par

Por otro lado, el aprendizaje Métrico y la Función de Pérdida Triplet es el mecanismo clave para poder organizar el espacio latente por su significado. Mientras que el SSL aprende el cómo se ve una seña, el aprendizaje métrico le enseña al modelo, qué significa. Como se puede ver, la Pérdida Triplet estructura el espacio de una manera que las representaciones de señas semánticamente similares, siendo estas el ancla y el positivo, por ejemplo "gracias" en ASL y "gracias" en LSC, estén más cerca que las de señas no relacionadas, siendo estas el ancla y el negativo, por ejemplo "gracias" y "lunes" en ASL. Por otra parte, esta teoría es muy importante para poder superar las variaciones o ruido superficial, como lo puede ser diferente, iluminación, ropa del intérprete, o incluso el idioma, para poder agrupar las señas por su equivalencia semántica.

\subsection{Modelado de Características Espacio-Temporales y la Hipótesis del Múltiple}

Por otro lado, estas teorías justifican la arquitectura de red neuronal específica que se eligió para el proyecto. Siendo esta el modelado de características espaciotemporales, donde el lenguaje de señas se caracteriza por ser intrínsecamente dinámico y para poder abordar estas características se emplea una arquitectura híbrida. La cual se conforma por un autoencoder convolucional 3D (3D-CNN), el cual extrae características locales que son capaces de fusionar el espacio y tiempo a corto plazo, es decir la forma de la mano mientras se mueve. Y una red neuronal recurrente (GRU) bidireccional, la cual se encarga de modelar las dependencias temporales a largo plazo, capturando de esta manera el contexto completo de la seña al procesar la secuencia de características en ambas direcciones. Teniendo en cuenta lo anterior, esta arquitectura determinada está teóricamente diseñada para ser capas de poder capturar desde los movimientos más pequeños hasta la estructura narrativa más completa de un gesto.\vspace{0.5cm}\par

Por otro lado, también se tiene la hipótesis del múltiple (Manifold Hypothesis), donde esta hipótesis señala que los datos de alta dimensionalidad del mundo real, como lo pueden ser los píxeles de un video en este caso, en realidad se encuentran en una estructura subyacente de baja dimensionalidad (un manifold). Este proyecto se basa en la presunción de que todos los videos posibles de la seña \enquote{aprender}, sin importar sus muchos píxeles que la componen, se pueden agrupar en una pequeña región de este manifold. Donde, en este caso, el objetivo del autoencoder es precisamente descubrir y parametrizar este manifold, utilizando las herramientas de visualización como UMAP y t-SNE. Por tanto, estas últimas herramientas, cuentan como métodos para validar empíricamente esta hipótesis, permitiendo analizar si el modelo ha sido capas de aprender una estructura coherente donde las señas se organizan de manera lógica.

\section{Marco Tecnológico y científico}

\subsection{Tecnología para el Procesamiento del Lenguaje de Señas}

\begin{itemize}

\item Reconocimiento de Lenguaje de Señas por Computadora\vspace{0.5cm}\par

\textbf{\small Reconocimiento de Señas Aisladas (ISLR): }Es la tarea de identificar señas individuales que se realizan con pausas claras y marcadas entre ellas. Además, es una aproximación más sencilla, pero limitada a determinados entornos controlados.

\textbf{\small Reconocimiento Continuo de Lenguaje de Señas (CSLR): }Esta es la tarea de transcribir una secuencia continua y fluida de señas, la cual puede ser una frase, una conversación o inclusive, como en el contexto de este proyecto, palabras que están compuestas por varios movimientos, a partir de un video. Esta es mucho más desafiante debido a la falta de pausas y la co-articulación.

\textbf{\small Traducción del Lenguaje de Señas (SLT): }Por otro lado, esta representa el objetivo final de muchos trabajos, que no solo reconoce las señas, sino que las convierte a texto, voz u otro medio de comunicación, en un idioma hablado, respetando las diferencias gramaticales y sintácticas entre ambas lenguas.

\item Desafíos Fundamentales en CSLR\vspace{0.5cm}\par

\textbf{\small Alineamiento Débilmente Supervisado: }Este representa un problema bastante común donde se dispone de un video y su transcripción textual completa, pero no de una correspondencia exacta y minuciosa, siendo este dicho el alineamiento entre cada fotograma y la seña específica, dificultando el entrenamiento de los modelos.

\textbf{\small Segmentación Temporal: }Esta es la dificultad de identificar dónde termina una seña y comienza la siguiente en un flujo continuo.

\textbf{\small Co-articulación: }Este es el fenómeno donde la ejecución de una seña es influenciada por las señas que van antes y después.

\textbf{\small Limitaciones de Datasets: }Es la escasez de conjuntos de datos públicos, grandes, diversos y de alta calidad, que representen todos dialectos y variaciones culturales.

\textbf{\small Robustez en Condiciones Reales: }Es la dificultad de los sistemas para manejar oclusiones, es decir, manos bloqueadas por algún otro elemento, variaciones de iluminación, fondos complejos y diferencias individuales en la manera de realizar las señas.

\end{itemize}

\subsection{Tecnología para el Procesamiento del Lenguaje de Señas}

\begin{itemize}

\item Redes Neuronales para Extracción de Características\vspace{0.5cm}\par

\textbf{\small Red Neuronal Convolucional (CNN): }Está diseñada para procesar datos en formato de rejilla como las imágenes. También se utiliza para la extracción de características espaciales, como formas o texturas de los fotogramas de un video.

\textbf{\small Red Neuronal Convolucional 3D (3D-CNN o C3D): }Esta es una configuración de la CNN que tiene la particularidad de operar sobre volúmenes de datos que tienen el formato alto x ancho x tiempo, que además tiene la capacidad de capturar simultáneamente características espaciales y temporales, haciéndola ideal para el movimiento entre fotogramas y para el análisis de gestos dinámicos.

\item Redes Neuronales para el Modelado de Secuencias\vspace{0.5cm}\par

\textbf{\small Red Neuronal Recurrente (RNN): }Esta está diseñada para procesar datos secuenciales al tener una memoria que le permite usar información de pasos anteriores para cambiar la salida actual.

\textbf{\small Red Neuronal Long Short-Term Memory (LSTM) y BiLSTM: }Esta red tiene la particularidad de estar diseñada para procesar datos secuenciales, que como sucede con la RNN, tiene una memoria que le permite usar información de pasos anteriores para cambiar la salida actual. Sin embargo, lo que diferencia a la LSTM es que es una RNN avanzada que puede resolver el problema de aprender dependencias a largo plazo. Por otro lado, una BiLSTM es aquella que procesa la secuencia en ambas direcciones, es decir, hacia adelante y hacia atrás, proporcionando de esta manera un contexto temporal más completo.

\item Arquitecturas de Aprendizaje de Representación\vspace{0.5cm}\par

\textbf{\small Autoencoder: }Es una arquitectura no supervisada que aprende a comprimir datos, es decir, una codificación en una representación de baja dimensionalidad llamada espacio latente, para luego reconstruir la entrada original, es decir una decodificación. Donde su objetivo es el aprendizaje de características y la reducción de dimensionalidad.

\textbf{\small Autoencoder Convolucional 3D (3D-CAE): }Es una implementación específica del autoencoder que usa capas convolucionales 3D para poder comprimir videos, capturando de esta manera patrones espaciotemporales de una manera eficiente.

\textbf{\small Transformer: }Esta es una arquitectura avanzada que utiliza mecanismos de atención para poder procesar contextos largos y capturar de esta manera relaciones complejas entre elementos de una secuencia, superando algunas limitaciones de las RNNs.

\item Mecanismo de Atención (Attention Mechanism)\vspace{0.5cm}\par

Este es un componente importante que permite a un modelo neuronal ponderar de manera dinámica la importancia de diferentes partes de una secuencia de entrada al generar una salida determinada. Es decir, en lugar de tratar todos los fotogramas o características por igual, el modelo aprende a prestar atención a los segmentos que tienen más relevancia.

\end{itemize}

\subsection{Estrategias de Aprendizaje, Optimización y Evaluación}

\begin{itemize}

\item Paradigmas de Aprendizaje\vspace{0.5cm}\par

\textbf{\small Aprendizaje Auto-Supervisado (Self-Supervised Learning): }Este es un paradigma donde el modelo puede aprender representaciones significativas directamente de los datos sin la necesidad de etiquetas manuales. Además, esta tarea se logra mediante la creación de tareas pretexto, como por ejemplo predecir un frame determinado que está más adelante en la secuencia.

\textbf{\small Aprendizaje Contrastivo (Contrastive Learning): }Este es un enfoque del aprendizaje auto-supervisado que se encarga de enseñar al modelo la creación un espacio de representación donde las muestras similares, como lo pueden ser dos clips de la misma seña, están juntas, y las muestras diferentes, como lo pueden ser videos de señas diferentes, están separadas.

\textbf{\small Temporal Contrastive Learning (TCLR): }Esta es una implementación específica para videos que se encarga de asegurarse que el modelo pueda aprender características diversas a lo largo del tiempo de entrenamiento.

\item Función de Pérdida Compuesta (Composite Loss Function)\vspace{0.5cm}\par

\textbf{\small Error Cuadrático Medio (MSE): }Esta metrica, es la que mide la diferencia promedio al cuadrado entre la entrada original y la reconstruida por el autoencoder, forzando una reconstrucción fiel.

\textbf{\small Pérdida Triplet (Triplet Loss): }Esta es una función de perdida que opera sobre un triplete, el cual está constituido de una ancla, un elemento positivo y uno negativo para estructurar semánticamente el espacio latente, minimizando de esta manera la distancia entre señas de la misma clase y maximizando la distancia entre señas de clases diferentes.

\textbf{\small Divergencia de Kullback-Leibler (KL): }Esta se encarga de desempeñar un papel como un término de regularización para forzar que el espacio latente se ajuste a una distribución de probabilidad conocida como lo puede ser la normal, resultando en un espacio más suave y bien organizado, evitando el sobreajuste.

\item Técnicas de Reducción de Dimensionalidad y Visualización\vspace{0.5cm}\par

\textbf{\small UMAP / t-SNE: }Estos son algoritmos utilizados para poder visualizar el espacio latente de alta dimensionalidad en un gráfico 2D o 3D. Se usan en el proyecto, ya que tienen la particularidad de permitir la inspección visualmente de si el modelo ha logrado agrupar señas similares.

\end{itemize}

    \let\cleardoublepage\clearpage
    
    % Capitulo 1
    \chapter{Capítulo 1: Preprocesamiento y organización de los datos}
    \section{Estado Inicial de los Datos}

\subsection{Origen y propósito de los Datos}

Luego de realizar la búsqueda exhaustiva que se menciona en la delimitación del problema, se encuentra que los candidatos ideales se redujeron a 3. Estos datasets se encuentran representados en los lenguajes de señas, inglés, indio y ruso.\vspace{0.5cm}\par

En primera estancia, con el desarrollo del proyecto se utilizó el conjunto de datos en Inglés, WLASL (Por sus siglas Word-Level American Sign Language). Así mismo, este conjunto de datos se considera que es actualmente el mayor repositorio de videos disponible públicamente para el reconocimiento de palabras individuales en el Lenguaje de Señas Americano (ASL) además de contener un vocabulario con 2,000 señas comunes\autocite{WLASL2020}.\vspace{0.5cm}\par

Por otro lado, el propósito más importante detrás de la creación del Dataset WLASL fue el facilitar la investigación en el campo de la comprensión del lenguaje de señas. Teniendo en cuenta lo anterior, se buscó desarrollar tecnologías que eventualmente pudieran mejorar la comunicación entre las comunidades sordas y también las oyentes, queriendo responder a la necesidad de herramientas más precisas para el reconocimiento y traducción de señas.\vspace{0.5cm}\par

A la hora de realizar la recolección de los datos, estos fueron recopilados de dos fuentes principales de internet, siendo la primera de estas, sitios web educativos, donde se extrajeron videos especializados en lenguaje de señas, como ASLU y ASL-LEX. La segunda fuente consistió en videos tutoriales de la plataforma YouTube, seleccionando solo los videos con títulos que describían de manera explícita y clara la seña mostrada\autocite{KaggleWLASL2025}.\vspace{0.5cm}\par

Por último, en la recolección de los datos, para poder asegurar que el conjunto de datos se enfocara exclusivamente en palabras individuales, se aplicó un criterio de filtrado estricto. El cual consistió en el descarte de todos los videos en los que la etiqueta o anotación de la seña estuviera compuesta por más de dos palabras en inglés.\vspace{0.5cm}\par

Es importante hacer la aclaración que el conjunto de datos WLASL fue desarrollado por Dongxu Li y Hongdong Li. Se especifica que su uso está restringido a fines académicos y computacionales, prohibiéndose de esta manera cualquier tipo de explotación comercial, el dataset se distribuye bajo la licencia Computational Use of Data Agreement (C-UDA).\vspace{0.5cm}\par

Por otro lado, el siguiente se denomina INCLUDE Indian Lexicon Sign Language Dataset. Este conjunto de datos fue desarrollado para abordar la carencia de un dataset público y estandarizado de Lenguaje de Señas Indio (ISL), lengua que es utilizada por más de 5 millones de personas sordas en India\autocite{Sridhar2020}. Donde el objetivo principal de su creación fue proporcionar un recurso robusto para poder entrenar y evaluar diferentes modelos de Reconocimiento de Lenguaje de Señas (SLR).\vspace{0.5cm}\par

Además, el dataset fue creado influenciado por la iniciativa AI4Bharat y también se detalla en su publicación académica que  para hacer la recopilación de los videos se contó con la colaboración de estudiantes sordos de la St. Louis School for the Deaf en Adyar, Chennai, quienes son intérpretes experimentados, garantizando de esta manera que las señas grabadas se asemejen a las condiciones de comunicación que se tienen en el día a día.\vspace{0.5cm}\par

No obstante, el conjunto de datos completo está conformado por un total de 4,292 videos, los cuales abarcan 263 señas de palabras distintas que pertenecen a 15 categorías diferentes, donde cada video muestra la acción de una única seña.\vspace{0.5cm}\par

En última estancia se tiene el dataset SLOVO, el cual es un repositorio de videos a gran escala que está enfocado en el Lenguaje de Señas Ruso (RSL). Por otro lado, sus creadores alegan que el origen del dataset surge de la necesidad de contar con recursos específicos para cada lengua de señas, ya que estas varían significativamente entre países, y a la dificultad general de recopilar este tipo de datos\autocite{Kapitanov2023}.\vspace{0.5cm}\par

Por esto mismo es que el dataset contiene 20,400 videos que representan 1,000 gestos o señas distintas, las cuales están interpretadas por un total de 194 intérpretes. Por esta razón, el tamaño total del conjunto de datos es de aproximadamente 16 GB, con una duración acumulada de video de 9.2 horas donde la calidad de las grabaciones es alta, al tener un 65\% de los videos en resolución FullHD y el resto en HD.

\subsection{Estructuración Original de los Datos}

Teniendo más claro el origen y motivo detrás de cada conjunto de datos, se analiza a mayor escala la estructura de cada uno de estos, al estar almacenados, grabados y etiquetados de diferentes maneras.\vspace{0.5cm}\par

El dataset WLASL viene organizado de la siguiente manera, hay una carpeta que contiene todos los videos, donde cada video tiene su id en el nombre. Un archivo JSON el cual contiene toda la información relacionada de cada video, como lo puede ser la etiqueta, número de frames, entre otros datos. Un archivo de texto que contiene los id´s de los videos que no están asociados a alguna etiqueta. Otro archivo TXT que contiene todas las etiquetas que existen en el dataset.\vspace{0.5cm}\par

Sin embargo, el dataset INCLUDE - ISL está estructurado de una manera diferente, está compuesto por diferentes carpetas anidadas donde se divide según el tipo de palabra o la categoría a la que puede pertenecer. Por ejemplo, la carpeta \enquote{adjetivos\_1de5} contiene otra carpeta llamada \enquote{Adjetivos} que a su vez tiene diferentes carpetas de adjetivos como \enquote{1. Ruidoso} que contienen todos los videos que tienen esa etiqueta.\vspace{0.5cm}\par

Por último se tiene el dataset SLOVO que contiene la carpeta de los videos, donde cada video está guardado con un serial específico y además un CSV, el cual contiene la asociación de cada video con diferentes datos relevantes como su etiqueta en ruso, la duración, tamaño, entre otras.\vspace{0.5cm}\par

\section{Preprocesamiento de los datos}

\subsection{Reestructuración de los Datos}

Debido a que los datos se organizan de maneras diferentes, se utilizan distintas aproximaciones para poder extraer los videos, asociarles una etiqueta y llegar a un mismo formato, para de esta manera poder almacenar en nuevos archivos H5 los diferentes datasets con una misma estructura. Con esto en mente se define que la estructura que se utilizara es que cada elemento en el archivo H5 será un video (conjunto de frames) y asociado a cada elemento se tendrá un atributo que se llama \enquote{gloss} el cual contendrá la etiqueta en inglés de cada video.\vspace{0.5cm}\par

Para empezar el proceso que se le realizó a los datos de WLASL consto de diferentes pasos iniciando con la lectura del archivo de texto \enquote{missing.txt}, este paso filtra videos problemáticos desde el inicio, evitando procesar datos que no existen o no son válidos, lo que ahorra tiempo y previene errores. Luego se lee el archivo \enquote{WLASL\_v0.3.json} el cual contiene metadatos, como los Id´s de los videos y sus etiquetas, se carga este archivo para asociar cada video con su significado. Es por esto que el siguiente paso consiste en crear un mapeo de video a etiqueta, procesarla y contarla para verificar que estén completas, cuando se procesa se asegura que no existan caracteres que no se puedan imprimir o que la etiqueta este vacía. Posteriormente, se crea un archivo HDF5 que servirá de contenedor estructurado para almacenar videos y sus etiquetas de una forma determinada, facilitando su uso en análisis posteriores. Luego se Verifica la existencia del archivo, evitando errores al intentar leer un archivo que no está presente, también se filtran videos sin etiqueta, asegurando que solo se procesen videos válidos. Después de asegurar la integridad de los archivos, se leen y guardan en el archivo HDF5 para por último registrar y guardar errores. El anterior procedimiento se resume y define en el siguiente diagrama de flujo.

\begin{figure}[H]
    \caption{Diagrama de flujo con el procesamiento de los datos del dataset de WLSL.03}
    \centering
    \includegraphics[width=0.8\textwidth]{Images/Imagenes Cap 1/DiagramaFlujoWLSL.PNG}
    \label{fig:DiagFWLSL}
\end{figure}

En cuanto al dataset de ISL, se comienza haciendo un recorrido de manera recursiva a la carpeta donde se encuentran los videos y cuando se encuentra un video se extrae la ruta completa del mismo combinado la carpeta y nombre del archivo, para lograr obtener el nombre de la carpeta donde está almacenado y el nombre de una manera fácil. Después se crea un identificador único por cada archivo que consta de la combinación anterior entre la carpeta y nombre, por ejemplo \enquote{Adjetives\_1of81. loud.MOV}. Luego se realiza una limpieza a las etiquetas, se realiza de esta manera porque en su versión original los nombre de los videos vienen con prefijos como \enquote{1. loud} y convertirlos en \enquote{loud}. Posteriormente, se crea una tupla que contiene la ruta del video, el nombre completo y su etiqueta limpia. Ya teniendo los videos y su respectiva etiqueta procesada, se prosigue a realizar  el guardado en el formato indicado. Se realiza el mismo proceso de guardado que en el caso de WLSL, por este motivo se va a obviar el proceso, teniendo el siguiente diagrama de flujo.

\begin{figure}[H]
    \caption{Diagrama de flujo con el procesamiento de los datos del dataset de ISL}
    \centering
    \includegraphics[width=0.8\textwidth]{Images/Imagenes Cap 1/DiagramaFlujoISL.PNG}
    \label{fig:DiagFISL}
\end{figure}

Por último, para el dataset SLOVO se tuvo que emplear pasos extra, pues como se menciona en la descripción del conjunto de datos, las etiquetas aparte de estar organizadas de otra manera, están en ruso, sin embargo, la traducción se realiza al final. Por esto es que se comienza con la lectura del CSV pasándolo a un dataframe para realizarle operaciones más fácilmente. Se crea un archivo H5 para poder almacenar los nuevos datos organizados, luego se lee el df extrayendo el id del video y su respectiva etiqueta para realizar una búsqueda por id en las subcarpetas donde se almacenan los videos. Con el nombre del video (id) y su etiqueta, se realiza el guardado en el archivo H5 de la misma manera que en los casos anteriores, por este motivo se obvia este proceso. Por último, se realiza la traducción de este nuevo archivo H5 comenzando por la extracción de las etiquetas y dejándolas en un archivo TXT, luego se utiliza la librería \enquote{googletrans import Translator} para traducir todas las etiquetas del TXT a inglés, seguido a esto se revisa manualmente que cada etiqueta esté traducida correctamente al inglés y que la palabra exista. Esta traducción se guarda en un archivo JSON en modo diccionario, donde está la clave original en ruso asignada con su traducción, para luego abrir el archivo H5 y empezar a reemplazar las etiquetas en ruso por las traducidas, teniendo como resultado el siguiente diagrama.  

\begin{figure}[H]
    \caption{Diagrama de flujo con el procesamiento de los datos del dataset de SLOVO}
    \centering
    \includegraphics[width=0.8\textwidth]{Images/Imagenes Cap 1/DiagramaFlujoSLOVO.PNG}
    \label{fig:DiagFSLOVO}
\end{figure}

\subsection{Selección de las etiquetas}

Para las diferentes pruebas realizadas se utiliza una herramienta que permite comparar todas las etiquetas procesadas que hay en común en los tres sets de datos, esto para poder asegurar que hay una equivalencia interlingüística entre etiquetas. Con esta herramienta, luego de realizar un análisis a todas las etiquetas existentes de los 3 datasets se encontró lo siguiente, en total se tienen 2000 etiquetas en WLSL, 263 etiquetas en ISL y 927 etiquetas en SLOVO. Las diferentes cantidades etiquetas en común son estas, 204 etiquetas comunes entre WLSL e ISL, 456 etiquetas comunes entre WLSL y SLOVO, 99 etiquetas comunes entre ISL y SLOVO. Teniendo 93 etiquetas comunes entre los tres diferentes datasets, las cuales se pueden representar con la siguiente lista de palabras. 

\begin{longtable}{ccc}

\multicolumn{3}{c}{\textbf{Etiquetas en común}}\\[0.5ex]\hline
\endfirsthead

\multicolumn{3}{c}{\textbf{Etiquetas en común}}\\[0.5ex]\hline
\endhead

animal & bad & beautiful \\
bird & black & blue \\
book & boy & brother \\
brown & cat & cheap \\
child & clean & cold \\
cow & daughter & dog \\
dry & evening & expensive \\
family & famous & father \\
fish & friend & girl \\
good & green & happy \\
hard & heavy & high \\
hot & hour & house \\
i & key & light \\
long & man & mean \\
minute & monday & money \\
month & morning & mother \\
mouse & new & newspaper \\
nice & night & old \\
orange & paper & pencil \\
pink & poor & price \\
red & religion & restaurant \\
saturday & school & second \\
short & sister & slow \\
soft & son & spring \\
strong & summer & sunday \\
thursday & time & today \\
tomorrow & train & tuesday \\
waiter & warm & week \\
white & wife & winter \\
woman & year & yellow \\
yesterday & you & young \\

\end{longtable}

\subsection{Redimensionamiento de tamaño y duración de las secuencias}

También se le realiza un preprocesamiento especial a los datos seleccionados el cual consiste en un recorte específico para uniformar la duración de las secuencias, un ajuste de las dimensiones de los videos para igualar los tamaños, un cambio a blanco y negro para que las secuencias se puedan procesar de manera más eficiente, y una separación del intérprete del fondo, dejando uno solo para las secuencias.\vspace{0.5cm}\par

Se decide optar por la utilización de estos cambios para mejorar la tasa de aprendizaje y su rendimiento. Para el mejor resultado del proyecto, se necesita el empleo de varias etiquetas y de múltiples videos, para que el tiempo de ejecución no sea muy grande, se consideran todas las opciones que hay disponibles para recortar la mayor cantidad de información, asegurando un resultado satisfactorio. Es por esto que se decide igualar todas las secuencias con una estrategia específica, el primer paso de esta es calcular la mediana de la cantidad de frames de todas las secuencias dadas. Luego, teniendo como referencia esta métrica, se realiza una comprobación para saber si la secuencia que se está procesando es menor, mayor o igual a la mediana.\vspace{0.5cm}\par 

Si es menor a la mediana, se crea una nueva secuencia que tiene el número de frames especifico, esto se logra usando una interpolación lineal para generar una nueva secuencia con una duración determinada, por ejemplo si la mediana fuera 8 se distribuyen los frames (originales e interpolados) uniformemente en la nueva secuencia. Los nuevos frames se crean como combinaciones ponderadas de los frames originales en posiciones intermedias, determinadas por los índices generados por la funcion \enquote{np.linspace}. Siguiendo el ejemplo, se añaden 3 frames a una secuencia de 5, donde los nuevos frames estarán distribuidos aproximadamente entre los frames originales, con pesos calculados según su posición relativa.\vspace{0.5cm}\par 

Si es mayor a la mediana se reduce la secuencia de video seleccionando un subconjunto de frames distribuidos uniformemente, eliminando de esta manera los frames sobrantes, cabe aclarar que en este caso no se realiza interpolación ni transformación, solo selecciona frames ya existentes. Por ejemplo, si la duración actual del clip son 10 frames y la mediana es 7, se utiliza la función \enquote{np.linspace(0, 9, num=7, dtype=int)} dando como resultado algo como [0, 1, 3, 4, 6, 7, 9]. Esta función selecciona 7 frames de los 10 originales, distribuidos lo más uniformemente posible para evitar errores como que se eliminen de manera seguida y perder fragmentos importantes de video.\vspace{0.5cm}\par 

Por último, si la secuencia resulta tener la misma duración de la mediana, se deja igual realizando una copia de los frames de la secuencia que se esté procesando en el momento. Posteriormente a este paso, se redimensionan los frames a 120 x 160 y se pasa a blanco y negro, ya que no se pierde la información de lo que está realizando el intérprete, y es necesario para poder procesar la mayor cantidad de secuencias posible con los recursos delimitados.\vspace{0.5cm}\par 

\subsection{Recorte del fondo de las secuencias}

Para asegurarse que el modelo se encargue de aprender las características temporales, se decide quitar el fondo y dejar solamente a los intérpretes. Se refuerza esta decisión teniendo en cuenta que es bastante probable que el modelo empiece a agrupar secuencias de video en el espacio latente según el fondo en el que se es grabado. Como, por ejemplo, que agrupe todas las secuencias de ISL que tenga un tablero en el fondo, o que agrupe todas las secuencias de WLSL que tengan un fondo blanco. Para lograr esto se diseña un algoritmo que tiene como base la utilización del modelo \enquote{yolov8m-seg.pt}. No se emplea alguna librería preentrenada de opencv o de Google porque se busca un enfoque que sea muy flexible y personalizable, se busca esta aproximación en mayor medida, porque hay algunos casos específicos que no se detectan de manera correcta con estas librerías. Como lo puede ser cuando se tiene la interpolación de frames, se tienen dos o una figura translúcida, pues es un frame de transición donde el intérprete esta en medio de dos señas.\vspace{0.5cm}\par

En el caso del dataset de ISL y WLSL coincidieron en que la mejor variación de hiperparámetros para hallar el mejor recorte es la utilización de un método híbrido. Este método es la combinación de dilatación de la máscara, con el fin de adaptarse mejor a la forma del intérprete, con un margen adicional alrededor del contorno, ofreciendo un enfoque más robusto. Sin embargo, para el dataset SLOVO la mejor opción fue el método de dilatación con un kernel 15x15.\vspace{0.5cm}\par

La elección de estos métodos, y que se emplee un margen al rededor de todos los interpretes, se debe principalmente a que es muy complicado generar un algoritmo que se adapte a todas las diferentes formas que puede producir un intérprete con pocos videos. Además, hay muchos intérpretes diferentes que como en el caso del dataset SLOVO todos estaban grabados de maneras diferentes, cambiando tanto el fondo como la iluminación bruscamente. Por este motivo es que no se llega a recortar enteramente la silueta de los videos, puede que funcione en la mayoría de los videos, pero si hay varias secuencias de video donde se recortan los dedos de la seña que se está haciendo, el método se considera insuficiente.\vspace{0.5cm}\par

Luego de realizar los recortes al revisar todos los frames para corroborar que se habían recortado correctamente, se pudo evidenciar que en contados momentos muy rara vez se producía un error donde dejaba un frame en negro, tambien algunas secuencias tenian frames en negro desde el inicio. En vista de que es un problema muy poco frecuente donde se recorta todo el contenido, se decidió que cuando se encuentre un frame de este tipo, fuera reemplazado por la combinación del que tiene antes y después.

\subsection{Recuento de los datos disponibles}

Con los datos preprocesados se hacen subconjuntos de los tres datasets que se trabajan. Estos subconjuntos están delimitados por sus etiquetas y se asegura que estas sean las mismas para cada subconjunto de cada uno de los lenguajes de señas, se crean subconjuntos para las siguientes cantidades de etiquetas; 5, 10, 20, 35, 50 y 72 etiquetas. Como se señaló, se crea un subconjunto por cada uno de los datasets, resultando en 18 subconjuntos de datos, que son equivalentes entre sí en cuanto a su denominación, en otras palabras, el subconjunto de 5 etiquetas de ISL tiene las mismas etiquetas que el dataset de 5 etiquetas en SLOVO.\vspace{0.5cm}\par

Esto se realiza para poder realizar comparaciones que estén en las mismas condiciones y para que los conjuntos de datos estén alineados con el objetivo del proyecto, el cual es encontrar una equivalencia entre lenguajes. Además de poder realizar diferentes simulaciones sin tener el problema que no permite utilizar todas las etiquetas al mismo tiempo debido a falta de recursos y poder computacional en la plataforma de AWS. Cabe aclarar que no se tiene la misma cantidad de videos por los tres idiomas en cada etiqueta, lo cual significa que se tiene un gran desbalance, donde en el peor caso posible un idioma podria superar a otro por mas de 50 videos. Para poder detectar cuáles son las etiquetas que tienen un mayor desbalance se decidió realizar una tabla, la cual contiene la etiqueta, la cantidad de videos que tiene por lenguaje, el mínimo de videos que tienen en algún lenguaje y por último el total de videos por etiqueta.\vspace{0.5cm}\par

\begin{longtable}{r l c c c c c}
\multicolumn{7}{c}{\textbf{RANKING DE ETIQUETAS POR BALANCE Y COMPLETITUD}} \\
\multicolumn{7}{c}{(Ordenado de la etiqueta más equilibrada a la menos equilibrada)} \\[0.5ex]\hline
\# & ETIQUETA & ISL & SLOVO & WLSL\_V03 & MÍNIMO (Balance) & TOTAL \\
\hline
\endfirsthead
\multicolumn{7}{c}{\textbf{RANKING DE GLOSAS POR BALANCE Y COMPLETITUD}} \\
\multicolumn{7}{c}{(Ordenado de la glosa más equilibrada a la menos equilibrada)} \\[0.5ex]\hline
\# & GLOSA & ISL & SLOVO & WLSL\_V03 & MÍNIMO (Balance) & TOTAL \\
\hline
\endhead
1 & short & 21 & 40 & 13 & 13 & 74 \\
2 & cold & 20 & 20 & 12 & 12 & 52 \\
3 & man & 19 & 20 & 12 & 12 & 51 \\
4 & brother & 20 & 20 & 11 & 11 & 51 \\
5 & mother & 19 & 20 & 11 & 11 & 50 \\
6 & woman & 19 & 20 & 11 & 11 & 50 \\
7 & dog & 18 & 20 & 11 & 11 & 49 \\
8 & family & 16 & 20 & 11 & 11 & 47 \\
9 & thursday & 11 & 20 & 11 & 11 & 42 \\
10 & good & 21 & 60 & 10 & 10 & 91 \\
11 & black & 19 & 40 & 10 & 10 & 69 \\
12 & bad & 21 & 20 & 10 & 10 & 51 \\
13 & hot & 21 & 20 & 10 & 10 & 51 \\
14 & daughter & 19 & 20 & 10 & 10 & 49 \\
15 & orange & 19 & 20 & 10 & 10 & 49 \\
16 & son & 19 & 20 & 10 & 10 & 49 \\
17 & white & 19 & 20 & 10 & 10 & 49 \\
18 & bird & 18 & 20 & 10 & 10 & 48 \\
19 & fish & 18 & 20 & 10 & 10 & 48 \\
20 & year & 11 & 20 & 10 & 10 & 41 \\
21 & yesterday & 11 & 20 & 10 & 10 & 41 \\
22 & dry & 21 & 20 & 9 & 9 & 50 \\
23 & new & 21 & 20 & 9 & 9 & 50 \\
24 & school & 20 & 20 & 9 & 9 & 49 \\
25 & child & 19 & 20 & 9 & 9 & 48 \\
26 & cow & 19 & 20 & 9 & 9 & 48 \\
27 & girl & 19 & 20 & 9 & 9 & 48 \\
28 & pink & 19 & 20 & 9 & 9 & 48 \\
29 & train & 19 & 20 & 9 & 9 & 48 \\
30 & animal & 18 & 20 & 9 & 9 & 47 \\
31 & cat & 18 & 20 & 9 & 9 & 47 \\
32 & minute & 11 & 20 & 9 & 9 & 40 \\
33 & today & 11 & 20 & 9 & 9 & 40 \\
34 & week & 11 & 20 & 9 & 9 & 40 \\
35 & you & 21 & 80 & 8 & 8 & 109 \\
36 & yellow & 19 & 40 & 8 & 8 & 67 \\
37 & happy & 21 & 20 & 8 & 8 & 49 \\
38 & slow & 21 & 20 & 8 & 8 & 49 \\
39 & boy & 20 & 20 & 8 & 8 & 48 \\
40 & brown & 20 & 20 & 8 & 8 & 48 \\
41 & blue & 19 & 20 & 8 & 8 & 47 \\
42 & wife & 19 & 20 & 8 & 8 & 47 \\
43 & sunday & 11 & 20 & 8 & 8 & 39 \\
44 & time & 11 & 20 & 8 & 8 & 39 \\
45 & expensive & 8 & 20 & 8 & 8 & 36 \\
46 & light & 8 & 20 & 8 & 8 & 36 \\
47 & mean & 8 & 20 & 8 & 8 & 36 \\
48 & soft & 8 & 20 & 8 & 8 & 36 \\
49 & strong & 8 & 20 & 8 & 8 & 36 \\
50 & beautiful & 8 & 40 & 7 & 7 & 55 \\
51 & old & 21 & 20 & 7 & 7 & 48 \\
52 & house & 20 & 20 & 7 & 7 & 47 \\
53 & restaurant & 20 & 20 & 7 & 7 & 47 \\
54 & father & 19 & 20 & 7 & 7 & 46 \\
55 & friend & 19 & 20 & 7 & 7 & 46 \\
56 & green & 19 & 20 & 7 & 7 & 46 \\
57 & red & 19 & 20 & 7 & 7 & 46 \\
58 & sister & 19 & 20 & 7 & 7 & 46 \\
59 & hour & 11 & 20 & 7 & 7 & 38 \\
60 & month & 11 & 20 & 7 & 7 & 38 \\
61 & saturday & 11 & 20 & 7 & 7 & 38 \\
62 & summer & 11 & 20 & 7 & 7 & 38 \\
63 & tomorrow & 11 & 20 & 7 & 7 & 38 \\
64 & tuesday & 11 & 20 & 7 & 7 & 38 \\
65 & cheap & 8 & 20 & 7 & 7 & 35 \\
66 & clean & 8 & 20 & 7 & 7 & 35 \\
67 & paper & 7 & 20 & 8 & 7 & 35 \\
68 & religion & 7 & 20 & 7 & 7 & 34 \\
69 & spring & 11 & 40 & 6 & 6 & 57 \\
70 & hard & 8 & 40 & 6 & 6 & 54 \\
71 & monday & 11 & 20 & 6 & 6 & 37 \\
72 & morning & 11 & 20 & 6 & 6 & 37 \\\
73 & winter & 11 & 20 & 6 & 6 & 37 \\
74 & famous & 8 & 20 & 6 & 6 & 34 \\
75 & high & 8 & 20 & 6 & 6 & 34 \\
76 & poor & 8 & 20 & 6 & 6 & 34 \\
77 & book & 7 & 20 & 6 & 6 & 33 \\
78 & key & 7 & 20 & 6 & 6 & 33 \\
79 & money & 7 & 20 & 6 & 6 & 33 \\
80 & price & 7 & 20 & 6 & 6 & 33 \\
81 & long & 21 & 20 & 5 & 5 & 46 \\
82 & warm & 21 & 20 & 5 & 5 & 46 \\
83 & young & 21 & 20 & 5 & 5 & 46 \\
84 & mouse & 18 & 20 & 5 & 5 & 43 \\
85 & night & 11 & 20 & 5 & 5 & 36 \\
86 & heavy & 8 & 20 & 5 & 5 & 33 \\
87 & pencil & 7 & 20 & 5 & 5 & 32 \\
88 & i & 21 & 60 & 4 & 4 & 85 \\
89 & evening & 11 & 20 & 4 & 4 & 35 \\
90 & newspaper & 7 & 20 & 4 & 4 & 31 \\
91 & nice & 4 & 20 & 6 & 4 & 30 \\
92 & waiter & 11 & 20 & 3 & 3 & 34 \\
93 & second & 3 & 20 & 6 & 3 & 29 \\
\hline
\end{longtable}

Teniendo en cuenta la tabla anterior, los subconjuntos que se mencionaron anteriormente van a estar definidos por el orden de la tabla, en otras palabras, los subconjuntos con 5 etiquetas tendrán las que aparecen en el top 5 de la tabla y así sucesivamente. Se realiza esto para asegurarse que se usaran las mejores etiquetas posibles en los diferentes experimentos.\vspace{0.5cm}\par

Siguiendo ese orden de ideas, el trabajo de investigación se trabajará con los siguientes subconjuntos de los tres Datasets principales en la medida que lo permitan los recursos computacionales:

\begin{itemize}

\item \textbf{\small Datasets que tienen las primeras 5 etiquetas: } En primera instancia se tiene \enquote{ISL\_5gloss.h5} con 99 videos, luego \enquote{SLOVO\_5gloss.h5} con 120 videos y por último a \enquote{WLSL\_5gloss.h5} conteniendo un total de 59 videos.

\item \textbf{\small Datasets que tienen las primeras 10 etiquetas: } También se tiene \enquote{ISL\_5gloss.h5} con 184 videos, luego \enquote{SLOVO\_5gloss.h5} con 260 videos y por último a \enquote{WLSL\_5gloss.h5} conteniendo un total de 113 videos.

\item \textbf{\small Datasets que tienen las primeras 20 etiquetas: } En esta división se tiene \enquote{ISL\_5gloss.h5} con 368 videos, luego \enquote{SLOVO\_5gloss.h5} con 480 videos y por último a \enquote{WLSL\_5gloss.h5} conteniendo un total de 213 videos.

\item \textbf{\small Datasets que tienen las primeras 35 etiquetas: } Por otro lado, se tiene \enquote{ISL\_5gloss.h5} con 626 videos, luego \enquote{SLOVO\_5gloss.h5} con 840 videos y por último a \enquote{WLSL\_5gloss.h5} conteniendo un total de 348 videos.

\item \textbf{\small Datasets que tienen las primeras 50 etiquetas: } Además, en esta subdivisión se tiene \enquote{ISL\_5gloss.h5} con 835 videos, luego \enquote{SLOVO\_5gloss.h5} con 1180 videos y por último a \enquote{WLSL\_5gloss.h5} conteniendo un total de 467 videos.

\item \textbf{\small Datasets que tienen las primeras 72 etiquetas: } En este conjunto se tiene \enquote{ISL\_5gloss.h5} con 1128 videos, luego \enquote{SLOVO\_5gloss.h5} con 1660 videos y por último a \enquote{WLSL\_5gloss.h5} conteniendo un total de 618 videos.

\item \textbf{\small Datasets que tienen todas las etiquetas: } Y finalmente, se tiene \enquote{ISL\_5gloss.h5} con 1355 videos, luego \enquote{SLOVO\_5gloss.h5} con 2120 videos y por último a \enquote{WLSL\_5gloss.h5} conteniendo un total de 728 videos.

\end{itemize}

Para resumir todo el proceso y quede mas claro todo el proceso de preprocesado de los datos, se propone este pequeño diagrama.

\begin{figure}[H]
    \caption{Diagrama de flujo con el resumen de todo el procesamiento de los datos de los tres Datasets}
    \centering
    \includegraphics[width=0.8\textwidth]{Images/Imagenes Cap 1/DiagramaFlujoFinal.PNG}
    \label{fig:DiagFFinal}
\end{figure}

Además, se presentan tres ejemplos de como se transforman las secuencias de video durante el proceso, en la primera fila se tienen ocho frames distribuidos de la secuencia original, luego en la segunda fila se tiene la secuencia luego del recorte inteligente, por último la secuencia final pertenecen al resultado que se le pasa al modelo sin fondo ni frames en negro. En el caso de la figura que tiene el ejemplo, WLSL presenta algunos errores que contiene el dataset en sus videos, donde al final se presentan algunos frames en negro. Este error se soluciona con el recorte inteligente, al ser pocos frames en negro, no se tienen en cuenta. Sin embargo, en algunas secuencias los frames en negro alcanzan procesos finales donde ya se eliminó el fondo. Por este motivo está la última etapa de verificación y depuración que se mencionó con anterioridad, es muy importante, aunque no aplique para todos los casos.

\begin{figure}[H]
    \caption{Evolución de las secuencias del dataset ISL durante el procesamiento}
    \centering
    \includegraphics[width=0.8\textwidth]{Images/Imagenes Cap 1/EvolucionSecISL.PNG}
    \label{fig:EvolSecISL}
\end{figure}

\begin{figure}[H]
    \caption{Evolución de las secuencias del dataset SLOVO durante el procesamiento}
    \centering
    \includegraphics[width=0.8\textwidth]{Images/Imagenes Cap 1/EvolucionSecSLOVO.PNG}
    \label{fig:EvolSecSLOVO}
\end{figure}

\begin{figure}[H]
    \caption{Evolución de las secuencias del dataset WLSL durante el procesamiento}
    \centering
    \includegraphics[width=0.8\textwidth]{Images/Imagenes Cap 1/EvolucionSecWLSL.PNG}
    \label{fig:EvolSecWLSL}
\end{figure}
    \let\cleardoublepage\clearpage
    
    % Capitulo 2
    \chapter{Capítulo 2: Aplicación los datos obtenidos a la técnica}
    \section{Estructuración de la técnica}

Se dividirá esta sección en las diferentes partes de la tecnica desarrollada. El orden de las subsecciones se da a partir del flujo de datos a traves del programa para poder cumplir el objetivo principal de la misma. El cual es, entrenar un modelo de aprendizaje profundo para aprender representaciones vectoriales, conocidas como \enquote{embeddings}, de videos de lenguaje de señas. La meta no solo es clasificar las señas, sino crear un \enquote{espacio latente} donde los videos con la misma seña estén agrupados y los videos con señas diferentes estén separadas.

\subsection{Preparación de datos}

Para preparar los datos se comienza con una división de los mismos. Por ejemplo, si se tiene un conjunto de 60 videos, este se divide en un conjunto de entrenamiento, con 48 videos, y otro de validación, con 12 videos. Es importante hacer una división estratificada para que la proporción de cada glosario sea la misma en ambos conjuntos, esta observación se realiza porque la cantidad de videos por etiqueta varía fuertemente dependiendo de la palabra que se esté utilizando. De no realizar la estratificación correctamente puede que todos los videos de una etiqueta resulten en train, pero no en test, haciendo que los conjuntos estén desbalanceados y no sean representativos.\vspace{0.5cm}\par

Luego se realiza una normalización donde los valores de los píxeles, que inicialmente van de 0 a 255, se escalan a un rango de 0 a 1. Es importante destacar que esta normalización se calcula usando solo los valores mínimos y máximos del conjunto de entrenamiento para evitar la filtración de información del conjunto de validación al modelo, lo que podría llegar a sesgar el rendimiento del modelo al exponerlo a datos que no debería saber durante la fase de entrenamiento. Además, antes de la normalización, los datos se convierten al tipo float32 para garantizar precisión en los cálculos. Posteriormente, se crean mapas de etiquetas a índices para ambos conjuntos, lo que facilita la organización de los datos por categorías para tareas posteriores como los es el aprendizaje con la técnica triplet. Por último, se liberan los datos originales de la memoria mediante la eliminación de variables y la recolección de basura, optimizando el uso de recursos computacionales.

\subsection{Construcción de las variantes}

El siguiente paso es muy importante para todo el entrenamiento. Esto se dice, pues se hace toda la transformación a cada video que se encuentre en los conjuntos de entrenamiento y validación. Esta transformación es la generación de quintetos basada en la técnica de perdida de triplet, es decir, que para cada video de entrada, en otras palabras el ancla, se generan cuatro variantes adicionales, generando de esta forma un grupo de 5. En este grupo se tiene el Ancla, la cual es el video sin cambios. Luego se tiene el Positivo Desplazado, El cual es el mismo video, pero con los fotogramas desplazados temporalmente en una posición, esta se considera una variación buena por su similitud temporal. Posteriormente, se tiene el Negativo de una etiqueta, este es un video aleatorio de una seña diferente. Es una variación que se considera mala. También se tiene el Negativo Invertido, este es el video original pero reproducido hacia atrás. Este también se considera una variación mala por su estructura temporal. Por último, se tiene el Negativo Permutado, el cual consiste en que los fotogramas del video original están orden aleatorio. Esta es la última variación temporalmente mala.\vspace{0.5cm}\par

Este nuevo formato de datos de entrada ahora tiene una dimensión extra de tamaño 5 para contener estas variantes. Por ejemplo, la forma de los datos de entrenamiento pasa a ser (48, 5, 7, 120, 160, 1). Donde 48 es el número de muestras, 5 el número de variantes, 7 el número de frames de las secuencias, 120 y 160 es  el tamaño de cada frame y por último el 1 son los canales.\vspace{0.5cm}\par

El propósito de la inclusión de estas variantes es que el modelo pueda aprender de la mejor manera el \enquote{embedding} que representa cada seña. El modelo aprende esta representación al ser forzado a resolver una tarea pretexto, que en este caso es diferenciar entre variaciones \enquote{buenas} y \enquote{malas} de un mismo video. Se afirma que es capas de aprender a capturar las características temporales y espaciales porque al tener el Ancla y el Positivo Desplazado con representaciones similares, el modelo aprende a ser invariante a pequeños cambios temporales. Por otro lado, al también forzar a que las representaciones del Ancla y los tres Negativos sean diferentes, el modelo aprende a ser más sensible a lo que hace única una seña, es decir, el orden temporal correcto y la dirección del movimiento.

\subsection{Construcción del modelo}

En el siguiente paso se construye lo que sería la base del modelo, un autoencoder Conv3D. Este modelo primero redimensiona los fotogramas de entrada pasando de 120x160 a 80x60 para reducir la carga computacional. Luego, codifica la secuencia de video en una representación latente más pequeña y finalmente la decodifica para reconstruir el video original.\vspace{0.5cm}\par

La codificación se realiza con cuello de botella, este se logra mediante diferentes capas convolucionales y de agrupación. Se utilizan capas Conv3D que a diferencia de las capas Conv2D que analizan imágenes que están estáticas, estas utilizan filtros tridimensionales. Esto quiere decir que el filtro no solo se desliza sobre la altura y el ancho del fotograma, sino que también a través del eje del tiempo. Esto es algo muy importante, pues es lo que hace que el modelo aprenda características espaciotemporales, como lo pueden ser patrones de movimiento, gestos y la dinámica de la seña como tal, en lugar de solo formas estáticas\autocite[p.~4]{Dey2024}\autocite[p.~1355,1369]{Innocente2025}. El número de filtros, primero 16 y luego 32, aumentan la capacidad del modelo para aprender características más complejas en cada nivel. También se usan capas MaxPooling3D después de cada convolución. Por ejemplo, una capa con pool\_size=(1, 2, 2) se encarga de reducir la dimensionalidad espacial, dividiendo la altura y ancho por 2, pero mantiene intacta la dimensión temporal. Esto hace que la representación sea más abstracta, robusta a movimientos cortos y previene el overfitting\autocite[p.~2924]{Dey2024}, cabe aclarar que también esto hacer que la longitud original de la secuencia de la seña sea igual. Por otro lado, también se usan capas BatchNormalization, las cuales se utilizan después de las convoluciones para poder estabilizar y acelerar el proceso de entrenamiento, normalizando de esta manera las activaciones de la capa anterior. Por último, se emplea el Cuello de Botella, bottleneck\_sequence, esta se considera la salida final del codificador, es una secuencia de tensores que representa la versión más comprimida y abstracta del video original. Esta también se considera la representación latente.\vspace{0.5cm}\par

En cuanto a la decodificación, se realiza el proceso inverso, esto quiere decir que se toma la representación latente compacta y se expande para reconstruir el video original. Teniendo esto en mente, se usan capas Conv3DTranspose, las cuales son el complemento de las capas Conv3D y MaxPooling3D. Estas mediante el uso de strides=(1, 2, 2) pueden realizar una \enquote{deconvolución} o también conocida como \enquote{upsampling}, duplicando las dimensiones espaciales mientras aprenden a rellenar los detalles perdidos durante la codificación. Además, se tiene una capa de Salida, esta se considera la capa final, es una Conv3D con un número de filtros igual a los canales del video original. Esta hace uso de una función de activación sigmoid, se escoge esta función porque los píxeles de la imagen de entrada se normalizan típicamente al rango [0, 1], permitiendo que la función sigmoide se asegure que los píxeles del video reconstruido también se encuentren en este mismo rango, haciendo de esta manera que la función de pérdida sea más efectiva.\vspace{0.5cm}\par

El modelo se encarga de devolver dos valores muy importantes, siendo el primero la reconstrucción. Aunque el objetivo del proyecto no es tener el video final reconstruido, juega un papel relevante porque es la manera en la que se calculara la pérdida de reconstrucción para obligar al autoencoder a aprender y mantener la información esencial del video como lo pueden ser la cantidad de dedos levantados entre otros indicadores. Mientras que la otra salida es la representación latente del cuello de botella. Esta salida es la que se utilizará en la etapa de aprendizaje contrastivo. Esta arquitectura con dos salidas es extremadamente eficiente, esto se puede afirmar porque con una sola pasada hacia adelante a través del codificador, se puede obtener las representaciones latentes necesarias para la pérdida de triplet y luego, al continuar por el decodificador, también se obtiene la salida para tener la pérdida de reconstrucción. Como se puede ver, esto permite entrenar el modelo para dos tareas simultáneamente, siendo estas la discriminación y reconstrucción, potenciando el aprendizaje de representaciones que son a la vez discriminativas y buenas en contenido.\vspace{0.5cm}\par

Sin embargo, el modelo también tiene una particularidad y es que el autoencoder se envuelve en una clase personalizada llamada \enquote{TemporalTripletAutoencoder}, esta clase le añade dos capas adicionales. La primera es una capa GRU Bidireccional para poder analizar la secuencia temporal proveniente del cuello de botella del autoencoder. Mientras que la otra es una capa de Pooling final para condensar la salida de la GRU en un único vector de características de dimensión 256 para cada video. Esta es una implementación común y potente para tareas de video, dondeen este caso en lugar de un autoencoder, generalmente se utiliza una red Conv3D o una CNN espaciotemporal, que actúa como la parte codificadora para extraer características visuales, donde despues estas características se pasan a una red recurrente para analizar la secuencia\autocite[p.~1367]{Innocente2025}\autocite[p.~1]{Inamdar2023}\autocite[p.~2923]{Dey2024}.\vspace{0.5cm}\par

Se toma esta aproximación por las ventajas que significa tener cierta encapsulación y lógica Compleja, es decir, el wrapper encapsula no solo la arquitectura sino que también toda la lógica de entrenamiento. Esto permite gestionar de forma ordenada los múltiples componentes de la función de pérdida, en este caso de reconstrucción, varias pérdidas de tripleta, etc., y sus respectivos pesos que se definen en el constructor. Esto también significa que hay más control del flujo de datos. Además, proporciona un control explícito sobre el paso hacia adelante, en otras palabras, esto significa que define exactamente cómo los datos fluyen desde la entrada, a través del autoencoder, y luego a través de las nuevas capas temporales para producir el vector final. No obstante, también proporciona flexibilidad en el entrenamiento, pudiendo sobreescribir el método \enquote{train\_step} de Keras para implementar un ciclo de entrenamiento completamente personalizado, donde se calculan y combinan las diferentes pérdidas, algo que es más complicado en modelo secuencial simple.\vspace{0.5cm}\par

Esta nueva a aproximación agrega más capas que procesan la salida del cuello de botella del autoencoder. Para poder realizarlo, antes de que la secuencia pueda ser procesada por la GRU, se prepara porque el cuello de botella del autoencoder produce una secuencia de mapas de características 3D, con la siguiente forma \enquote{(Tiempo, Alto, Ancho, Canales)}. Sin embargo, la capa GRU espera una secuencia de vectores 1D, con forma\enquote{(Tiempo, Características)}\autocite[p.~3]{Inamdar2023}\autocite[p.~2923]{Dey2024}.\vspace{0.5cm}\par

Con este paso previo realizado, se puede comenzar con la agregación Temporal usando una GRU Bidireccional. Se puede considerar que esta es la capa más importante para entender la dinámica temporal de una seña. Esto porque una Unidad Recurrente Cerrada (GRU) es un tipo de Red Neuronal Recurrente (RNN) que está diseñada para procesar secuencias. Esto quiere decir que posee filtros internos que le permiten decidir qué información de los pasos anteriores es relevante mantener y cuál se puede olvidar, permitiéndole de esta manera capturar detalles a lo largo del tiempo. Por ejemplo, se puede decir que es como leer un dicho, para poder entenderlo la GRU tiene la capacidad de recordar el contexto inicial para entender las palabras finales.\vspace{0.5cm}\par

Ahora bien, aplicado al proyecto, una GRU estándar procesa la secuencia en un solo sentido, pero con un envoltorio bidireccional duplica la capa GRU. Mientras una procesa la secuencia hacia adelante, la otra la procesa hacia atrás para luego juntar ambas salidas. Esto es muy importante para el reconocimiento de señas, ya que el significado de un gesto depende no solo de lo que el intérprete ya realizo, sino que también de lo que hará después. Esto le da al modelo un contexto completo de toda la secuencia en cada punto de tiempo\autocite[p.~1371]{Innocente2025}.\vspace{0.5cm}\par

Ahora bien, teniendo en cuenta lo anterior, la GRU Bidireccional tiene como resultado una secuencia de vectores enriquecidos con contexto temporal. Sin embargo, para la tarea de comparación que serán las distintas pérdidas, se necesita de un único vector de características que represente toda una secuencia de video. Para esto, la capa GlobalAveragePooling1D toma la secuencia de salida de la GRU y calcula el promedio de todos los vectores a lo largo de la dimensión temporal. Dando como resultado una condensación de una secuencia de longitud variable en un único vector de tamaño fijo.\vspace{0.5cm}\par

A continuación, se muestra un diagrama que tiene toda la arquitectura del modelo antes descrita y las relaciones que tiene cada parte entre sí. Además, se muestran las interacciones de las diferentes perdidas triplet, donde la flecha verde indican que se atraen, mientras que la roja indica que hay cierto grado de repulsión. Se verá con más detalle adelante como funcionan estas perdidas en el espacio latente de visualización.\vspace{0.5cm}\par

\begin{figure}[H]
    \caption{Diagrama de la construcción del modelo}
    \centering
    \includegraphics[width=0.8\textwidth]{Images/Imagenes Cap 2/ResumenMod.png}
    \label{fig:ResumenMod}
\end{figure}

\subsection{Entrenamiento del Modelo}

Como se mencionó con anterioridad, se tiene un modelo wrapper que facilita la personalización del entrenamiento. Por este motivo se tiene su propio \enquote{train\_step}, que es el núcleo de la lógica de aprendizaje. Con esta modificación, cada paso de entrenamiento, calcula una pérdida combinada que se encarga de incluir las diferentes perdidas que se mencionaron con anterioridad en la construcción de las variantes.\vspace{0.5cm}\par

Durante el entrenamiento, las triplet losses, al usar las distancias cuadradas, crean un espacio latente métrico donde la geometría puede llegar a reflejar similitudes semánticas y temporales. Al diferenciar shifts leves, es decir, los positivos, de cambios drásticos, como lo pueden ser negativos como gloss diferentes o alteraciones temporales, el modelo puede llegar a generalizar mejor los datos ruidosos.\vspace{0.5cm}\par

\begin{figure}[H]
    \caption{Comportamiento esperado para el Triplet Loss entre secuencias}
    \centering
    \includegraphics[width=0.8\textwidth]{Images/Imagenes Cap 2/GrafTriplet.png}
    \label{fig:GrafTripletSupuesto}
\end{figure}

Por ejemplo, puede ignorar las variaciones menores, pero detecta cambios en el significado o la secuencia. Por otro lado, usar distancias cuadradas es computacionalmente barato y junto a márgenes grandes permiten separaciones fuertes sin sobrepenalizar. Esta combinación de triplets múltiples permite multi-task learning, es decir, reconstrucción para fidelidad, inter-gloss para discriminación, y temporales para invariancia al orden.\vspace{0.5cm}\par

En cuanto a su compilación, el modelo se compila con el optimizador Adam con una tasa de aprendizaje inicial baja. Por otro lado, utiliza una serie de configuraciones de Callbacks y entrenamiento específicas, estas son, EarlyStopping que sirve para detener el entrenamiento si el rendimiento en el conjunto de validación deja de mejorar, y la otra es ReduceLROnPlateau que funciona para reducir la tasa de aprendizaje si el entrenamiento se estanca. Para que se pueda realizar el entrenamiento de manera efectiva y completa en la plataforma de AWS, se limita el entrenamiento durante un máximo de 100 épocas, pero se detiene antes si se llega a activar el EarlyStopping, restaurando los mejores pesos de la época.

\subsection{Evaluación del Modelo}

En cuanto a la evaluación Post-Entrenamiento, se tienen en cuenta las métricas finales donde se mide el modelo final con los mejores pesos en el conjunto de validación completo para obtener las métricas de rendimiento finales. Esto lo hace por medio de una prueba de sensibilidad temporal, la cual se ejecuta con una función que verifica si el modelo aprendió correctamente la estructura temporal, midiendo si los vectores de los videos desplazados están, en promedio, más cerca de los originales que los vectores de los videos invertidos y permutados. Si el resultado se cumple, se puede confirmar que el aprendizaje fue exitoso.\vspace{0.5cm}\par

Ademas de estas medidas se utilizan una serie de 

Esperar a las diferentes modificaciones para sacar metricas

\subsection{Visualizacion de los Resultados}

Esperar a las diferentes modificaciones para sacar metricasx2

\section{Aplicación de los datos obtenidos y documentación de resultados}

\subsection{Con el dataset de WLSL}

\subsubsection{Con 5 etiquetas}

\subsubsection{Con 20 etiquetas}

\subsubsection{Con 50 etiquetas}

\subsubsection{Con 93 etiquetas}

\subsection{Con el dataset de ISL}

\subsubsection{Con 5 etiquetas}

\subsubsection{Con 20 etiquetas}

\subsubsection{Con 50 etiquetas}

\subsubsection{Con 93 etiquetas}

\subsection{Con el dataset de SLOVO}

\subsubsection{Con 20 etiquetas}

\subsubsection{Con 50 etiquetas}

\subsubsection{Con 93 etiquetas}

\subsection{Con los tres conjuntos de datos}

\subsubsection{Con 20 etiquetas}

\subsubsection{Con 50 etiquetas}

\subsubsection{Con 93 etiquetas}
    \let\cleardoublepage\clearpage
    
    % Capitulo 3
    \chapter{Capítulo 3: Evaluación los resultados obtenidos y futuras investigaciones}
    %Analsis de resultados El documento "Batch feature standardization network with triplet loss" sustenta la preocupación por los conjuntos de datos desbalanceados. Menciona que una de las dificultades en la detección de anomalías en video es que el número de eventos normales supera con creces el de eventos anómalos, lo que resulta en "unbalanced datasets" (conjuntos de datos desbalanceados)

Esta idea se sustenta de manera conceptual en el documento "Batch feature standardization network with triplet loss". El artículo propone un módulo de estandarización de características donde la media y la desviación estándar se calculan 

únicamente a partir de las "bolsas negativas" (datos normales, análogos a un conjunto de entrenamiento). Luego, estos valores se utilizan para estandarizar tanto las bolsas positivas como las negativas. Este procedimiento tiene el mismo propósito que el descrito en tu párrafo: evitar que la información de un conjunto (en este caso, las "bolsas positivas" que pueden contener datos anómalos) contamine el proceso de estandarización, lo cual se alinea perfectamente con la idea de prevenir la filtración de datos del conjunto de validación.

Creación de categorías para Triplet Loss: La creación de categorías o mapas de etiquetas para facilitar el aprendizaje con la técnica de triplet loss está respaldada por dos documentos:


"Batch feature standardization network with triplet loss": Utiliza explícitamente la pérdida triplet y, para ello, clasifica las "bolsas" de datos en categorías como "bolsas positivas fuertes", "bolsas positivas débiles" y "bolsas negativas", que funcionan como ancla (anchor), positivo y negativo en el entrenamiento.

"A novel triplet loss architecture...": También se centra en una arquitectura basada en triplet loss para detectar cambios, lo cual implica inherentemente la comparación entre tripletas de datos (ancla, positivo, negativo).
    \let\cleardoublepage\clearpage
    
    % Anexos (Opcional)
    % \chapter*{Anexos}
    % \section{Gráficas de los Experimentos Realizados}

\subsection{Experimentos del dataset de WLSL}

\subsubsection{Con 2 etiquetas}

\begin{figure}[H]
    \caption{Esta gráfica compara el ratio Semántico del modelo comparado con un \enquote{baseline} o punto de referencia. Siendo este el modelo sin entrenar. El eje Y representa el ratio semántico, que mide la capacidad del modelo para diferenciar entre palabras diferentes, mientras que el eje X representa las épocas de entrenamiento.}
    \centering
    \includegraphics[width=0.8\textwidth]{Images/Imagenes Cap 2/GraficasExperimentos/WLSL/2gloss/baseline1.png}
    \label{fig:WLSLG2baseline1}
\end{figure}

\begin{figure}[H]
    \caption{Esta gráfica evalúa la capacidad del modelo para entender el orden temporal de las secuencias de video comparadas con sus respectivos \enquote{baselines} del modelo no entrenado. Muestra la distancia euclidiana promedio entre la secuencia original y sus versiones alteradas (shifted, inverted, permuted).}
    \centering
    \includegraphics[width=0.8\textwidth]{Images/Imagenes Cap 2/GraficasExperimentos/WLSL/2gloss/baseline2.png}
    \label{fig:WLSLG2baseline2}
\end{figure}

\begin{figure}[H]
    \caption{Esta gráfica compara el rendimiento del modelo principal con un modelo más simple en términos de la pérdida total de validación. El eje Y representa el valor de la pérdida, una métrica que indica cuán bien el modelo está aprendiendo, donde los valores más bajos son mejores, y el eje X representa las épocas.}
    \centering
    \includegraphics[width=0.8\textwidth]{Images/Imagenes Cap 2/GraficasExperimentos/WLSL/2gloss/baseline3.png}
    \label{fig:WLSLG2baseline3}
\end{figure}

\begin{figure}[H]
    \caption{Esta gráfica compara el ratio semántico del modelo comparado con dos \enquote{baselines} o puntos de referencia. Siendo estos el modelo sin entrenar y la representación de PCA. El eje Y representa el ratio semántico, que mide la capacidad del modelo para diferenciar entre palabras diferentes, mientras que el eje X representa las épocas de entrenamiento.}
    \centering
    \includegraphics[width=0.8\textwidth]{Images/Imagenes Cap 2/GraficasExperimentos/WLSL/2gloss/baseline4.png}
    \label{fig:WLSLG2baseline4}
\end{figure}

\begin{figure}[H]
    \caption{Este gráfico mide directamente la calidad de la separación semántica en el espacio latente para palabras con la misma y diferente clase. El eje Y representa la distancia euclidiana promedio y el eje X son las épocas.}
    \centering
    \includegraphics[width=0.8\textwidth]{Images/Imagenes Cap 2/GraficasExperimentos/WLSL/2gloss/gen1.PNG}
    \label{fig:WLSLG2gen1}
\end{figure}

\begin{figure}[H]
    \caption{Este gráfico muestra la evolución de la \enquote{Pérdida Total} a lo largo de 100 épocas de entrenamiento. El eje Y representa el valor de la pérdida, una métrica que indica cuán bien el modelo está aprendiendo donde valores más bajos son mejores. El eje X representa las épocas, es decir, cada ciclo completo de entrenamiento sobre el conjunto de datos.}
    \centering
    \includegraphics[width=0.8\textwidth]{Images/Imagenes Cap 2/GraficasExperimentos/WLSL/2gloss/loss1.PNG}
    \label{fig:WLSLG2loss1}
\end{figure}

\begin{figure}[H]
    \caption{Esta gráfica ilustra la \enquote{Pérdida de Reconstrucción}, que mide qué tan bien el autoencoder del modelo puede reconstruir la entrada original después de haberla comprimido en un espacio latente. Al igual que en la gráfica anterior, el eje Y es el valor de la pérdida y el eje X son las épocas.}
    \centering
    \includegraphics[width=0.8\textwidth]{Images/Imagenes Cap 2/GraficasExperimentos/WLSL/2gloss/loss2.PNG}
    \label{fig:WLSLG2loss2}
\end{figure}

\begin{figure}[H]
    \caption{Este gráfico muestra la \enquote{Pérdida Triplet Semántica}, una métrica clave que evalúa si el modelo puede diferenciar entre distintos glosarios, en este caso, las señas \enquote{brother} y \enquote{cold}. El objetivo es que las representaciones de un mismo glosario estén más cerca entre sí que las de glosarios diferentes. Igualmente, el eje Y representa el valor de esta pérdida, mientras que el eje X indica las épocas de entrenamiento.}
    \centering
    \includegraphics[width=0.8\textwidth]{Images/Imagenes Cap 2/GraficasExperimentos/WLSL/2gloss/loss3.PNG}
    \label{fig:WLSLG2loss3}
\end{figure}

\begin{figure}[H]
    \caption{Esta visualización se enfoca en la sensibilidad temporal del modelo, puesto que, mide la diferencia entre la secuencia original de un video y su versión invertida. Esto quiere decir que el modelo debe aprender que una secuencia invertida es significativamente diferente de la original. El eje Y representa el valor de esta pérdida, mientras que el eje X indica las épocas de entrenamiento.}
    \centering
    \includegraphics[width=0.8\textwidth]{Images/Imagenes Cap 2/GraficasExperimentos/WLSL/2gloss/loss4.PNG}
    \label{fig:WLSLG2loss4}
\end{figure}

\begin{figure}[H]
    \caption{Similar a la gráfica anterior, esta también evalúa la sensibilidad temporal, pero en este caso, compara la secuencia original con una versión donde los fotogramas han sido desordenados aleatoriamente. Donde el objetivo es que el modelo reconozca que una secuencia permutada es muy diferente de la original. El eje Y representa el valor de esta pérdida, mientras que el eje X indica las épocas de entrenamiento.}
    \centering
    \includegraphics[width=0.8\textwidth]{Images/Imagenes Cap 2/GraficasExperimentos/WLSL/2gloss/loss5.PNG}
    \label{fig:WLSLG2loss5}
\end{figure}

\begin{figure}[H]
    \caption{Esta grafica muestra el espacio latente en la mejor epoca (100) utilizando pca, donde las señas \enquote{brother} estan con color azul y \enquote{cold} en naranja. Las diferentes formas de los puntos representan una variante diferente del video, siendo el circulo el original, la cruz el desplazado, el cuadrado el desordenado, y la equis la invertida.}
    \centering
    \includegraphics[width=0.8\textwidth]{Images/Imagenes Cap 2/GraficasExperimentos/WLSL/2gloss/elpca0.PNG}
    \label{fig:WLSLG2elpca0}
\end{figure}

\begin{figure}[H]
    \caption{Esta grafica muestra el espacio latente en la mejor epoca (100) utilizando umap, donde las señas \enquote{brother} estan con color azul y \enquote{cold} en naranja. Las diferentes formas de los puntos representan una variante diferente del video, siendo el circulo el original, la cruz el desplazado, el cuadrado el desordenado, y la equis la invertida.}
    \centering
    \includegraphics[width=0.8\textwidth]{Images/Imagenes Cap 2/GraficasExperimentos/WLSL/2gloss/elumap0.PNG}
    \label{fig:WLSLG2elumap0}
\end{figure}

\begin{figure}[H]
    \caption{Esta grafica muestra el espacio latente en la epoca 5 utilizando pca, donde las señas \enquote{brother} estan con color azul y \enquote{cold} en naranja. Las diferentes formas de los puntos representan una variante diferente del video, siendo el circulo el original, la cruz el desplazado, el cuadrado el desordenado, y la equis la invertida.}
    \centering
    \includegraphics[width=0.8\textwidth]{Images/Imagenes Cap 2/GraficasExperimentos/WLSL/2gloss/elpca1.PNG}
    \label{fig:WLSLG2elpca1}
\end{figure}

\begin{figure}[H]
    \caption{Esta grafica muestra el espacio latente en la epoca 5 utilizando umap, donde las señas \enquote{brother} estan con color azul y \enquote{cold} en naranja. Las diferentes formas de los puntos representan una variante diferente del video, siendo el circulo el original, la cruz el desplazado, el cuadrado el desordenado, y la equis la invertida.}
    \centering
    \includegraphics[width=0.8\textwidth]{Images/Imagenes Cap 2/GraficasExperimentos/WLSL/2gloss/elumap1.PNG}
    \label{fig:WLSLG2elumap1}
\end{figure}

\begin{figure}[H]
    \caption{Esta grafica muestra el espacio latente en la epoca 15 utilizando pca, donde las señas \enquote{brother} estan con color azul y \enquote{cold} en naranja. Las diferentes formas de los puntos representan una variante diferente del video, siendo el circulo el original, la cruz el desplazado, el cuadrado el desordenado, y la equis la invertida.}
    \centering
    \includegraphics[width=0.8\textwidth]{Images/Imagenes Cap 2/GraficasExperimentos/WLSL/2gloss/elpca2.PNG}
    \label{fig:WLSLG2elpca1}
\end{figure}

\begin{figure}[H]
    \caption{Esta grafica muestra el espacio latente en la epoca 15 utilizando umap, donde las señas \enquote{brother} estan con color azul y \enquote{cold} en naranja. Las diferentes formas de los puntos representan una variante diferente del video, siendo el circulo el original, la cruz el desplazado, el cuadrado el desordenado, y la equis la invertida.}
    \centering
    \includegraphics[width=0.8\textwidth]{Images/Imagenes Cap 2/GraficasExperimentos/WLSL/2gloss/elumap2.PNG}
    \label{fig:WLSLG2elumap2}
\end{figure}

\begin{figure}[H]
    \caption{Esta grafica muestra el espacio latente en la epoca 25 utilizando pca, donde las señas \enquote{brother} estan con color azul y \enquote{cold} en naranja. Las diferentes formas de los puntos representan una variante diferente del video, siendo el circulo el original, la cruz el desplazado, el cuadrado el desordenado, y la equis la invertida.}
    \centering
    \includegraphics[width=0.8\textwidth]{Images/Imagenes Cap 2/GraficasExperimentos/WLSL/2gloss/elpca3.PNG}
    \label{fig:WLSLG2elpca3}
\end{figure}

\begin{figure}[H]
    \caption{Esta grafica muestra el espacio latente en la epoca 25 utilizando umap, donde las señas \enquote{brother} estan con color azul y \enquote{cold} en naranja. Las diferentes formas de los puntos representan una variante diferente del video, siendo el circulo el original, la cruz el desplazado, el cuadrado el desordenado, y la equis la invertida.}
    \centering
    \includegraphics[width=0.8\textwidth]{Images/Imagenes Cap 2/GraficasExperimentos/WLSL/2gloss/elumap3.PNG}
    \label{fig:WLSLG2elumap3}
\end{figure}

\begin{figure}[H]
    \caption{Esta grafica muestra el espacio latente en la epoca 45 utilizando pca, donde las señas \enquote{brother} estan con color azul y \enquote{cold} en naranja. Las diferentes formas de los puntos representan una variante diferente del video, siendo el circulo el original, la cruz el desplazado, el cuadrado el desordenado, y la equis la invertida.}
    \centering
    \includegraphics[width=0.8\textwidth]{Images/Imagenes Cap 2/GraficasExperimentos/WLSL/2gloss/elpca4.PNG}
    \label{fig:WLSLG2elpca4}
\end{figure}

\begin{figure}[H]
    \caption{Esta grafica muestra el espacio latente en la epoca 45 utilizando umap, donde las señas \enquote{brother} estan con color azul y \enquote{cold} en naranja. Las diferentes formas de los puntos representan una variante diferente del video, siendo el circulo el original, la cruz el desplazado, el cuadrado el desordenado, y la equis la invertida.}
    \centering
    \includegraphics[width=0.8\textwidth]{Images/Imagenes Cap 2/GraficasExperimentos/WLSL/2gloss/elumap4.PNG}
    \label{fig:WLSLG2elumap4}
\end{figure}

\begin{figure}[H]
    \caption{Esta grafica muestra el espacio latente en la epoca 65 utilizando pca, donde las señas \enquote{brother} estan con color azul y \enquote{cold} en naranja. Las diferentes formas de los puntos representan una variante diferente del video, siendo el circulo el original, la cruz el desplazado, el cuadrado el desordenado, y la equis la invertida.}
    \centering
    \includegraphics[width=0.8\textwidth]{Images/Imagenes Cap 2/GraficasExperimentos/WLSL/2gloss/elpca5.PNG}
    \label{fig:WLSLG2elpca5}
\end{figure}

\begin{figure}[H]
    \caption{Esta grafica muestra el espacio latente en la epoca 65 utilizando umap, donde las señas \enquote{brother} estan con color azul y \enquote{cold} en naranja. Las diferentes formas de los puntos representan una variante diferente del video, siendo el circulo el original, la cruz el desplazado, el cuadrado el desordenado, y la equis la invertida.}
    \centering
    \includegraphics[width=0.8\textwidth]{Images/Imagenes Cap 2/GraficasExperimentos/WLSL/2gloss/elumap5.PNG}
    \label{fig:WLSLG2elumap5}
\end{figure}

\begin{figure}[H]
    \caption{Esta grafica muestra el espacio latente en la epoca 85 utilizando pca, donde las señas \enquote{brother} estan con color azul y \enquote{cold} en naranja. Las diferentes formas de los puntos representan una variante diferente del video, siendo el circulo el original, la cruz el desplazado, el cuadrado el desordenado, y la equis la invertida.}
    \centering
    \includegraphics[width=0.8\textwidth]{Images/Imagenes Cap 2/GraficasExperimentos/WLSL/2gloss/elpca6.PNG}
    \label{fig:WLSLG2elpca6}
\end{figure}

\begin{figure}[H]
    \caption{Esta grafica muestra el espacio latente en la epoca 85 utilizando umap, donde las señas \enquote{brother} estan con color azul y \enquote{cold} en naranja. Las diferentes formas de los puntos representan una variante diferente del video, siendo el circulo el original, la cruz el desplazado, el cuadrado el desordenado, y la equis la invertida.}
    \centering
    \includegraphics[width=0.8\textwidth]{Images/Imagenes Cap 2/GraficasExperimentos/WLSL/2gloss/elumap6.PNG}
    \label{fig:WLSLG2elumap6}
\end{figure}

\subsubsection{Con 3 etiquetas}

\begin{figure}[H]
    \caption{Esta gráfica compara el ratio Semántico del modelo comparado con un \enquote{baseline} o punto de referencia. Siendo este el modelo sin entrenar. El eje Y representa el ratio semántico, que mide la capacidad del modelo para diferenciar entre palabras diferentes, mientras que el eje X representa las épocas de entrenamiento.}
    \centering
    \includegraphics[width=0.8\textwidth]{Images/Imagenes Cap 2/GraficasExperimentos/WLSL/3gloss/baseline1.png}
    \label{fig:WLSLG3baseline1}
\end{figure}

\begin{figure}[H]
    \caption{Esta gráfica evalúa la capacidad del modelo para entender el orden temporal de las secuencias de video comparadas con sus respectivos \enquote{baselines} del modelo no entrenado. Muestra la distancia euclidiana promedio entre la secuencia original y sus versiones alteradas (shifted, inverted, permuted).}
    \centering
    \includegraphics[width=0.8\textwidth]{Images/Imagenes Cap 2/GraficasExperimentos/WLSL/3gloss/baseline2.png}
    \label{fig:WLSLG3baseline2}
\end{figure}

\begin{figure}[H]
    \caption{Esta gráfica compara el rendimiento del modelo principal con un modelo más simple en términos de la pérdida total de validación. El eje Y representa el valor de la pérdida, una métrica que indica cuán bien el modelo está aprendiendo, donde los valores más bajos son mejores, y el eje X representa las épocas.}
    \centering
    \includegraphics[width=0.8\textwidth]{Images/Imagenes Cap 2/GraficasExperimentos/WLSL/3gloss/baseline3.png}
    \label{fig:WLSLG3baseline3}
\end{figure}

\begin{figure}[H]
    \caption{Esta gráfica compara el ratio semántico del modelo comparado con dos \enquote{baselines} o puntos de referencia. Siendo estos el modelo sin entrenar y la representación de PCA. El eje Y representa el ratio semántico, que mide la capacidad del modelo para diferenciar entre palabras diferentes, mientras que el eje X representa las épocas de entrenamiento.}
    \centering
    \includegraphics[width=0.8\textwidth]{Images/Imagenes Cap 2/GraficasExperimentos/WLSL/3gloss/baseline4.png}
    \label{fig:WLSLG3baseline4}
\end{figure}

\begin{figure}[H]
    \caption{Este gráfico mide directamente la calidad de la separación semántica en el espacio latente para palabras con la misma y diferente clase. El eje Y representa la distancia euclidiana promedio y el eje X son las épocas.}
    \centering
    \includegraphics[width=0.8\textwidth]{Images/Imagenes Cap 2/GraficasExperimentos/WLSL/3gloss/gen1.PNG}
    \label{fig:WLSLG3gen1}
\end{figure}

\begin{figure}[H]
    \caption{Este gráfico muestra la evolución de la \enquote{Pérdida Total} a lo largo de 100 épocas de entrenamiento. El eje Y representa el valor de la pérdida, una métrica que indica cuán bien el modelo está aprendiendo donde valores más bajos son mejores. El eje X representa las épocas, es decir, cada ciclo completo de entrenamiento sobre el conjunto de datos.}
    \centering
    \includegraphics[width=0.8\textwidth]{Images/Imagenes Cap 2/GraficasExperimentos/WLSL/3gloss/loss1.PNG}
    \label{fig:WLSLG3loss1}
\end{figure}

\begin{figure}[H]
    \caption{Esta gráfica ilustra la \enquote{Pérdida de Reconstrucción}, que mide qué tan bien el autoencoder del modelo puede reconstruir la entrada original después de haberla comprimido en un espacio latente. Al igual que en la gráfica anterior, el eje Y es el valor de la pérdida y el eje X son las épocas.}
    \centering
    \includegraphics[width=0.8\textwidth]{Images/Imagenes Cap 2/GraficasExperimentos/WLSL/3gloss/loss2.PNG}
    \label{fig:WLSLG3loss2}
\end{figure}

\begin{figure}[H]
    \caption{Este gráfico muestra la \enquote{Pérdida Triplet Semántica}, una métrica clave que evalúa si el modelo puede diferenciar entre distintos glosarios, en este caso, las señas \enquote{brother}, \enquote{cold} y \enquote{man}. El objetivo es que las representaciones de un mismo glosario estén más cerca entre sí que las de glosarios diferentes. Igualmente, el eje Y representa el valor de esta pérdida, mientras que el eje X indica las épocas de entrenamiento.}
    \centering
    \includegraphics[width=0.8\textwidth]{Images/Imagenes Cap 2/GraficasExperimentos/WLSL/3gloss/loss3.PNG}
    \label{fig:WLSLG3loss3}
\end{figure}

\begin{figure}[H]
    \caption{Esta visualización se enfoca en la sensibilidad temporal del modelo, puesto que, mide la diferencia entre la secuencia original de un video y su versión invertida. Esto quiere decir que el modelo debe aprender que una secuencia invertida es significativamente diferente de la original. El eje Y representa el valor de esta pérdida, mientras que el eje X indica las épocas de entrenamiento.}
    \centering
    \includegraphics[width=0.8\textwidth]{Images/Imagenes Cap 2/GraficasExperimentos/WLSL/3gloss/loss4.PNG}
    \label{fig:WLSLG3loss4}
\end{figure}

\begin{figure}[H]
    \caption{Similar a la gráfica anterior, esta también evalúa la sensibilidad temporal, pero en este caso, compara la secuencia original con una versión donde los fotogramas han sido desordenados aleatoriamente. Donde el objetivo es que el modelo reconozca que una secuencia permutada es muy diferente de la original. El eje Y representa el valor de esta pérdida, mientras que el eje X indica las épocas de entrenamiento.}
    \centering
    \includegraphics[width=0.8\textwidth]{Images/Imagenes Cap 2/GraficasExperimentos/WLSL/3gloss/loss5.PNG}
    \label{fig:WLSLG3loss5}
\end{figure}

\begin{figure}[H]
    \caption{Esta grafica muestra el espacio latente en la mejor epoca (100) utilizando pca, donde las señas \enquote{brother} estan con color azul, \enquote{cold} en naranja y \enquote{man} en verde. Las diferentes formas de los puntos representan una variante diferente del video, siendo el circulo el original, la cruz el desplazado, el cuadrado el desordenado, y la equis la invertida.}
    \centering
    \includegraphics[width=0.8\textwidth]{Images/Imagenes Cap 2/GraficasExperimentos/WLSL/3gloss/elpca0.PNG}
    \label{fig:WLSLG3elpca0}
\end{figure}

\begin{figure}[H]
    \caption{Esta grafica muestra el espacio latente en la mejor epoca (100) utilizando umap, donde las señas \enquote{brother} estan con color azul, \enquote{cold} en naranja y \enquote{man} en verde. Las diferentes formas de los puntos representan una variante diferente del video, siendo el circulo el original, la cruz el desplazado, el cuadrado el desordenado, y la equis la invertida.}
    \centering
    \includegraphics[width=0.8\textwidth]{Images/Imagenes Cap 2/GraficasExperimentos/WLSL/3gloss/elumap0.PNG}
    \label{fig:WLSLG3elumap0}
\end{figure}

\begin{figure}[H]
    \caption{Esta grafica muestra el espacio latente en la epoca 5 utilizando pca, donde las señas \enquote{brother} estan con color azul, \enquote{cold} en naranja y \enquote{man} en verde. Las diferentes formas de los puntos representan una variante diferente del video, siendo el circulo el original, la cruz el desplazado, el cuadrado el desordenado, y la equis la invertida.}
    \centering
    \includegraphics[width=0.8\textwidth]{Images/Imagenes Cap 2/GraficasExperimentos/WLSL/3gloss/elpca1.PNG}
    \label{fig:WLSLG3elpca1}
\end{figure}

\begin{figure}[H]
    \caption{Esta grafica muestra el espacio latente en la epoca 5 utilizando umap, donde las señas \enquote{brother} estan con color azul, \enquote{cold} en naranja y \enquote{man} en verde. Las diferentes formas de los puntos representan una variante diferente del video, siendo el circulo el original, la cruz el desplazado, el cuadrado el desordenado, y la equis la invertida.}
    \centering
    \includegraphics[width=0.8\textwidth]{Images/Imagenes Cap 2/GraficasExperimentos/WLSL/3gloss/elumap1.PNG}
    \label{fig:WLSLG3elumap1}
\end{figure}

\begin{figure}[H]
    \caption{Esta grafica muestra el espacio latente en la epoca 15 utilizando pca, donde las señas \enquote{brother} estan con color azul, \enquote{cold} en naranja y \enquote{man} en verde. Las diferentes formas de los puntos representan una variante diferente del video, siendo el circulo el original, la cruz el desplazado, el cuadrado el desordenado, y la equis la invertida.}
    \centering
    \includegraphics[width=0.8\textwidth]{Images/Imagenes Cap 2/GraficasExperimentos/WLSL/3gloss/elpca2.PNG}
    \label{fig:WLSLG3elpca1}
\end{figure}

\begin{figure}[H]
    \caption{Esta grafica muestra el espacio latente en la epoca 15 utilizando umap, donde las señas \enquote{brother} estan con color azul, \enquote{cold} en naranja y \enquote{man} en verde. Las diferentes formas de los puntos representan una variante diferente del video, siendo el circulo el original, la cruz el desplazado, el cuadrado el desordenado, y la equis la invertida.}
    \centering
    \includegraphics[width=0.8\textwidth]{Images/Imagenes Cap 2/GraficasExperimentos/WLSL/3gloss/elumap2.PNG}
    \label{fig:WLSLG3elumap2}
\end{figure}

\begin{figure}[H]
    \caption{Esta grafica muestra el espacio latente en la epoca 25 utilizando pca, donde las señas \enquote{brother} estan con color azul, \enquote{cold} en naranja y \enquote{man} en verde. Las diferentes formas de los puntos representan una variante diferente del video, siendo el circulo el original, la cruz el desplazado, el cuadrado el desordenado, y la equis la invertida.}
    \centering
    \includegraphics[width=0.8\textwidth]{Images/Imagenes Cap 2/GraficasExperimentos/WLSL/3gloss/elpca3.PNG}
    \label{fig:WLSLG3elpca3}
\end{figure}

\begin{figure}[H]
    \caption{Esta grafica muestra el espacio latente en la epoca 25 utilizando umap, donde las señas \enquote{brother} estan con color azul, \enquote{cold} en naranja y \enquote{man} en verde. Las diferentes formas de los puntos representan una variante diferente del video, siendo el circulo el original, la cruz el desplazado, el cuadrado el desordenado, y la equis la invertida.}
    \centering
    \includegraphics[width=0.8\textwidth]{Images/Imagenes Cap 2/GraficasExperimentos/WLSL/3gloss/elumap3.PNG}
    \label{fig:WLSLG3elumap3}
\end{figure}

\begin{figure}[H]
    \caption{Esta grafica muestra el espacio latente en la epoca 45 utilizando pca, donde las señas \enquote{brother} estan con color azul, \enquote{cold} en naranja y \enquote{man} en verde. Las diferentes formas de los puntos representan una variante diferente del video, siendo el circulo el original, la cruz el desplazado, el cuadrado el desordenado, y la equis la invertida.}
    \centering
    \includegraphics[width=0.8\textwidth]{Images/Imagenes Cap 2/GraficasExperimentos/WLSL/3gloss/elpca4.PNG}
    \label{fig:WLSLG3elpca4}
\end{figure}

\begin{figure}[H]
    \caption{Esta grafica muestra el espacio latente en la epoca 45 utilizando umap, donde las señas \enquote{brother} estan con color azul, \enquote{cold} en naranja y \enquote{man} en verde. Las diferentes formas de los puntos representan una variante diferente del video, siendo el circulo el original, la cruz el desplazado, el cuadrado el desordenado, y la equis la invertida.}
    \centering
    \includegraphics[width=0.8\textwidth]{Images/Imagenes Cap 2/GraficasExperimentos/WLSL/3gloss/elumap4.PNG}
    \label{fig:WLSLG3elumap4}
\end{figure}

\begin{figure}[H]
    \caption{Esta grafica muestra el espacio latente en la epoca 65 utilizando pca, donde las señas \enquote{brother} estan con color azul, \enquote{cold} en naranja y \enquote{man} en verde. Las diferentes formas de los puntos representan una variante diferente del video, siendo el circulo el original, la cruz el desplazado, el cuadrado el desordenado, y la equis la invertida.}
    \centering
    \includegraphics[width=0.8\textwidth]{Images/Imagenes Cap 2/GraficasExperimentos/WLSL/3gloss/elpca5.PNG}
    \label{fig:WLSLG3elpca5}
\end{figure}

\begin{figure}[H]
    \caption{Esta grafica muestra el espacio latente en la epoca 65 utilizando umap, donde las señas \enquote{brother} estan con color azul, \enquote{cold} en naranja y \enquote{man} en verde. Las diferentes formas de los puntos representan una variante diferente del video, siendo el circulo el original, la cruz el desplazado, el cuadrado el desordenado, y la equis la invertida.}
    \centering
    \includegraphics[width=0.8\textwidth]{Images/Imagenes Cap 2/GraficasExperimentos/WLSL/3gloss/elumap5.PNG}
    \label{fig:WLSLG3elumap5}
\end{figure}

\begin{figure}[H]
    \caption{Esta grafica muestra el espacio latente en la epoca 85 utilizando pca, donde las señas \enquote{brother} estan con color azul, \enquote{cold} en naranja y \enquote{man} en verde. Las diferentes formas de los puntos representan una variante diferente del video, siendo el circulo el original, la cruz el desplazado, el cuadrado el desordenado, y la equis la invertida.}
    \centering
    \includegraphics[width=0.8\textwidth]{Images/Imagenes Cap 2/GraficasExperimentos/WLSL/3gloss/elpca6.PNG}
    \label{fig:WLSLG3elpca6}
\end{figure}

\begin{figure}[H]
    \caption{Esta grafica muestra el espacio latente en la epoca 85 utilizando umap, donde las señas \enquote{brother} estan con color azul, \enquote{cold} en naranja y \enquote{man} en verde. Las diferentes formas de los puntos representan una variante diferente del video, siendo el circulo el original, la cruz el desplazado, el cuadrado el desordenado, y la equis la invertida.}
    \centering
    \includegraphics[width=0.8\textwidth]{Images/Imagenes Cap 2/GraficasExperimentos/WLSL/3gloss/elumap6.PNG}
    \label{fig:WLSLG3elumap6}
\end{figure}

\iffalse

\subsubsection{Con 5 etiquetas}

\begin{figure}[H]
    \caption{Esta gráfica compara el ratio Semántico del modelo comparado con un \enquote{baseline} o punto de referencia. Siendo este el modelo sin entrenar. El eje Y representa el ratio semántico, que mide la capacidad del modelo para diferenciar entre palabras diferentes, mientras que el eje X representa las épocas de entrenamiento.}
    \centering
    \includegraphics[width=0.8\textwidth]{Images/Imagenes Cap 2/GraficasExperimentos/WLSL/5gloss/baseline1.png}
    \label{fig:WLSLG5baseline1}
\end{figure}

\begin{figure}[H]
    \caption{Esta gráfica evalúa la capacidad del modelo para entender el orden temporal de las secuencias de video comparadas con sus respectivos \enquote{baselines} del modelo no entrenado. Muestra la distancia euclidiana promedio entre la secuencia original y sus versiones alteradas (shifted, inverted, permuted).}
    \centering
    \includegraphics[width=0.8\textwidth]{Images/Imagenes Cap 2/GraficasExperimentos/WLSL/5gloss/baseline2.png}
    \label{fig:WLSLG5baseline2}
\end{figure}

\begin{figure}[H]
    \caption{Esta gráfica compara el rendimiento del modelo principal con un modelo más simple en términos de la pérdida total de validación. El eje Y representa el valor de la pérdida, una métrica que indica cuán bien el modelo está aprendiendo, donde los valores más bajos son mejores, y el eje X representa las épocas.}
    \centering
    \includegraphics[width=0.8\textwidth]{Images/Imagenes Cap 2/GraficasExperimentos/WLSL/5gloss/baseline3.png}
    \label{fig:WLSLG5baseline3}
\end{figure}

\begin{figure}[H]
    \caption{Esta gráfica compara el ratio semántico del modelo comparado con dos \enquote{baselines} o puntos de referencia. Siendo estos el modelo sin entrenar y la representación de PCA. El eje Y representa el ratio semántico, que mide la capacidad del modelo para diferenciar entre palabras diferentes, mientras que el eje X representa las épocas de entrenamiento.}
    \centering
    \includegraphics[width=0.8\textwidth]{Images/Imagenes Cap 2/GraficasExperimentos/WLSL/5gloss/baseline4.png}
    \label{fig:WLSLG5baseline4}
\end{figure}

\begin{figure}[H]
    \caption{Este gráfico mide directamente la calidad de la separación semántica en el espacio latente para palabras con la misma y diferente clase. El eje Y representa la distancia euclidiana promedio y el eje X son las épocas.}
    \centering
    \includegraphics[width=0.8\textwidth]{Images/Imagenes Cap 2/GraficasExperimentos/WLSL/5gloss/gen1.PNG}
    \label{fig:WLSLG5gen1}
\end{figure}

\begin{figure}[H]
    \caption{Este gráfico muestra la evolución de la \enquote{Pérdida Total} a lo largo de 100 épocas de entrenamiento. El eje Y representa el valor de la pérdida, una métrica que indica cuán bien el modelo está aprendiendo donde valores más bajos son mejores. El eje X representa las épocas, es decir, cada ciclo completo de entrenamiento sobre el conjunto de datos.}
    \centering
    \includegraphics[width=0.8\textwidth]{Images/Imagenes Cap 2/GraficasExperimentos/WLSL/5gloss/loss1.PNG}
    \label{fig:WLSLG5loss1}
\end{figure}

\begin{figure}[H]
    \caption{Esta gráfica ilustra la \enquote{Pérdida de Reconstrucción}, que mide qué tan bien el autoencoder del modelo puede reconstruir la entrada original después de haberla comprimido en un espacio latente. Al igual que en la gráfica anterior, el eje Y es el valor de la pérdida y el eje X son las épocas.}
    \centering
    \includegraphics[width=0.8\textwidth]{Images/Imagenes Cap 2/GraficasExperimentos/WLSL/5gloss/loss2.PNG}
    \label{fig:WLSLG5loss2}
\end{figure}

\begin{figure}[H]
    \caption{Este gráfico muestra la \enquote{Pérdida Triplet Semántica}, una métrica clave que evalúa si el modelo puede diferenciar entre distintos glosarios, en este caso, las señas \enquote{brother}, \enquote{cold}, \enquote{man}, \enquote{mother} y \enquote{short}. El objetivo es que las representaciones de un mismo glosario estén más cerca entre sí que las de glosarios diferentes. Igualmente, el eje Y representa el valor de esta pérdida, mientras que el eje X indica las épocas de entrenamiento.}
    \centering
    \includegraphics[width=0.8\textwidth]{Images/Imagenes Cap 2/GraficasExperimentos/WLSL/5gloss/loss3.PNG}
    \label{fig:WLSLG5loss3}
\end{figure}

\begin{figure}[H]
    \caption{Esta visualización se enfoca en la sensibilidad temporal del modelo, puesto que, mide la diferencia entre la secuencia original de un video y su versión invertida. Esto quiere decir que el modelo debe aprender que una secuencia invertida es significativamente diferente de la original. El eje Y representa el valor de esta pérdida, mientras que el eje X indica las épocas de entrenamiento.}
    \centering
    \includegraphics[width=0.8\textwidth]{Images/Imagenes Cap 2/GraficasExperimentos/WLSL/5gloss/loss4.PNG}
    \label{fig:WLSLG5loss4}
\end{figure}

\begin{figure}[H]
    \caption{Similar a la gráfica anterior, esta también evalúa la sensibilidad temporal, pero en este caso, compara la secuencia original con una versión donde los fotogramas han sido desordenados aleatoriamente. Donde el objetivo es que el modelo reconozca que una secuencia permutada es muy diferente de la original. El eje Y representa el valor de esta pérdida, mientras que el eje X indica las épocas de entrenamiento.}
    \centering
    \includegraphics[width=0.8\textwidth]{Images/Imagenes Cap 2/GraficasExperimentos/WLSL/5gloss/loss5.PNG}
    \label{fig:WLSLG5loss5}
\end{figure}

\begin{figure}[H]
    \caption{Esta grafica muestra el espacio latente en la mejor epoca (86) utilizando pca, donde las señas \enquote{brother} estan con color azul, \enquote{cold} en naranja, \enquote{man} en verde, \enquote{mother} en rojo y \enquote{short} en morado. Las diferentes formas de los puntos representan una variante diferente del video, siendo el circulo el original, la cruz el desplazado, el cuadrado el desordenado, y la equis la invertida.}
    \centering
    \includegraphics[width=0.8\textwidth]{Images/Imagenes Cap 2/GraficasExperimentos/WLSL/5gloss/elpca0.PNG}
    \label{fig:WLSLG5elpca0}
\end{figure}

\begin{figure}[H]
    \caption{Esta grafica muestra el espacio latente en la mejor epoca (86) utilizando umap, donde las señas \enquote{brother} estan con color azul, \enquote{cold} en naranja, \enquote{man} en verde, \enquote{mother} en rojo y \enquote{short} en morado. Las diferentes formas de los puntos representan una variante diferente del video, siendo el circulo el original, la cruz el desplazado, el cuadrado el desordenado, y la equis la invertida.}
    \centering
    \includegraphics[width=0.8\textwidth]{Images/Imagenes Cap 2/GraficasExperimentos/WLSL/5gloss/elumap0.PNG}
    \label{fig:WLSLG5elumap0}
\end{figure}

\begin{figure}[H]
    \caption{Esta grafica muestra el espacio latente en la epoca 5 utilizando pca, donde las señas \enquote{brother} estan con color azul, \enquote{cold} en naranja, \enquote{man} en verde, \enquote{mother} en rojo y \enquote{short} en morado. Las diferentes formas de los puntos representan una variante diferente del video, siendo el circulo el original, la cruz el desplazado, el cuadrado el desordenado, y la equis la invertida.}
    \centering
    \includegraphics[width=0.8\textwidth]{Images/Imagenes Cap 2/GraficasExperimentos/WLSL/5gloss/elpca1.PNG}
    \label{fig:WLSLG5elpca1}
\end{figure}

\begin{figure}[H]
    \caption{Esta grafica muestra el espacio latente en la epoca 5 utilizando umap, donde las señas \enquote{brother} estan con color azul, \enquote{cold} en naranja, \enquote{man} en verde, \enquote{mother} en rojo y \enquote{short} en morado. Las diferentes formas de los puntos representan una variante diferente del video, siendo el circulo el original, la cruz el desplazado, el cuadrado el desordenado, y la equis la invertida.}
    \centering
    \includegraphics[width=0.8\textwidth]{Images/Imagenes Cap 2/GraficasExperimentos/WLSL/5gloss/elumap1.PNG}
    \label{fig:WLSLG5elumap1}
\end{figure}

\begin{figure}[H]
    \caption{Esta grafica muestra el espacio latente en la epoca 15 utilizando pca, donde las señas \enquote{brother} estan con color azul, \enquote{cold} en naranja, \enquote{man} en verde, \enquote{mother} en rojo y \enquote{short} en morado. Las diferentes formas de los puntos representan una variante diferente del video, siendo el circulo el original, la cruz el desplazado, el cuadrado el desordenado, y la equis la invertida.}
    \centering
    \includegraphics[width=0.8\textwidth]{Images/Imagenes Cap 2/GraficasExperimentos/WLSL/5gloss/elpca2.PNG}
    \label{fig:WLSLG5elpca1}
\end{figure}

\begin{figure}[H]
    \caption{Esta grafica muestra el espacio latente en la epoca 15 utilizando umap, donde las señas \enquote{brother} estan con color azul, \enquote{cold} en naranja, \enquote{man} en verde, \enquote{mother} en rojo y \enquote{short} en morado. Las diferentes formas de los puntos representan una variante diferente del video, siendo el circulo el original, la cruz el desplazado, el cuadrado el desordenado, y la equis la invertida.}
    \centering
    \includegraphics[width=0.8\textwidth]{Images/Imagenes Cap 2/GraficasExperimentos/WLSL/5gloss/elumap2.PNG}
    \label{fig:WLSLG5elumap2}
\end{figure}

\begin{figure}[H]
    \caption{Esta grafica muestra el espacio latente en la epoca 25 utilizando pca, donde las señas \enquote{brother} estan con color azul, \enquote{cold} en naranja, \enquote{man} en verde, \enquote{mother} en rojo y \enquote{short} en morado. Las diferentes formas de los puntos representan una variante diferente del video, siendo el circulo el original, la cruz el desplazado, el cuadrado el desordenado, y la equis la invertida.}
    \centering
    \includegraphics[width=0.8\textwidth]{Images/Imagenes Cap 2/GraficasExperimentos/WLSL/5gloss/elpca3.PNG}
    \label{fig:WLSLG5elpca3}
\end{figure}

\begin{figure}[H]
    \caption{Esta grafica muestra el espacio latente en la epoca 25 utilizando umap, donde las señas \enquote{brother} estan con color azul, \enquote{cold} en naranja, \enquote{man} en verde, \enquote{mother} en rojo y \enquote{short} en morado. Las diferentes formas de los puntos representan una variante diferente del video, siendo el circulo el original, la cruz el desplazado, el cuadrado el desordenado, y la equis la invertida.}
    \centering
    \includegraphics[width=0.8\textwidth]{Images/Imagenes Cap 2/GraficasExperimentos/WLSL/5gloss/elumap3.PNG}
    \label{fig:WLSLG5elumap3}
\end{figure}

\begin{figure}[H]
    \caption{Esta grafica muestra el espacio latente en la epoca 45 utilizando pca, donde las señas \enquote{brother} estan con color azul, \enquote{cold} en naranja, \enquote{man} en verde, \enquote{mother} en rojo y \enquote{short} en morado. Las diferentes formas de los puntos representan una variante diferente del video, siendo el circulo el original, la cruz el desplazado, el cuadrado el desordenado, y la equis la invertida.}
    \centering
    \includegraphics[width=0.8\textwidth]{Images/Imagenes Cap 2/GraficasExperimentos/WLSL/5gloss/elpca4.PNG}
    \label{fig:WLSLG5elpca4}
\end{figure}

\begin{figure}[H]
    \caption{Esta grafica muestra el espacio latente en la epoca 45 utilizando umap, donde las señas \enquote{brother} estan con color azul, \enquote{cold} en naranja, \enquote{man} en verde, \enquote{mother} en rojo y \enquote{short} en morado. Las diferentes formas de los puntos representan una variante diferente del video, siendo el circulo el original, la cruz el desplazado, el cuadrado el desordenado, y la equis la invertida.}
    \centering
    \includegraphics[width=0.8\textwidth]{Images/Imagenes Cap 2/GraficasExperimentos/WLSL/5gloss/elumap4.PNG}
    \label{fig:WLSLG5elumap4}
\end{figure}

\begin{figure}[H]
    \caption{Esta grafica muestra el espacio latente en la epoca 65 utilizando pca, donde las señas \enquote{brother} estan con color azul, \enquote{cold} en naranja, \enquote{man} en verde, \enquote{mother} en rojo y \enquote{short} en morado. Las diferentes formas de los puntos representan una variante diferente del video, siendo el circulo el original, la cruz el desplazado, el cuadrado el desordenado, y la equis la invertida.}
    \centering
    \includegraphics[width=0.8\textwidth]{Images/Imagenes Cap 2/GraficasExperimentos/WLSL/5gloss/elpca5.PNG}
    \label{fig:WLSLG5elpca5}
\end{figure}

\begin{figure}[H]
    \caption{Esta grafica muestra el espacio latente en la epoca 65 utilizando umap, donde las señas \enquote{brother} estan con color azul, \enquote{cold} en naranja, \enquote{man} en verde, \enquote{mother} en rojo y \enquote{short} en morado. Las diferentes formas de los puntos representan una variante diferente del video, siendo el circulo el original, la cruz el desplazado, el cuadrado el desordenado, y la equis la invertida.}
    \centering
    \includegraphics[width=0.8\textwidth]{Images/Imagenes Cap 2/GraficasExperimentos/WLSL/5gloss/elumap5.PNG}
    \label{fig:WLSLG5elumap5}
\end{figure}

\begin{figure}[H]
    \caption{Esta grafica muestra el espacio latente en la epoca 85 utilizando pca, donde las señas \enquote{brother} estan con color azul, \enquote{cold} en naranja, \enquote{man} en verde, \enquote{mother} en rojo y \enquote{short} en morado. Las diferentes formas de los puntos representan una variante diferente del video, siendo el circulo el original, la cruz el desplazado, el cuadrado el desordenado, y la equis la invertida.}
    \centering
    \includegraphics[width=0.8\textwidth]{Images/Imagenes Cap 2/GraficasExperimentos/WLSL/5gloss/elpca6.PNG}
    \label{fig:WLSLG5elpca6}
\end{figure}

\begin{figure}[H]
    \caption{Esta grafica muestra el espacio latente en la epoca 85 utilizando umap, donde las señas \enquote{brother} estan con color azul, \enquote{cold} en naranja, \enquote{man} en verde, \enquote{mother} en rojo y \enquote{short} en morado. Las diferentes formas de los puntos representan una variante diferente del video, siendo el circulo el original, la cruz el desplazado, el cuadrado el desordenado, y la equis la invertida.}
    \centering
    \includegraphics[width=0.8\textwidth]{Images/Imagenes Cap 2/GraficasExperimentos/WLSL/5gloss/elumap6.PNG}
    \label{fig:WLSLG5elumap6}
\end{figure}

\subsubsection{Con 7 etiquetas}

\begin{figure}[H]
    \caption{Esta gráfica compara el ratio Semántico del modelo comparado con un \enquote{baseline} o punto de referencia. Siendo este el modelo sin entrenar. El eje Y representa el ratio semántico, que mide la capacidad del modelo para diferenciar entre palabras diferentes, mientras que el eje X representa las épocas de entrenamiento.}
    \centering
    \includegraphics[width=0.8\textwidth]{Images/Imagenes Cap 2/GraficasExperimentos/WLSL/7gloss/baseline1.png}
    \label{fig:WLSL75baseline1}
\end{figure}

\begin{figure}[H]
    \caption{Esta gráfica evalúa la capacidad del modelo para entender el orden temporal de las secuencias de video comparadas con sus respectivos \enquote{baselines} del modelo no entrenado. Muestra la distancia euclidiana promedio entre la secuencia original y sus versiones alteradas (shifted, inverted, permuted).}
    \centering
    \includegraphics[width=0.8\textwidth]{Images/Imagenes Cap 2/GraficasExperimentos/WLSL/7gloss/baseline2.png}
    \label{fig:WLSL75baseline2}
\end{figure}

\begin{figure}[H]
    \caption{Esta gráfica compara el rendimiento del modelo principal con un modelo más simple en términos de la pérdida total de validación. El eje Y representa el valor de la pérdida, una métrica que indica cuán bien el modelo está aprendiendo, donde los valores más bajos son mejores, y el eje X representa las épocas.}
    \centering
    \includegraphics[width=0.8\textwidth]{Images/Imagenes Cap 2/GraficasExperimentos/WLSL/7gloss/baseline3.png}
    \label{fig:WLSL75baseline3}
\end{figure}

\begin{figure}[H]
    \caption{Esta gráfica compara el ratio semántico del modelo comparado con dos \enquote{baselines} o puntos de referencia. Siendo estos el modelo sin entrenar y la representación de PCA. El eje Y representa el ratio semántico, que mide la capacidad del modelo para diferenciar entre palabras diferentes, mientras que el eje X representa las épocas de entrenamiento.}
    \centering
    \includegraphics[width=0.8\textwidth]{Images/Imagenes Cap 2/GraficasExperimentos/WLSL/7gloss/baseline4.png}
    \label{fig:WLSL75baseline4}
\end{figure}

\begin{figure}[H]
    \caption{Este gráfico mide directamente la calidad de la separación semántica en el espacio latente para palabras con la misma y diferente clase. El eje Y representa la distancia euclidiana promedio y el eje X son las épocas.}
    \centering
    \includegraphics[width=0.8\textwidth]{Images/Imagenes Cap 2/GraficasExperimentos/WLSL/7gloss/gen1.PNG}
    \label{fig:WLSL75gen1}
\end{figure}

\begin{figure}[H]
    \caption{Este gráfico muestra la evolución de la \enquote{Pérdida Total} a lo largo de 100 épocas de entrenamiento. El eje Y representa el valor de la pérdida, una métrica que indica cuán bien el modelo está aprendiendo donde valores más bajos son mejores. El eje X representa las épocas, es decir, cada ciclo completo de entrenamiento sobre el conjunto de datos.}
    \centering
    \includegraphics[width=0.8\textwidth]{Images/Imagenes Cap 2/GraficasExperimentos/WLSL/7gloss/loss1.PNG}
    \label{fig:WLSL75loss1}
\end{figure}

\begin{figure}[H]
    \caption{Esta gráfica ilustra la \enquote{Pérdida de Reconstrucción}, que mide qué tan bien el autoencoder del modelo puede reconstruir la entrada original después de haberla comprimido en un espacio latente. Al igual que en la gráfica anterior, el eje Y es el valor de la pérdida y el eje X son las épocas.}
    \centering
    \includegraphics[width=0.8\textwidth]{Images/Imagenes Cap 2/GraficasExperimentos/WLSL/7gloss/loss2.PNG}
    \label{fig:WLSL75loss2}
\end{figure}

\begin{figure}[H]
    \caption{Este gráfico muestra la \enquote{Pérdida Triplet Semántica}, una métrica clave que evalúa si el modelo puede diferenciar entre distintos glosarios, en este caso, las señas \enquote{brother}, \enquote{cold}, \enquote{dog}, \enquote{family}, \enquote{good}, \enquote{man} y \enquote{mother}. El objetivo es que las representaciones de un mismo glosario estén más cerca entre sí que las de glosarios diferentes. Igualmente, el eje Y representa el valor de esta pérdida, mientras que el eje X indica las épocas de entrenamiento.}
    \centering
    \includegraphics[width=0.8\textwidth]{Images/Imagenes Cap 2/GraficasExperimentos/WLSL/7gloss/loss3.PNG}
    \label{fig:WLSL75loss3}
\end{figure}

\begin{figure}[H]
    \caption{Esta visualización se enfoca en la sensibilidad temporal del modelo, puesto que, mide la diferencia entre la secuencia original de un video y su versión invertida. Esto quiere decir que el modelo debe aprender que una secuencia invertida es significativamente diferente de la original. El eje Y representa el valor de esta pérdida, mientras que el eje X indica las épocas de entrenamiento.}
    \centering
    \includegraphics[width=0.8\textwidth]{Images/Imagenes Cap 2/GraficasExperimentos/WLSL/7gloss/loss4.PNG}
    \label{fig:WLSL75loss4}
\end{figure}

\begin{figure}[H]
    \caption{Similar a la gráfica anterior, esta también evalúa la sensibilidad temporal, pero en este caso, compara la secuencia original con una versión donde los fotogramas han sido desordenados aleatoriamente. Donde el objetivo es que el modelo reconozca que una secuencia permutada es muy diferente de la original. El eje Y representa el valor de esta pérdida, mientras que el eje X indica las épocas de entrenamiento.}
    \centering
    \includegraphics[width=0.8\textwidth]{Images/Imagenes Cap 2/GraficasExperimentos/WLSL/7gloss/loss5.PNG}
    \label{fig:WLSL75loss5}
\end{figure}

\begin{figure}[H]
    \caption{Esta grafica muestra el espacio latente en la mejor epoca (100) utilizando pca, donde las señas \enquote{brother} estan con color azul, \enquote{cold} en naranja, \enquote{dog} en verde, \enquote{family} en rojo, \enquote{good} en morado, \enquote{man} en cafe y \enquote{mother} en rosado. Las diferentes formas de los puntos representan una variante diferente del video, siendo el circulo el original, la cruz el desplazado, el cuadrado el desordenado, y la equis la invertida.}
    \centering
    \includegraphics[width=0.8\textwidth]{Images/Imagenes Cap 2/GraficasExperimentos/WLSL/7gloss/elpca0.PNG}
    \label{fig:WLSL75elpca0}
\end{figure}

\begin{figure}[H]
    \caption{Esta grafica muestra el espacio latente en la mejor epoca (100) utilizando umap, donde las señas \enquote{brother} estan con color azul, \enquote{cold} en naranja, \enquote{dog} en verde, \enquote{family} en rojo, \enquote{good} en morado, \enquote{man} en cafe y \enquote{mother} en rosado. Las diferentes formas de los puntos representan una variante diferente del video, siendo el circulo el original, la cruz el desplazado, el cuadrado el desordenado, y la equis la invertida.}
    \centering
    \includegraphics[width=0.8\textwidth]{Images/Imagenes Cap 2/GraficasExperimentos/WLSL/7gloss/elumap0.PNG}
    \label{fig:WLSL75elumap0}
\end{figure}

\begin{figure}[H]
    \caption{Esta grafica muestra el espacio latente en la epoca 5 utilizando pca, donde las señas \enquote{brother} estan con color azul, \enquote{cold} en naranja, \enquote{dog} en verde, \enquote{family} en rojo, \enquote{good} en morado, \enquote{man} en cafe y \enquote{mother} en rosado. Las diferentes formas de los puntos representan una variante diferente del video, siendo el circulo el original, la cruz el desplazado, el cuadrado el desordenado, y la equis la invertida.}
    \centering
    \includegraphics[width=0.8\textwidth]{Images/Imagenes Cap 2/GraficasExperimentos/WLSL/7gloss/elpca1.PNG}
    \label{fig:WLSL75elpca1}
\end{figure}

\begin{figure}[H]
    \caption{Esta grafica muestra el espacio latente en la epoca 5 utilizando umap, donde las señas \enquote{brother} estan con color azul, \enquote{cold} en naranja, \enquote{dog} en verde, \enquote{family} en rojo, \enquote{good} en morado, \enquote{man} en cafe y \enquote{mother} en rosado. Las diferentes formas de los puntos representan una variante diferente del video, siendo el circulo el original, la cruz el desplazado, el cuadrado el desordenado, y la equis la invertida.}
    \centering
    \includegraphics[width=0.8\textwidth]{Images/Imagenes Cap 2/GraficasExperimentos/WLSL/7gloss/elumap1.PNG}
    \label{fig:WLSL75elumap1}
\end{figure}

\begin{figure}[H]
    \caption{Esta grafica muestra el espacio latente en la epoca 15 utilizando pca, donde las señas \enquote{brother} estan con color azul, \enquote{cold} en naranja, \enquote{dog} en verde, \enquote{family} en rojo, \enquote{good} en morado, \enquote{man} en cafe y \enquote{mother} en rosado. Las diferentes formas de los puntos representan una variante diferente del video, siendo el circulo el original, la cruz el desplazado, el cuadrado el desordenado, y la equis la invertida.}
    \centering
    \includegraphics[width=0.8\textwidth]{Images/Imagenes Cap 2/GraficasExperimentos/WLSL/7gloss/elpca2.PNG}
    \label{fig:WLSL75elpca1}
\end{figure}

\begin{figure}[H]
    \caption{Esta grafica muestra el espacio latente en la epoca 15 utilizando umap, donde las señas \enquote{brother} estan con color azul, \enquote{cold} en naranja, \enquote{dog} en verde, \enquote{family} en rojo, \enquote{good} en morado, \enquote{man} en cafe y \enquote{mother} en rosado. Las diferentes formas de los puntos representan una variante diferente del video, siendo el circulo el original, la cruz el desplazado, el cuadrado el desordenado, y la equis la invertida.}
    \centering
    \includegraphics[width=0.8\textwidth]{Images/Imagenes Cap 2/GraficasExperimentos/WLSL/7gloss/elumap2.PNG}
    \label{fig:WLSL75elumap2}
\end{figure}

\begin{figure}[H]
    \caption{Esta grafica muestra el espacio latente en la epoca 25 utilizando pca, donde las señas \enquote{brother} estan con color azul, \enquote{cold} en naranja, \enquote{dog} en verde, \enquote{family} en rojo, \enquote{good} en morado, \enquote{man} en cafe y \enquote{mother} en rosado. Las diferentes formas de los puntos representan una variante diferente del video, siendo el circulo el original, la cruz el desplazado, el cuadrado el desordenado, y la equis la invertida.}
    \centering
    \includegraphics[width=0.8\textwidth]{Images/Imagenes Cap 2/GraficasExperimentos/WLSL/7gloss/elpca3.PNG}
    \label{fig:WLSL75elpca3}
\end{figure}

\begin{figure}[H]
    \caption{Esta grafica muestra el espacio latente en la epoca 25 utilizando umap, donde las señas \enquote{brother} estan con color azul, \enquote{cold} en naranja, \enquote{dog} en verde, \enquote{family} en rojo, \enquote{good} en morado, \enquote{man} en cafe y \enquote{mother} en rosado. Las diferentes formas de los puntos representan una variante diferente del video, siendo el circulo el original, la cruz el desplazado, el cuadrado el desordenado, y la equis la invertida.}
    \centering
    \includegraphics[width=0.8\textwidth]{Images/Imagenes Cap 2/GraficasExperimentos/WLSL/7gloss/elumap3.PNG}
    \label{fig:WLSL75elumap3}
\end{figure}

\begin{figure}[H]
    \caption{Esta grafica muestra el espacio latente en la epoca 45 utilizando pca, donde las señas \enquote{brother} estan con color azul, \enquote{cold} en naranja, \enquote{dog} en verde, \enquote{family} en rojo, \enquote{good} en morado, \enquote{man} en cafe y \enquote{mother} en rosado. Las diferentes formas de los puntos representan una variante diferente del video, siendo el circulo el original, la cruz el desplazado, el cuadrado el desordenado, y la equis la invertida.}
    \centering
    \includegraphics[width=0.8\textwidth]{Images/Imagenes Cap 2/GraficasExperimentos/WLSL/7gloss/elpca4.PNG}
    \label{fig:WLSL75elpca4}
\end{figure}

\begin{figure}[H]
    \caption{Esta grafica muestra el espacio latente en la epoca 45 utilizando umap, donde las señas \enquote{brother} estan con color azul, \enquote{cold} en naranja, \enquote{dog} en verde, \enquote{family} en rojo, \enquote{good} en morado, \enquote{man} en cafe y \enquote{mother} en rosado. Las diferentes formas de los puntos representan una variante diferente del video, siendo el circulo el original, la cruz el desplazado, el cuadrado el desordenado, y la equis la invertida.}
    \centering
    \includegraphics[width=0.8\textwidth]{Images/Imagenes Cap 2/GraficasExperimentos/WLSL/7gloss/elumap4.PNG}
    \label{fig:WLSL75elumap4}
\end{figure}

\begin{figure}[H]
    \caption{Esta grafica muestra el espacio latente en la epoca 65 utilizando pca, donde las señas \enquote{brother} estan con color azul, \enquote{cold} en naranja, \enquote{dog} en verde, \enquote{family} en rojo, \enquote{good} en morado, \enquote{man} en cafe y \enquote{mother} en rosado. Las diferentes formas de los puntos representan una variante diferente del video, siendo el circulo el original, la cruz el desplazado, el cuadrado el desordenado, y la equis la invertida.}
    \centering
    \includegraphics[width=0.8\textwidth]{Images/Imagenes Cap 2/GraficasExperimentos/WLSL/7gloss/elpca5.PNG}
    \label{fig:WLSL75elpca5}
\end{figure}

\begin{figure}[H]
    \caption{Esta grafica muestra el espacio latente en la epoca 65 utilizando umap, donde las señas \enquote{brother} estan con color azul, \enquote{cold} en naranja, \enquote{dog} en verde, \enquote{family} en rojo, \enquote{good} en morado, \enquote{man} en cafe y \enquote{mother} en rosado. Las diferentes formas de los puntos representan una variante diferente del video, siendo el circulo el original, la cruz el desplazado, el cuadrado el desordenado, y la equis la invertida.}
    \centering
    \includegraphics[width=0.8\textwidth]{Images/Imagenes Cap 2/GraficasExperimentos/WLSL/7gloss/elumap5.PNG}
    \label{fig:WLSL75elumap5}
\end{figure}

\begin{figure}[H]
    \caption{Esta grafica muestra el espacio latente en la epoca 85 utilizando pca, donde las señas \enquote{brother} estan con color azul, \enquote{cold} en naranja, \enquote{dog} en verde, \enquote{family} en rojo, \enquote{good} en morado, \enquote{man} en cafe y \enquote{mother} en rosado. Las diferentes formas de los puntos representan una variante diferente del video, siendo el circulo el original, la cruz el desplazado, el cuadrado el desordenado, y la equis la invertida.}
    \centering
    \includegraphics[width=0.8\textwidth]{Images/Imagenes Cap 2/GraficasExperimentos/WLSL/7gloss/elpca6.PNG}
    \label{fig:WLSL75elpca6}
\end{figure}

\begin{figure}[H]
    \caption{Esta grafica muestra el espacio latente en la epoca 85 utilizando umap, donde las señas \enquote{brother} estan con color azul, \enquote{cold} en naranja, \enquote{dog} en verde, \enquote{family} en rojo, \enquote{good} en morado, \enquote{man} en cafe y \enquote{mother} en rosado. Las diferentes formas de los puntos representan una variante diferente del video, siendo el circulo el original, la cruz el desplazado, el cuadrado el desordenado, y la equis la invertida.}
    \centering
    \includegraphics[width=0.8\textwidth]{Images/Imagenes Cap 2/GraficasExperimentos/WLSL/7gloss/elumap6.PNG}
    \label{fig:WLSL75elumap6}
\end{figure}

\fi

\subsection{Con el dataset de ISL}

\subsubsection{Con 2 etiquetas}

\begin{figure}[H]
    \caption{Esta gráfica compara el ratio Semántico del modelo comparado con un \enquote{baseline} o punto de referencia. Siendo este el modelo sin entrenar. El eje Y representa el ratio semántico, que mide la capacidad del modelo para diferenciar entre palabras diferentes, mientras que el eje X representa las épocas de entrenamiento.}
    \centering
    \includegraphics[width=0.8\textwidth]{Images/Imagenes Cap 2/GraficasExperimentos/ISL/2gloss/baseline1.png}
    \label{fig:ISLG2baseline1}
\end{figure}

\begin{figure}[H]
    \caption{Esta gráfica evalúa la capacidad del modelo para entender el orden temporal de las secuencias de video comparadas con sus respectivos \enquote{baselines} del modelo no entrenado. Muestra la distancia euclidiana promedio entre la secuencia original y sus versiones alteradas (shifted, inverted, permuted).}
    \centering
    \includegraphics[width=0.8\textwidth]{Images/Imagenes Cap 2/GraficasExperimentos/ISL/2gloss/baseline2.png}
    \label{fig:ISLG2baseline2}
\end{figure}

\begin{figure}[H]
    \caption{Esta gráfica compara el rendimiento del modelo principal con un modelo más simple en términos de la pérdida total de validación. El eje Y representa el valor de la pérdida, una métrica que indica cuán bien el modelo está aprendiendo, donde los valores más bajos son mejores, y el eje X representa las épocas.}
    \centering
    \includegraphics[width=0.8\textwidth]{Images/Imagenes Cap 2/GraficasExperimentos/ISL/2gloss/baseline3.png}
    \label{fig:ISLG2baseline3}
\end{figure}

\begin{figure}[H]
    \caption{Esta gráfica compara el ratio semántico del modelo comparado con dos \enquote{baselines} o puntos de referencia. Siendo estos el modelo sin entrenar y la representación de PCA. El eje Y representa el ratio semántico, que mide la capacidad del modelo para diferenciar entre palabras diferentes, mientras que el eje X representa las épocas de entrenamiento.}
    \centering
    \includegraphics[width=0.8\textwidth]{Images/Imagenes Cap 2/GraficasExperimentos/ISL/2gloss/baseline4.png}
    \label{fig:ISLG2baseline4}
\end{figure}

\begin{figure}[H]
    \caption{Este gráfico mide directamente la calidad de la separación semántica en el espacio latente para palabras con la misma y diferente clase. El eje Y representa la distancia euclidiana promedio y el eje X son las épocas.}
    \centering
    \includegraphics[width=0.8\textwidth]{Images/Imagenes Cap 2/GraficasExperimentos/ISL/2gloss/gen1.PNG}
    \label{fig:ISLG2gen1}
\end{figure}

\begin{figure}[H]
    \caption{Este gráfico muestra la evolución de la \enquote{Pérdida Total} a lo largo de 100 épocas de entrenamiento. El eje Y representa el valor de la pérdida, una métrica que indica cuán bien el modelo está aprendiendo donde valores más bajos son mejores. El eje X representa las épocas, es decir, cada ciclo completo de entrenamiento sobre el conjunto de datos.}
    \centering
    \includegraphics[width=0.8\textwidth]{Images/Imagenes Cap 2/GraficasExperimentos/ISL/2gloss/loss1.PNG}
    \label{fig:ISLG2loss1}
\end{figure}

\begin{figure}[H]
    \caption{Esta gráfica ilustra la \enquote{Pérdida de Reconstrucción}, que mide qué tan bien el autoencoder del modelo puede reconstruir la entrada original después de haberla comprimido en un espacio latente. Al igual que en la gráfica anterior, el eje Y es el valor de la pérdida y el eje X son las épocas.}
    \centering
    \includegraphics[width=0.8\textwidth]{Images/Imagenes Cap 2/GraficasExperimentos/ISL/2gloss/loss2.PNG}
    \label{fig:ISLG2loss2}
\end{figure}

\begin{figure}[H]
    \caption{Este gráfico muestra la \enquote{Pérdida Triplet Semántica}, una métrica clave que evalúa si el modelo puede diferenciar entre distintos glosarios, en este caso, las señas \enquote{brother} y \enquote{cold}. El objetivo es que las representaciones de un mismo glosario estén más cerca entre sí que las de glosarios diferentes. Igualmente, el eje Y representa el valor de esta pérdida, mientras que el eje X indica las épocas de entrenamiento.}
    \centering
    \includegraphics[width=0.8\textwidth]{Images/Imagenes Cap 2/GraficasExperimentos/ISL/2gloss/loss3.PNG}
    \label{fig:ISLG2loss3}
\end{figure}

\begin{figure}[H]
    \caption{Esta visualización se enfoca en la sensibilidad temporal del modelo, puesto que, mide la diferencia entre la secuencia original de un video y su versión invertida. Esto quiere decir que el modelo debe aprender que una secuencia invertida es significativamente diferente de la original. El eje Y representa el valor de esta pérdida, mientras que el eje X indica las épocas de entrenamiento.}
    \centering
    \includegraphics[width=0.8\textwidth]{Images/Imagenes Cap 2/GraficasExperimentos/ISL/2gloss/loss4.PNG}
    \label{fig:ISLG2loss4}
\end{figure}

\begin{figure}[H]
    \caption{Similar a la gráfica anterior, esta también evalúa la sensibilidad temporal, pero en este caso, compara la secuencia original con una versión donde los fotogramas han sido desordenados aleatoriamente. Donde el objetivo es que el modelo reconozca que una secuencia permutada es muy diferente de la original. El eje Y representa el valor de esta pérdida, mientras que el eje X indica las épocas de entrenamiento.}
    \centering
    \includegraphics[width=0.8\textwidth]{Images/Imagenes Cap 2/GraficasExperimentos/ISL/2gloss/loss5.PNG}
    \label{fig:ISLG2loss5}
\end{figure}

\begin{figure}[H]
    \caption{Esta grafica muestra el espacio latente en la mejor epoca (100) utilizando pca, donde las señas \enquote{brother} estan con color azul y \enquote{cold} en naranja. Las diferentes formas de los puntos representan una variante diferente del video, siendo el circulo el original, la cruz el desplazado, el cuadrado el desordenado, y la equis la invertida.}
    \centering
    \includegraphics[width=0.8\textwidth]{Images/Imagenes Cap 2/GraficasExperimentos/ISL/2gloss/elpca0.PNG}
    \label{fig:ISLG2elpca0}
\end{figure}

\begin{figure}[H]
    \caption{Esta grafica muestra el espacio latente en la mejor epoca (100) utilizando umap, donde las señas \enquote{brother} estan con color azul y \enquote{cold} en naranja. Las diferentes formas de los puntos representan una variante diferente del video, siendo el circulo el original, la cruz el desplazado, el cuadrado el desordenado, y la equis la invertida.}
    \centering
    \includegraphics[width=0.8\textwidth]{Images/Imagenes Cap 2/GraficasExperimentos/ISL/2gloss/elumap0.PNG}
    \label{fig:ISLG2elumap0}
\end{figure}

\begin{figure}[H]
    \caption{Esta grafica muestra el espacio latente en la epoca 5 utilizando pca, donde las señas \enquote{brother} estan con color azul y \enquote{cold} en naranja. Las diferentes formas de los puntos representan una variante diferente del video, siendo el circulo el original, la cruz el desplazado, el cuadrado el desordenado, y la equis la invertida.}
    \centering
    \includegraphics[width=0.8\textwidth]{Images/Imagenes Cap 2/GraficasExperimentos/ISL/2gloss/elpca1.PNG}
    \label{fig:ISLG2elpca1}
\end{figure}

\begin{figure}[H]
    \caption{Esta grafica muestra el espacio latente en la epoca 5 utilizando umap, donde las señas \enquote{brother} estan con color azul y \enquote{cold} en naranja. Las diferentes formas de los puntos representan una variante diferente del video, siendo el circulo el original, la cruz el desplazado, el cuadrado el desordenado, y la equis la invertida.}
    \centering
    \includegraphics[width=0.8\textwidth]{Images/Imagenes Cap 2/GraficasExperimentos/ISL/2gloss/elumap1.PNG}
    \label{fig:ISLG2elumap1}
\end{figure}

\begin{figure}[H]
    \caption{Esta grafica muestra el espacio latente en la epoca 15 utilizando pca, donde las señas \enquote{brother} estan con color azul y \enquote{cold} en naranja. Las diferentes formas de los puntos representan una variante diferente del video, siendo el circulo el original, la cruz el desplazado, el cuadrado el desordenado, y la equis la invertida.}
    \centering
    \includegraphics[width=0.8\textwidth]{Images/Imagenes Cap 2/GraficasExperimentos/ISL/2gloss/elpca2.PNG}
    \label{fig:ISLG2elpca1}
\end{figure}

\begin{figure}[H]
    \caption{Esta grafica muestra el espacio latente en la epoca 15 utilizando umap, donde las señas \enquote{brother} estan con color azul y \enquote{cold} en naranja. Las diferentes formas de los puntos representan una variante diferente del video, siendo el circulo el original, la cruz el desplazado, el cuadrado el desordenado, y la equis la invertida.}
    \centering
    \includegraphics[width=0.8\textwidth]{Images/Imagenes Cap 2/GraficasExperimentos/ISL/2gloss/elumap2.PNG}
    \label{fig:ISLG2elumap2}
\end{figure}

\begin{figure}[H]
    \caption{Esta grafica muestra el espacio latente en la epoca 25 utilizando pca, donde las señas \enquote{brother} estan con color azul y \enquote{cold} en naranja. Las diferentes formas de los puntos representan una variante diferente del video, siendo el circulo el original, la cruz el desplazado, el cuadrado el desordenado, y la equis la invertida.}
    \centering
    \includegraphics[width=0.8\textwidth]{Images/Imagenes Cap 2/GraficasExperimentos/ISL/2gloss/elpca3.PNG}
    \label{fig:ISLG2elpca3}
\end{figure}

\begin{figure}[H]
    \caption{Esta grafica muestra el espacio latente en la epoca 25 utilizando umap, donde las señas \enquote{brother} estan con color azul y \enquote{cold} en naranja. Las diferentes formas de los puntos representan una variante diferente del video, siendo el circulo el original, la cruz el desplazado, el cuadrado el desordenado, y la equis la invertida.}
    \centering
    \includegraphics[width=0.8\textwidth]{Images/Imagenes Cap 2/GraficasExperimentos/ISL/2gloss/elumap3.PNG}
    \label{fig:ISLG2elumap3}
\end{figure}

\begin{figure}[H]
    \caption{Esta grafica muestra el espacio latente en la epoca 45 utilizando pca, donde las señas \enquote{brother} estan con color azul y \enquote{cold} en naranja. Las diferentes formas de los puntos representan una variante diferente del video, siendo el circulo el original, la cruz el desplazado, el cuadrado el desordenado, y la equis la invertida.}
    \centering
    \includegraphics[width=0.8\textwidth]{Images/Imagenes Cap 2/GraficasExperimentos/ISL/2gloss/elpca4.PNG}
    \label{fig:ISLG2elpca4}
\end{figure}

\begin{figure}[H]
    \caption{Esta grafica muestra el espacio latente en la epoca 45 utilizando umap, donde las señas \enquote{brother} estan con color azul y \enquote{cold} en naranja. Las diferentes formas de los puntos representan una variante diferente del video, siendo el circulo el original, la cruz el desplazado, el cuadrado el desordenado, y la equis la invertida.}
    \centering
    \includegraphics[width=0.8\textwidth]{Images/Imagenes Cap 2/GraficasExperimentos/ISL/2gloss/elumap4.PNG}
    \label{fig:ISLG2elumap4}
\end{figure}

\begin{figure}[H]
    \caption{Esta grafica muestra el espacio latente en la epoca 65 utilizando pca, donde las señas \enquote{brother} estan con color azul y \enquote{cold} en naranja. Las diferentes formas de los puntos representan una variante diferente del video, siendo el circulo el original, la cruz el desplazado, el cuadrado el desordenado, y la equis la invertida.}
    \centering
    \includegraphics[width=0.8\textwidth]{Images/Imagenes Cap 2/GraficasExperimentos/ISL/2gloss/elpca5.PNG}
    \label{fig:ISLG2elpca5}
\end{figure}

\begin{figure}[H]
    \caption{Esta grafica muestra el espacio latente en la epoca 65 utilizando umap, donde las señas \enquote{brother} estan con color azul y \enquote{cold} en naranja. Las diferentes formas de los puntos representan una variante diferente del video, siendo el circulo el original, la cruz el desplazado, el cuadrado el desordenado, y la equis la invertida.}
    \centering
    \includegraphics[width=0.8\textwidth]{Images/Imagenes Cap 2/GraficasExperimentos/ISL/2gloss/elumap5.PNG}
    \label{fig:ISLG2elumap5}
\end{figure}

\begin{figure}[H]
    \caption{Esta grafica muestra el espacio latente en la epoca 85 utilizando pca, donde las señas \enquote{brother} estan con color azul y \enquote{cold} en naranja. Las diferentes formas de los puntos representan una variante diferente del video, siendo el circulo el original, la cruz el desplazado, el cuadrado el desordenado, y la equis la invertida.}
    \centering
    \includegraphics[width=0.8\textwidth]{Images/Imagenes Cap 2/GraficasExperimentos/ISL/2gloss/elpca6.PNG}
    \label{fig:ISLG2elpca6}
\end{figure}

\begin{figure}[H]
    \caption{Esta grafica muestra el espacio latente en la epoca 85 utilizando umap, donde las señas \enquote{brother} estan con color azul y \enquote{cold} en naranja. Las diferentes formas de los puntos representan una variante diferente del video, siendo el circulo el original, la cruz el desplazado, el cuadrado el desordenado, y la equis la invertida.}
    \centering
    \includegraphics[width=0.8\textwidth]{Images/Imagenes Cap 2/GraficasExperimentos/ISL/2gloss/elumap6.PNG}
    \label{fig:ISLG2elumap6}
\end{figure}

\subsubsection{Con 3 etiquetas}

\begin{figure}[H]
    \caption{Esta gráfica compara el ratio Semántico del modelo comparado con un \enquote{baseline} o punto de referencia. Siendo este el modelo sin entrenar. El eje Y representa el ratio semántico, que mide la capacidad del modelo para diferenciar entre palabras diferentes, mientras que el eje X representa las épocas de entrenamiento.}
    \centering
    \includegraphics[width=0.8\textwidth]{Images/Imagenes Cap 2/GraficasExperimentos/ISL/3gloss/baseline1.png}
    \label{fig:ISLG3baseline1}
\end{figure}

\begin{figure}[H]
    \caption{Esta gráfica evalúa la capacidad del modelo para entender el orden temporal de las secuencias de video comparadas con sus respectivos \enquote{baselines} del modelo no entrenado. Muestra la distancia euclidiana promedio entre la secuencia original y sus versiones alteradas (shifted, inverted, permuted).}
    \centering
    \includegraphics[width=0.8\textwidth]{Images/Imagenes Cap 2/GraficasExperimentos/ISL/3gloss/baseline2.png}
    \label{fig:ISLG3baseline2}
\end{figure}

\begin{figure}[H]
    \caption{Esta gráfica compara el rendimiento del modelo principal con un modelo más simple en términos de la pérdida total de validación. El eje Y representa el valor de la pérdida, una métrica que indica cuán bien el modelo está aprendiendo, donde los valores más bajos son mejores, y el eje X representa las épocas.}
    \centering
    \includegraphics[width=0.8\textwidth]{Images/Imagenes Cap 2/GraficasExperimentos/ISL/3gloss/baseline3.png}
    \label{fig:ISLG3baseline3}
\end{figure}

\begin{figure}[H]
    \caption{Esta gráfica compara el ratio semántico del modelo comparado con dos \enquote{baselines} o puntos de referencia. Siendo estos el modelo sin entrenar y la representación de PCA. El eje Y representa el ratio semántico, que mide la capacidad del modelo para diferenciar entre palabras diferentes, mientras que el eje X representa las épocas de entrenamiento.}
    \centering
    \includegraphics[width=0.8\textwidth]{Images/Imagenes Cap 2/GraficasExperimentos/ISL/3gloss/baseline4.png}
    \label{fig:ISLG3baseline4}
\end{figure}

\begin{figure}[H]
    \caption{Este gráfico mide directamente la calidad de la separación semántica en el espacio latente para palabras con la misma y diferente clase. El eje Y representa la distancia euclidiana promedio y el eje X son las épocas.}
    \centering
    \includegraphics[width=0.8\textwidth]{Images/Imagenes Cap 2/GraficasExperimentos/ISL/3gloss/gen1.PNG}
    \label{fig:ISLG3gen1}
\end{figure}

\begin{figure}[H]
    \caption{Este gráfico muestra la evolución de la \enquote{Pérdida Total} a lo largo de 100 épocas de entrenamiento. El eje Y representa el valor de la pérdida, una métrica que indica cuán bien el modelo está aprendiendo donde valores más bajos son mejores. El eje X representa las épocas, es decir, cada ciclo completo de entrenamiento sobre el conjunto de datos.}
    \centering
    \includegraphics[width=0.8\textwidth]{Images/Imagenes Cap 2/GraficasExperimentos/ISL/3gloss/loss1.PNG}
    \label{fig:ISLG3loss1}
\end{figure}

\begin{figure}[H]
    \caption{Esta gráfica ilustra la \enquote{Pérdida de Reconstrucción}, que mide qué tan bien el autoencoder del modelo puede reconstruir la entrada original después de haberla comprimido en un espacio latente. Al igual que en la gráfica anterior, el eje Y es el valor de la pérdida y el eje X son las épocas.}
    \centering
    \includegraphics[width=0.8\textwidth]{Images/Imagenes Cap 2/GraficasExperimentos/ISL/3gloss/loss2.PNG}
    \label{fig:ISLG3loss2}
\end{figure}

\begin{figure}[H]
    \caption{Este gráfico muestra la \enquote{Pérdida Triplet Semántica}, una métrica clave que evalúa si el modelo puede diferenciar entre distintos glosarios, en este caso, las señas \enquote{brother}, \enquote{cold} y \enquote{man}. El objetivo es que las representaciones de un mismo glosario estén más cerca entre sí que las de glosarios diferentes. Igualmente, el eje Y representa el valor de esta pérdida, mientras que el eje X indica las épocas de entrenamiento.}
    \centering
    \includegraphics[width=0.8\textwidth]{Images/Imagenes Cap 2/GraficasExperimentos/ISL/3gloss/loss3.PNG}
    \label{fig:ISLG3loss3}
\end{figure}

\begin{figure}[H]
    \caption{Esta visualización se enfoca en la sensibilidad temporal del modelo, puesto que, mide la diferencia entre la secuencia original de un video y su versión invertida. Esto quiere decir que el modelo debe aprender que una secuencia invertida es significativamente diferente de la original. El eje Y representa el valor de esta pérdida, mientras que el eje X indica las épocas de entrenamiento.}
    \centering
    \includegraphics[width=0.8\textwidth]{Images/Imagenes Cap 2/GraficasExperimentos/ISL/3gloss/loss4.PNG}
    \label{fig:ISLG3loss4}
\end{figure}

\begin{figure}[H]
    \caption{Similar a la gráfica anterior, esta también evalúa la sensibilidad temporal, pero en este caso, compara la secuencia original con una versión donde los fotogramas han sido desordenados aleatoriamente. Donde el objetivo es que el modelo reconozca que una secuencia permutada es muy diferente de la original. El eje Y representa el valor de esta pérdida, mientras que el eje X indica las épocas de entrenamiento.}
    \centering
    \includegraphics[width=0.8\textwidth]{Images/Imagenes Cap 2/GraficasExperimentos/ISL/3gloss/loss5.PNG}
    \label{fig:ISLG3loss5}
\end{figure}

\begin{figure}[H]
    \caption{Esta grafica muestra el espacio latente en la mejor epoca (100) utilizando pca, donde las señas \enquote{brother} estan con color azul, \enquote{cold} en naranja y \enquote{man} en verde. Las diferentes formas de los puntos representan una variante diferente del video, siendo el circulo el original, la cruz el desplazado, el cuadrado el desordenado, y la equis la invertida.}
    \centering
    \includegraphics[width=0.8\textwidth]{Images/Imagenes Cap 2/GraficasExperimentos/ISL/3gloss/elpca0.PNG}
    \label{fig:ISLG3elpca0}
\end{figure}

\begin{figure}[H]
    \caption{Esta grafica muestra el espacio latente en la mejor epoca (100) utilizando umap, donde las señas \enquote{brother} estan con color azul, \enquote{cold} en naranja y \enquote{man} en verde. Las diferentes formas de los puntos representan una variante diferente del video, siendo el circulo el original, la cruz el desplazado, el cuadrado el desordenado, y la equis la invertida.}
    \centering
    \includegraphics[width=0.8\textwidth]{Images/Imagenes Cap 2/GraficasExperimentos/ISL/3gloss/elumap0.PNG}
    \label{fig:ISLG3elumap0}
\end{figure}

\begin{figure}[H]
    \caption{Esta grafica muestra el espacio latente en la epoca 5 utilizando pca, donde las señas \enquote{brother} estan con color azul, \enquote{cold} en naranja y \enquote{man} en verde. Las diferentes formas de los puntos representan una variante diferente del video, siendo el circulo el original, la cruz el desplazado, el cuadrado el desordenado, y la equis la invertida.}
    \centering
    \includegraphics[width=0.8\textwidth]{Images/Imagenes Cap 2/GraficasExperimentos/ISL/3gloss/elpca1.PNG}
    \label{fig:ISLG3elpca1}
\end{figure}

\begin{figure}[H]
    \caption{Esta grafica muestra el espacio latente en la epoca 5 utilizando umap, donde las señas \enquote{brother} estan con color azul, \enquote{cold} en naranja y \enquote{man} en verde. Las diferentes formas de los puntos representan una variante diferente del video, siendo el circulo el original, la cruz el desplazado, el cuadrado el desordenado, y la equis la invertida.}
    \centering
    \includegraphics[width=0.8\textwidth]{Images/Imagenes Cap 2/GraficasExperimentos/ISL/3gloss/elumap1.PNG}
    \label{fig:ISLG3elumap1}
\end{figure}

\begin{figure}[H]
    \caption{Esta grafica muestra el espacio latente en la epoca 15 utilizando pca, donde las señas \enquote{brother} estan con color azul, \enquote{cold} en naranja y \enquote{man} en verde. Las diferentes formas de los puntos representan una variante diferente del video, siendo el circulo el original, la cruz el desplazado, el cuadrado el desordenado, y la equis la invertida.}
    \centering
    \includegraphics[width=0.8\textwidth]{Images/Imagenes Cap 2/GraficasExperimentos/ISL/3gloss/elpca2.PNG}
    \label{fig:ISLG3elpca1}
\end{figure}

\begin{figure}[H]
    \caption{Esta grafica muestra el espacio latente en la epoca 15 utilizando umap, donde las señas \enquote{brother} estan con color azul, \enquote{cold} en naranja y \enquote{man} en verde. Las diferentes formas de los puntos representan una variante diferente del video, siendo el circulo el original, la cruz el desplazado, el cuadrado el desordenado, y la equis la invertida.}
    \centering
    \includegraphics[width=0.8\textwidth]{Images/Imagenes Cap 2/GraficasExperimentos/ISL/3gloss/elumap2.PNG}
    \label{fig:ISLG3elumap2}
\end{figure}

\begin{figure}[H]
    \caption{Esta grafica muestra el espacio latente en la epoca 25 utilizando pca, donde las señas \enquote{brother} estan con color azul, \enquote{cold} en naranja y \enquote{man} en verde. Las diferentes formas de los puntos representan una variante diferente del video, siendo el circulo el original, la cruz el desplazado, el cuadrado el desordenado, y la equis la invertida.}
    \centering
    \includegraphics[width=0.8\textwidth]{Images/Imagenes Cap 2/GraficasExperimentos/ISL/3gloss/elpca3.PNG}
    \label{fig:ISLG3elpca3}
\end{figure}

\begin{figure}[H]
    \caption{Esta grafica muestra el espacio latente en la epoca 25 utilizando umap, donde las señas \enquote{brother} estan con color azul, \enquote{cold} en naranja y \enquote{man} en verde. Las diferentes formas de los puntos representan una variante diferente del video, siendo el circulo el original, la cruz el desplazado, el cuadrado el desordenado, y la equis la invertida.}
    \centering
    \includegraphics[width=0.8\textwidth]{Images/Imagenes Cap 2/GraficasExperimentos/ISL/3gloss/elumap3.PNG}
    \label{fig:ISLG3elumap3}
\end{figure}

\begin{figure}[H]
    \caption{Esta grafica muestra el espacio latente en la epoca 45 utilizando pca, donde las señas \enquote{brother} estan con color azul, \enquote{cold} en naranja y \enquote{man} en verde. Las diferentes formas de los puntos representan una variante diferente del video, siendo el circulo el original, la cruz el desplazado, el cuadrado el desordenado, y la equis la invertida.}
    \centering
    \includegraphics[width=0.8\textwidth]{Images/Imagenes Cap 2/GraficasExperimentos/ISL/3gloss/elpca4.PNG}
    \label{fig:ISLG3elpca4}
\end{figure}

\begin{figure}[H]
    \caption{Esta grafica muestra el espacio latente en la epoca 45 utilizando umap, donde las señas \enquote{brother} estan con color azul, \enquote{cold} en naranja y \enquote{man} en verde. Las diferentes formas de los puntos representan una variante diferente del video, siendo el circulo el original, la cruz el desplazado, el cuadrado el desordenado, y la equis la invertida.}
    \centering
    \includegraphics[width=0.8\textwidth]{Images/Imagenes Cap 2/GraficasExperimentos/ISL/3gloss/elumap4.PNG}
    \label{fig:ISLG3elumap4}
\end{figure}

\begin{figure}[H]
    \caption{Esta grafica muestra el espacio latente en la epoca 65 utilizando pca, donde las señas \enquote{brother} estan con color azul, \enquote{cold} en naranja y \enquote{man} en verde. Las diferentes formas de los puntos representan una variante diferente del video, siendo el circulo el original, la cruz el desplazado, el cuadrado el desordenado, y la equis la invertida.}
    \centering
    \includegraphics[width=0.8\textwidth]{Images/Imagenes Cap 2/GraficasExperimentos/ISL/3gloss/elpca5.PNG}
    \label{fig:ISLG3elpca5}
\end{figure}

\begin{figure}[H]
    \caption{Esta grafica muestra el espacio latente en la epoca 65 utilizando umap, donde las señas \enquote{brother} estan con color azul, \enquote{cold} en naranja y \enquote{man} en verde. Las diferentes formas de los puntos representan una variante diferente del video, siendo el circulo el original, la cruz el desplazado, el cuadrado el desordenado, y la equis la invertida.}
    \centering
    \includegraphics[width=0.8\textwidth]{Images/Imagenes Cap 2/GraficasExperimentos/ISL/3gloss/elumap5.PNG}
    \label{fig:ISLG3elumap5}
\end{figure}

\begin{figure}[H]
    \caption{Esta grafica muestra el espacio latente en la epoca 85 utilizando pca, donde las señas \enquote{brother} estan con color azul, \enquote{cold} en naranja y \enquote{man} en verde. Las diferentes formas de los puntos representan una variante diferente del video, siendo el circulo el original, la cruz el desplazado, el cuadrado el desordenado, y la equis la invertida.}
    \centering
    \includegraphics[width=0.8\textwidth]{Images/Imagenes Cap 2/GraficasExperimentos/ISL/3gloss/elpca6.PNG}
    \label{fig:ISLG3elpca6}
\end{figure}

\begin{figure}[H]
    \caption{Esta grafica muestra el espacio latente en la epoca 85 utilizando umap, donde las señas \enquote{brother} estan con color azul, \enquote{cold} en naranja y \enquote{man} en verde. Las diferentes formas de los puntos representan una variante diferente del video, siendo el circulo el original, la cruz el desplazado, el cuadrado el desordenado, y la equis la invertida.}
    \centering
    \includegraphics[width=0.8\textwidth]{Images/Imagenes Cap 2/GraficasExperimentos/ISL/3gloss/elumap6.PNG}
    \label{fig:ISLG3elumap6}
\end{figure}

\subsection{Con el dataset de SLOVO}

\subsubsection{Con 2 etiquetas}

\begin{figure}[H]
    \caption{Esta gráfica compara el ratio Semántico del modelo comparado con un \enquote{baseline} o punto de referencia. Siendo este el modelo sin entrenar. El eje Y representa el ratio semántico, que mide la capacidad del modelo para diferenciar entre palabras diferentes, mientras que el eje X representa las épocas de entrenamiento.}
    \centering
    \includegraphics[width=0.8\textwidth]{Images/Imagenes Cap 2/GraficasExperimentos/SLOVO/2gloss/baseline1.png}
    \label{fig:SLOVOG2baseline1}
\end{figure}

\begin{figure}[H]
    \caption{Esta gráfica evalúa la capacidad del modelo para entender el orden temporal de las secuencias de video comparadas con sus respectivos \enquote{baselines} del modelo no entrenado. Muestra la distancia euclidiana promedio entre la secuencia original y sus versiones alteradas (shifted, inverted, permuted).}
    \centering
    \includegraphics[width=0.8\textwidth]{Images/Imagenes Cap 2/GraficasExperimentos/SLOVO/2gloss/baseline2.png}
    \label{fig:SLOVOG2baseline2}
\end{figure}

\begin{figure}[H]
    \caption{Esta gráfica compara el rendimiento del modelo principal con un modelo más simple en términos de la pérdida total de validación. El eje Y representa el valor de la pérdida, una métrica que indica cuán bien el modelo está aprendiendo, donde los valores más bajos son mejores, y el eje X representa las épocas.}
    \centering
    \includegraphics[width=0.8\textwidth]{Images/Imagenes Cap 2/GraficasExperimentos/SLOVO/2gloss/baseline3.png}
    \label{fig:SLOVOG2baseline3}
\end{figure}

\begin{figure}[H]
    \caption{Esta gráfica compara el ratio semántico del modelo comparado con dos \enquote{baselines} o puntos de referencia. Siendo estos el modelo sin entrenar y la representación de PCA. El eje Y representa el ratio semántico, que mide la capacidad del modelo para diferenciar entre palabras diferentes, mientras que el eje X representa las épocas de entrenamiento.}
    \centering
    \includegraphics[width=0.8\textwidth]{Images/Imagenes Cap 2/GraficasExperimentos/SLOVO/2gloss/baseline4.png}
    \label{fig:SLOVOG2baseline4}
\end{figure}

\begin{figure}[H]
    \caption{Este gráfico mide directamente la calidad de la separación semántica en el espacio latente para palabras con la misma y diferente clase. El eje Y representa la distancia euclidiana promedio y el eje X son las épocas.}
    \centering
    \includegraphics[width=0.8\textwidth]{Images/Imagenes Cap 2/GraficasExperimentos/SLOVO/2gloss/gen1.PNG}
    \label{fig:SLOVOG2gen1}
\end{figure}

\begin{figure}[H]
    \caption{Este gráfico muestra la evolución de la \enquote{Pérdida Total} a lo largo de 100 épocas de entrenamiento. El eje Y representa el valor de la pérdida, una métrica que indica cuán bien el modelo está aprendiendo donde valores más bajos son mejores. El eje X representa las épocas, es decir, cada ciclo completo de entrenamiento sobre el conjunto de datos.}
    \centering
    \includegraphics[width=0.8\textwidth]{Images/Imagenes Cap 2/GraficasExperimentos/SLOVO/2gloss/loss1.PNG}
    \label{fig:SLOVOG2loss1}
\end{figure}

\begin{figure}[H]
    \caption{Esta gráfica ilustra la \enquote{Pérdida de Reconstrucción}, que mide qué tan bien el autoencoder del modelo puede reconstruir la entrada original después de haberla comprimido en un espacio latente. Al igual que en la gráfica anterior, el eje Y es el valor de la pérdida y el eje X son las épocas.}
    \centering
    \includegraphics[width=0.8\textwidth]{Images/Imagenes Cap 2/GraficasExperimentos/SLOVO/2gloss/loss2.PNG}
    \label{fig:SLOVOG2loss2}
\end{figure}

\begin{figure}[H]
    \caption{Este gráfico muestra la \enquote{Pérdida Triplet Semántica}, una métrica clave que evalúa si el modelo puede diferenciar entre distintos glosarios, en este caso, las señas \enquote{brother} y \enquote{cold}. El objetivo es que las representaciones de un mismo glosario estén más cerca entre sí que las de glosarios diferentes. Igualmente, el eje Y representa el valor de esta pérdida, mientras que el eje X indica las épocas de entrenamiento.}
    \centering
    \includegraphics[width=0.8\textwidth]{Images/Imagenes Cap 2/GraficasExperimentos/SLOVO/2gloss/loss3.PNG}
    \label{fig:SLOVOG2loss3}
\end{figure}

\begin{figure}[H]
    \caption{Esta visualización se enfoca en la sensibilidad temporal del modelo, puesto que, mide la diferencia entre la secuencia original de un video y su versión invertida. Esto quiere decir que el modelo debe aprender que una secuencia invertida es significativamente diferente de la original. El eje Y representa el valor de esta pérdida, mientras que el eje X indica las épocas de entrenamiento.}
    \centering
    \includegraphics[width=0.8\textwidth]{Images/Imagenes Cap 2/GraficasExperimentos/SLOVO/2gloss/loss4.PNG}
    \label{fig:SLOVOG2loss4}
\end{figure}

\begin{figure}[H]
    \caption{Similar a la gráfica anterior, esta también evalúa la sensibilidad temporal, pero en este caso, compara la secuencia original con una versión donde los fotogramas han sido desordenados aleatoriamente. Donde el objetivo es que el modelo reconozca que una secuencia permutada es muy diferente de la original. El eje Y representa el valor de esta pérdida, mientras que el eje X indica las épocas de entrenamiento.}
    \centering
    \includegraphics[width=0.8\textwidth]{Images/Imagenes Cap 2/GraficasExperimentos/SLOVO/2gloss/loss5.PNG}
    \label{fig:SLOVOG2loss5}
\end{figure}

\begin{figure}[H]
    \caption{Esta grafica muestra el espacio latente en la mejor epoca (100) utilizando pca, donde las señas \enquote{brother} estan con color azul y \enquote{cold} en naranja. Las diferentes formas de los puntos representan una variante diferente del video, siendo el circulo el original, la cruz el desplazado, el cuadrado el desordenado, y la equis la invertida.}
    \centering
    \includegraphics[width=0.8\textwidth]{Images/Imagenes Cap 2/GraficasExperimentos/SLOVO/2gloss/elpca0.PNG}
    \label{fig:SLOVOG2elpca0}
\end{figure}

\begin{figure}[H]
    \caption{Esta grafica muestra el espacio latente en la mejor epoca (100) utilizando umap, donde las señas \enquote{brother} estan con color azul y \enquote{cold} en naranja. Las diferentes formas de los puntos representan una variante diferente del video, siendo el circulo el original, la cruz el desplazado, el cuadrado el desordenado, y la equis la invertida.}
    \centering
    \includegraphics[width=0.8\textwidth]{Images/Imagenes Cap 2/GraficasExperimentos/SLOVO/2gloss/elumap0.PNG}
    \label{fig:SLOVOG2elumap0}
\end{figure}

\begin{figure}[H]
    \caption{Esta grafica muestra el espacio latente en la epoca 5 utilizando pca, donde las señas \enquote{brother} estan con color azul y \enquote{cold} en naranja. Las diferentes formas de los puntos representan una variante diferente del video, siendo el circulo el original, la cruz el desplazado, el cuadrado el desordenado, y la equis la invertida.}
    \centering
    \includegraphics[width=0.8\textwidth]{Images/Imagenes Cap 2/GraficasExperimentos/SLOVO/2gloss/elpca1.PNG}
    \label{fig:SLOVOG2elpca1}
\end{figure}

\begin{figure}[H]
    \caption{Esta grafica muestra el espacio latente en la epoca 5 utilizando umap, donde las señas \enquote{brother} estan con color azul y \enquote{cold} en naranja. Las diferentes formas de los puntos representan una variante diferente del video, siendo el circulo el original, la cruz el desplazado, el cuadrado el desordenado, y la equis la invertida.}
    \centering
    \includegraphics[width=0.8\textwidth]{Images/Imagenes Cap 2/GraficasExperimentos/SLOVO/2gloss/elumap1.PNG}
    \label{fig:SLOVOG2elumap1}
\end{figure}

\begin{figure}[H]
    \caption{Esta grafica muestra el espacio latente en la epoca 15 utilizando pca, donde las señas \enquote{brother} estan con color azul y \enquote{cold} en naranja. Las diferentes formas de los puntos representan una variante diferente del video, siendo el circulo el original, la cruz el desplazado, el cuadrado el desordenado, y la equis la invertida.}
    \centering
    \includegraphics[width=0.8\textwidth]{Images/Imagenes Cap 2/GraficasExperimentos/SLOVO/2gloss/elpca2.PNG}
    \label{fig:SLOVOG2elpca1}
\end{figure}

\begin{figure}[H]
    \caption{Esta grafica muestra el espacio latente en la epoca 15 utilizando umap, donde las señas \enquote{brother} estan con color azul y \enquote{cold} en naranja. Las diferentes formas de los puntos representan una variante diferente del video, siendo el circulo el original, la cruz el desplazado, el cuadrado el desordenado, y la equis la invertida.}
    \centering
    \includegraphics[width=0.8\textwidth]{Images/Imagenes Cap 2/GraficasExperimentos/SLOVO/2gloss/elumap2.PNG}
    \label{fig:SLOVOG2elumap2}
\end{figure}

\begin{figure}[H]
    \caption{Esta grafica muestra el espacio latente en la epoca 25 utilizando pca, donde las señas \enquote{brother} estan con color azul y \enquote{cold} en naranja. Las diferentes formas de los puntos representan una variante diferente del video, siendo el circulo el original, la cruz el desplazado, el cuadrado el desordenado, y la equis la invertida.}
    \centering
    \includegraphics[width=0.8\textwidth]{Images/Imagenes Cap 2/GraficasExperimentos/SLOVO/2gloss/elpca3.PNG}
    \label{fig:SLOVOG2elpca3}
\end{figure}

\begin{figure}[H]
    \caption{Esta grafica muestra el espacio latente en la epoca 25 utilizando umap, donde las señas \enquote{brother} estan con color azul y \enquote{cold} en naranja. Las diferentes formas de los puntos representan una variante diferente del video, siendo el circulo el original, la cruz el desplazado, el cuadrado el desordenado, y la equis la invertida.}
    \centering
    \includegraphics[width=0.8\textwidth]{Images/Imagenes Cap 2/GraficasExperimentos/SLOVO/2gloss/elumap3.PNG}
    \label{fig:SLOVOG2elumap3}
\end{figure}

\begin{figure}[H]
    \caption{Esta grafica muestra el espacio latente en la epoca 45 utilizando pca, donde las señas \enquote{brother} estan con color azul y \enquote{cold} en naranja. Las diferentes formas de los puntos representan una variante diferente del video, siendo el circulo el original, la cruz el desplazado, el cuadrado el desordenado, y la equis la invertida.}
    \centering
    \includegraphics[width=0.8\textwidth]{Images/Imagenes Cap 2/GraficasExperimentos/SLOVO/2gloss/elpca4.PNG}
    \label{fig:SLOVOG2elpca4}
\end{figure}

\begin{figure}[H]
    \caption{Esta grafica muestra el espacio latente en la epoca 45 utilizando umap, donde las señas \enquote{brother} estan con color azul y \enquote{cold} en naranja. Las diferentes formas de los puntos representan una variante diferente del video, siendo el circulo el original, la cruz el desplazado, el cuadrado el desordenado, y la equis la invertida.}
    \centering
    \includegraphics[width=0.8\textwidth]{Images/Imagenes Cap 2/GraficasExperimentos/SLOVO/2gloss/elumap4.PNG}
    \label{fig:SLOVOG2elumap4}
\end{figure}

\begin{figure}[H]
    \caption{Esta grafica muestra el espacio latente en la epoca 65 utilizando pca, donde las señas \enquote{brother} estan con color azul y \enquote{cold} en naranja. Las diferentes formas de los puntos representan una variante diferente del video, siendo el circulo el original, la cruz el desplazado, el cuadrado el desordenado, y la equis la invertida.}
    \centering
    \includegraphics[width=0.8\textwidth]{Images/Imagenes Cap 2/GraficasExperimentos/SLOVO/2gloss/elpca5.PNG}
    \label{fig:SLOVOG2elpca5}
\end{figure}

\begin{figure}[H]
    \caption{Esta grafica muestra el espacio latente en la epoca 65 utilizando umap, donde las señas \enquote{brother} estan con color azul y \enquote{cold} en naranja. Las diferentes formas de los puntos representan una variante diferente del video, siendo el circulo el original, la cruz el desplazado, el cuadrado el desordenado, y la equis la invertida.}
    \centering
    \includegraphics[width=0.8\textwidth]{Images/Imagenes Cap 2/GraficasExperimentos/SLOVO/2gloss/elumap5.PNG}
    \label{fig:SLOVOG2elumap5}
\end{figure}

\begin{figure}[H]
    \caption{Esta grafica muestra el espacio latente en la epoca 85 utilizando pca, donde las señas \enquote{brother} estan con color azul y \enquote{cold} en naranja. Las diferentes formas de los puntos representan una variante diferente del video, siendo el circulo el original, la cruz el desplazado, el cuadrado el desordenado, y la equis la invertida.}
    \centering
    \includegraphics[width=0.8\textwidth]{Images/Imagenes Cap 2/GraficasExperimentos/SLOVO/2gloss/elpca6.PNG}
    \label{fig:SLOVOG2elpca6}
\end{figure}

\begin{figure}[H]
    \caption{Esta grafica muestra el espacio latente en la epoca 85 utilizando umap, donde las señas \enquote{brother} estan con color azul y \enquote{cold} en naranja. Las diferentes formas de los puntos representan una variante diferente del video, siendo el circulo el original, la cruz el desplazado, el cuadrado el desordenado, y la equis la invertida.}
    \centering
    \includegraphics[width=0.8\textwidth]{Images/Imagenes Cap 2/GraficasExperimentos/SLOVO/2gloss/elumap6.PNG}
    \label{fig:SLOVOG2elumap6}
\end{figure}

\subsubsection{Con 3 etiquetas}

\begin{figure}[H]
    \caption{Esta gráfica compara el ratio Semántico del modelo comparado con un \enquote{baseline} o punto de referencia. Siendo este el modelo sin entrenar. El eje Y representa el ratio semántico, que mide la capacidad del modelo para diferenciar entre palabras diferentes, mientras que el eje X representa las épocas de entrenamiento.}
    \centering
    \includegraphics[width=0.8\textwidth]{Images/Imagenes Cap 2/GraficasExperimentos/SLOVO/3gloss/baseline1.png}
    \label{fig:SLOVOG3baseline1}
\end{figure}

\begin{figure}[H]
    \caption{Esta gráfica evalúa la capacidad del modelo para entender el orden temporal de las secuencias de video comparadas con sus respectivos \enquote{baselines} del modelo no entrenado. Muestra la distancia euclidiana promedio entre la secuencia original y sus versiones alteradas (shifted, inverted, permuted).}
    \centering
    \includegraphics[width=0.8\textwidth]{Images/Imagenes Cap 2/GraficasExperimentos/SLOVO/3gloss/baseline2.png}
    \label{fig:SLOVOG3baseline2}
\end{figure}

\begin{figure}[H]
    \caption{Esta gráfica compara el rendimiento del modelo principal con un modelo más simple en términos de la pérdida total de validación. El eje Y representa el valor de la pérdida, una métrica que indica cuán bien el modelo está aprendiendo, donde los valores más bajos son mejores, y el eje X representa las épocas.}
    \centering
    \includegraphics[width=0.8\textwidth]{Images/Imagenes Cap 2/GraficasExperimentos/SLOVO/3gloss/baseline3.png}
    \label{fig:SLOVOG3baseline3}
\end{figure}

\begin{figure}[H]
    \caption{Esta gráfica compara el ratio semántico del modelo comparado con dos \enquote{baselines} o puntos de referencia. Siendo estos el modelo sin entrenar y la representación de PCA. El eje Y representa el ratio semántico, que mide la capacidad del modelo para diferenciar entre palabras diferentes, mientras que el eje X representa las épocas de entrenamiento.}
    \centering
    \includegraphics[width=0.8\textwidth]{Images/Imagenes Cap 2/GraficasExperimentos/SLOVO/3gloss/baseline4.png}
    \label{fig:SLOVOG3baseline4}
\end{figure}

\begin{figure}[H]
    \caption{Este gráfico mide directamente la calidad de la separación semántica en el espacio latente para palabras con la misma y diferente clase. El eje Y representa la distancia euclidiana promedio y el eje X son las épocas.}
    \centering
    \includegraphics[width=0.8\textwidth]{Images/Imagenes Cap 2/GraficasExperimentos/SLOVO/3gloss/gen1.PNG}
    \label{fig:SLOVOG3gen1}
\end{figure}

\begin{figure}[H]
    \caption{Este gráfico muestra la evolución de la \enquote{Pérdida Total} a lo largo de 100 épocas de entrenamiento. El eje Y representa el valor de la pérdida, una métrica que indica cuán bien el modelo está aprendiendo donde valores más bajos son mejores. El eje X representa las épocas, es decir, cada ciclo completo de entrenamiento sobre el conjunto de datos.}
    \centering
    \includegraphics[width=0.8\textwidth]{Images/Imagenes Cap 2/GraficasExperimentos/SLOVO/3gloss/loss1.PNG}
    \label{fig:SLOVOG3loss1}
\end{figure}

\begin{figure}[H]
    \caption{Esta gráfica ilustra la \enquote{Pérdida de Reconstrucción}, que mide qué tan bien el autoencoder del modelo puede reconstruir la entrada original después de haberla comprimido en un espacio latente. Al igual que en la gráfica anterior, el eje Y es el valor de la pérdida y el eje X son las épocas.}
    \centering
    \includegraphics[width=0.8\textwidth]{Images/Imagenes Cap 2/GraficasExperimentos/SLOVO/3gloss/loss2.PNG}
    \label{fig:SLOVOG3loss2}
\end{figure}

\begin{figure}[H]
    \caption{Este gráfico muestra la \enquote{Pérdida Triplet Semántica}, una métrica clave que evalúa si el modelo puede diferenciar entre distintos glosarios, en este caso, las señas \enquote{brother}, \enquote{cold} y \enquote{man}. El objetivo es que las representaciones de un mismo glosario estén más cerca entre sí que las de glosarios diferentes. Igualmente, el eje Y representa el valor de esta pérdida, mientras que el eje X indica las épocas de entrenamiento.}
    \centering
    \includegraphics[width=0.8\textwidth]{Images/Imagenes Cap 2/GraficasExperimentos/SLOVO/3gloss/loss3.PNG}
    \label{fig:SLOVOG3loss3}
\end{figure}

\begin{figure}[H]
    \caption{Esta visualización se enfoca en la sensibilidad temporal del modelo, puesto que, mide la diferencia entre la secuencia original de un video y su versión invertida. Esto quiere decir que el modelo debe aprender que una secuencia invertida es significativamente diferente de la original. El eje Y representa el valor de esta pérdida, mientras que el eje X indica las épocas de entrenamiento.}
    \centering
    \includegraphics[width=0.8\textwidth]{Images/Imagenes Cap 2/GraficasExperimentos/SLOVO/3gloss/loss4.PNG}
    \label{fig:SLOVOG3loss4}
\end{figure}

\begin{figure}[H]
    \caption{Similar a la gráfica anterior, esta también evalúa la sensibilidad temporal, pero en este caso, compara la secuencia original con una versión donde los fotogramas han sido desordenados aleatoriamente. Donde el objetivo es que el modelo reconozca que una secuencia permutada es muy diferente de la original. El eje Y representa el valor de esta pérdida, mientras que el eje X indica las épocas de entrenamiento.}
    \centering
    \includegraphics[width=0.8\textwidth]{Images/Imagenes Cap 2/GraficasExperimentos/SLOVO/3gloss/loss5.PNG}
    \label{fig:SLOVOG3loss5}
\end{figure}

\begin{figure}[H]
    \caption{Esta grafica muestra el espacio latente en la mejor epoca (100) utilizando pca, donde las señas \enquote{brother} estan con color azul, \enquote{cold} en naranja y \enquote{man} en verde. Las diferentes formas de los puntos representan una variante diferente del video, siendo el circulo el original, la cruz el desplazado, el cuadrado el desordenado, y la equis la invertida.}
    \centering
    \includegraphics[width=0.8\textwidth]{Images/Imagenes Cap 2/GraficasExperimentos/SLOVO/3gloss/elpca0.PNG}
    \label{fig:SLOVOG3elpca0}
\end{figure}

\begin{figure}[H]
    \caption{Esta grafica muestra el espacio latente en la mejor epoca (100) utilizando umap, donde las señas \enquote{brother} estan con color azul, \enquote{cold} en naranja y \enquote{man} en verde. Las diferentes formas de los puntos representan una variante diferente del video, siendo el circulo el original, la cruz el desplazado, el cuadrado el desordenado, y la equis la invertida.}
    \centering
    \includegraphics[width=0.8\textwidth]{Images/Imagenes Cap 2/GraficasExperimentos/SLOVO/3gloss/elumap0.PNG}
    \label{fig:SLOVOG3elumap0}
\end{figure}

\begin{figure}[H]
    \caption{Esta grafica muestra el espacio latente en la epoca 5 utilizando pca, donde las señas \enquote{brother} estan con color azul, \enquote{cold} en naranja y \enquote{man} en verde. Las diferentes formas de los puntos representan una variante diferente del video, siendo el circulo el original, la cruz el desplazado, el cuadrado el desordenado, y la equis la invertida.}
    \centering
    \includegraphics[width=0.8\textwidth]{Images/Imagenes Cap 2/GraficasExperimentos/SLOVO/3gloss/elpca1.PNG}
    \label{fig:SLOVOG3elpca1}
\end{figure}

\begin{figure}[H]
    \caption{Esta grafica muestra el espacio latente en la epoca 5 utilizando umap, donde las señas \enquote{brother} estan con color azul, \enquote{cold} en naranja y \enquote{man} en verde. Las diferentes formas de los puntos representan una variante diferente del video, siendo el circulo el original, la cruz el desplazado, el cuadrado el desordenado, y la equis la invertida.}
    \centering
    \includegraphics[width=0.8\textwidth]{Images/Imagenes Cap 2/GraficasExperimentos/SLOVO/3gloss/elumap1.PNG}
    \label{fig:SLOVOG3elumap1}
\end{figure}

\begin{figure}[H]
    \caption{Esta grafica muestra el espacio latente en la epoca 15 utilizando pca, donde las señas \enquote{brother} estan con color azul, \enquote{cold} en naranja y \enquote{man} en verde. Las diferentes formas de los puntos representan una variante diferente del video, siendo el circulo el original, la cruz el desplazado, el cuadrado el desordenado, y la equis la invertida.}
    \centering
    \includegraphics[width=0.8\textwidth]{Images/Imagenes Cap 2/GraficasExperimentos/SLOVO/3gloss/elpca2.PNG}
    \label{fig:SLOVOG3elpca1}
\end{figure}

\begin{figure}[H]
    \caption{Esta grafica muestra el espacio latente en la epoca 15 utilizando umap, donde las señas \enquote{brother} estan con color azul, \enquote{cold} en naranja y \enquote{man} en verde. Las diferentes formas de los puntos representan una variante diferente del video, siendo el circulo el original, la cruz el desplazado, el cuadrado el desordenado, y la equis la invertida.}
    \centering
    \includegraphics[width=0.8\textwidth]{Images/Imagenes Cap 2/GraficasExperimentos/SLOVO/3gloss/elumap2.PNG}
    \label{fig:SLOVOG3elumap2}
\end{figure}

\begin{figure}[H]
    \caption{Esta grafica muestra el espacio latente en la epoca 25 utilizando pca, donde las señas \enquote{brother} estan con color azul, \enquote{cold} en naranja y \enquote{man} en verde. Las diferentes formas de los puntos representan una variante diferente del video, siendo el circulo el original, la cruz el desplazado, el cuadrado el desordenado, y la equis la invertida.}
    \centering
    \includegraphics[width=0.8\textwidth]{Images/Imagenes Cap 2/GraficasExperimentos/SLOVO/3gloss/elpca3.PNG}
    \label{fig:SLOVOG3elpca3}
\end{figure}

\begin{figure}[H]
    \caption{Esta grafica muestra el espacio latente en la epoca 25 utilizando umap, donde las señas \enquote{brother} estan con color azul, \enquote{cold} en naranja y \enquote{man} en verde. Las diferentes formas de los puntos representan una variante diferente del video, siendo el circulo el original, la cruz el desplazado, el cuadrado el desordenado, y la equis la invertida.}
    \centering
    \includegraphics[width=0.8\textwidth]{Images/Imagenes Cap 2/GraficasExperimentos/SLOVO/3gloss/elumap3.PNG}
    \label{fig:SLOVOG3elumap3}
\end{figure}

\begin{figure}[H]
    \caption{Esta grafica muestra el espacio latente en la epoca 45 utilizando pca, donde las señas \enquote{brother} estan con color azul, \enquote{cold} en naranja y \enquote{man} en verde. Las diferentes formas de los puntos representan una variante diferente del video, siendo el circulo el original, la cruz el desplazado, el cuadrado el desordenado, y la equis la invertida.}
    \centering
    \includegraphics[width=0.8\textwidth]{Images/Imagenes Cap 2/GraficasExperimentos/SLOVO/3gloss/elpca4.PNG}
    \label{fig:SLOVOG3elpca4}
\end{figure}

\begin{figure}[H]
    \caption{Esta grafica muestra el espacio latente en la epoca 45 utilizando umap, donde las señas \enquote{brother} estan con color azul, \enquote{cold} en naranja y \enquote{man} en verde. Las diferentes formas de los puntos representan una variante diferente del video, siendo el circulo el original, la cruz el desplazado, el cuadrado el desordenado, y la equis la invertida.}
    \centering
    \includegraphics[width=0.8\textwidth]{Images/Imagenes Cap 2/GraficasExperimentos/SLOVO/3gloss/elumap4.PNG}
    \label{fig:SLOVOG3elumap4}
\end{figure}

\begin{figure}[H]
    \caption{Esta grafica muestra el espacio latente en la epoca 65 utilizando pca, donde las señas \enquote{brother} estan con color azul, \enquote{cold} en naranja y \enquote{man} en verde. Las diferentes formas de los puntos representan una variante diferente del video, siendo el circulo el original, la cruz el desplazado, el cuadrado el desordenado, y la equis la invertida.}
    \centering
    \includegraphics[width=0.8\textwidth]{Images/Imagenes Cap 2/GraficasExperimentos/SLOVO/3gloss/elpca5.PNG}
    \label{fig:SLOVOG3elpca5}
\end{figure}

\begin{figure}[H]
    \caption{Esta grafica muestra el espacio latente en la epoca 65 utilizando umap, donde las señas \enquote{brother} estan con color azul, \enquote{cold} en naranja y \enquote{man} en verde. Las diferentes formas de los puntos representan una variante diferente del video, siendo el circulo el original, la cruz el desplazado, el cuadrado el desordenado, y la equis la invertida.}
    \centering
    \includegraphics[width=0.8\textwidth]{Images/Imagenes Cap 2/GraficasExperimentos/SLOVO/3gloss/elumap5.PNG}
    \label{fig:SLOVOG3elumap5}
\end{figure}

\begin{figure}[H]
    \caption{Esta grafica muestra el espacio latente en la epoca 85 utilizando pca, donde las señas \enquote{brother} estan con color azul, \enquote{cold} en naranja y \enquote{man} en verde. Las diferentes formas de los puntos representan una variante diferente del video, siendo el circulo el original, la cruz el desplazado, el cuadrado el desordenado, y la equis la invertida.}
    \centering
    \includegraphics[width=0.8\textwidth]{Images/Imagenes Cap 2/GraficasExperimentos/SLOVO/3gloss/elpca6.PNG}
    \label{fig:SLOVOG3elpca6}
\end{figure}

\begin{figure}[H]
    \caption{Esta grafica muestra el espacio latente en la epoca 85 utilizando umap, donde las señas \enquote{brother} estan con color azul, \enquote{cold} en naranja y \enquote{man} en verde. Las diferentes formas de los puntos representan una variante diferente del video, siendo el circulo el original, la cruz el desplazado, el cuadrado el desordenado, y la equis la invertida.}
    \centering
    \includegraphics[width=0.8\textwidth]{Images/Imagenes Cap 2/GraficasExperimentos/SLOVO/3gloss/elumap6.PNG}
    \label{fig:SLOVOG3elumap6}
\end{figure}

\subsection{Con los tres conjuntos de datos}

\subsubsection{Con 2 etiquetas}

\begin{figure}[H]
    \caption{Esta gráfica compara el ratio Semántico del modelo comparado con un \enquote{baseline} o punto de referencia. Siendo este el modelo sin entrenar. El eje Y representa el ratio semántico, que mide la capacidad del modelo para diferenciar entre palabras diferentes, mientras que el eje X representa las épocas de entrenamiento.}
    \centering
    \includegraphics[width=0.8\textwidth]{Images/Imagenes Cap 2/GraficasExperimentos/Los3/2gloss/baseline1.png}
    \label{fig:Los3G2baseline1}
\end{figure}

\begin{figure}[H]
    \caption{Esta gráfica evalúa la capacidad del modelo para entender el orden temporal de las secuencias de video comparadas con sus respectivos \enquote{baselines} del modelo no entrenado. Muestra la distancia euclidiana promedio entre la secuencia original y sus versiones alteradas (shifted, inverted, permuted).}
    \centering
    \includegraphics[width=0.8\textwidth]{Images/Imagenes Cap 2/GraficasExperimentos/Los3/2gloss/baseline2.png}
    \label{fig:Los3G2baseline2}
\end{figure}

\begin{figure}[H]
    \caption{Esta gráfica compara el rendimiento del modelo principal con un modelo más simple en términos de la pérdida total de validación. El eje Y representa el valor de la pérdida, una métrica que indica cuán bien el modelo está aprendiendo, donde los valores más bajos son mejores, y el eje X representa las épocas.}
    \centering
    \includegraphics[width=0.8\textwidth]{Images/Imagenes Cap 2/GraficasExperimentos/Los3/2gloss/baseline3.png}
    \label{fig:Los3G2baseline3}
\end{figure}

\begin{figure}[H]
    \caption{Esta gráfica compara el ratio semántico del modelo comparado con dos \enquote{baselines} o puntos de referencia. Siendo estos el modelo sin entrenar y la representación de PCA. El eje Y representa el ratio semántico, que mide la capacidad del modelo para diferenciar entre palabras diferentes, mientras que el eje X representa las épocas de entrenamiento.}
    \centering
    \includegraphics[width=0.8\textwidth]{Images/Imagenes Cap 2/GraficasExperimentos/Los3/2gloss/baseline4.png}
    \label{fig:Los3G2baseline4}
\end{figure}

\begin{figure}[H]
    \caption{Este gráfico mide directamente la calidad de la separación semántica en el espacio latente para palabras con la misma y diferente clase. El eje Y representa la distancia euclidiana promedio y el eje X son las épocas.}
    \centering
    \includegraphics[width=0.8\textwidth]{Images/Imagenes Cap 2/GraficasExperimentos/Los3/2gloss/gen1.PNG}
    \label{fig:Los3G2gen1}
\end{figure}

\begin{figure}[H]
    \caption{Este gráfico muestra la evolución de la \enquote{Pérdida Total} a lo largo de 100 épocas de entrenamiento. El eje Y representa el valor de la pérdida, una métrica que indica cuán bien el modelo está aprendiendo donde valores más bajos son mejores. El eje X representa las épocas, es decir, cada ciclo completo de entrenamiento sobre el conjunto de datos.}
    \centering
    \includegraphics[width=0.8\textwidth]{Images/Imagenes Cap 2/GraficasExperimentos/Los3/2gloss/loss1.PNG}
    \label{fig:Los3G2loss1}
\end{figure}

\begin{figure}[H]
    \caption{Esta gráfica ilustra la \enquote{Pérdida de Reconstrucción}, que mide qué tan bien el autoencoder del modelo puede reconstruir la entrada original después de haberla comprimido en un espacio latente. Al igual que en la gráfica anterior, el eje Y es el valor de la pérdida y el eje X son las épocas.}
    \centering
    \includegraphics[width=0.8\textwidth]{Images/Imagenes Cap 2/GraficasExperimentos/Los3/2gloss/loss2.PNG}
    \label{fig:Los3G2loss2}
\end{figure}

\begin{figure}[H]
    \caption{Este gráfico muestra la \enquote{Pérdida Triplet Semántica}, una métrica clave que evalúa si el modelo puede diferenciar entre distintos glosarios, en este caso, las señas \enquote{brother} y \enquote{cold}. El objetivo es que las representaciones de un mismo glosario estén más cerca entre sí que las de glosarios diferentes. Igualmente, el eje Y representa el valor de esta pérdida, mientras que el eje X indica las épocas de entrenamiento.}
    \centering
    \includegraphics[width=0.8\textwidth]{Images/Imagenes Cap 2/GraficasExperimentos/Los3/2gloss/loss3.PNG}
    \label{fig:Los3G2loss3}
\end{figure}

\begin{figure}[H]
    \caption{Esta visualización se enfoca en la sensibilidad temporal del modelo, puesto que, mide la diferencia entre la secuencia original de un video y su versión invertida. Esto quiere decir que el modelo debe aprender que una secuencia invertida es significativamente diferente de la original. El eje Y representa el valor de esta pérdida, mientras que el eje X indica las épocas de entrenamiento.}
    \centering
    \includegraphics[width=0.8\textwidth]{Images/Imagenes Cap 2/GraficasExperimentos/Los3/2gloss/loss4.PNG}
    \label{fig:Los3G2loss4}
\end{figure}

\begin{figure}[H]
    \caption{Similar a la gráfica anterior, esta también evalúa la sensibilidad temporal, pero en este caso, compara la secuencia original con una versión donde los fotogramas han sido desordenados aleatoriamente. Donde el objetivo es que el modelo reconozca que una secuencia permutada es muy diferente de la original. El eje Y representa el valor de esta pérdida, mientras que el eje X indica las épocas de entrenamiento.}
    \centering
    \includegraphics[width=0.8\textwidth]{Images/Imagenes Cap 2/GraficasExperimentos/Los3/2gloss/loss5.PNG}
    \label{fig:Los3G2loss5}
\end{figure}

\begin{figure}[H]
    \caption{Esta grafica muestra el espacio latente en la mejor epoca (100) utilizando pca, donde las señas \enquote{brother} estan con color azul y \enquote{cold} en naranja. Las diferentes formas de los puntos representan una variante diferente del video, siendo el circulo el original, la cruz el desplazado, el cuadrado el desordenado, y la equis la invertida.}
    \centering
    \includegraphics[width=0.8\textwidth]{Images/Imagenes Cap 2/GraficasExperimentos/Los3/2gloss/elpca0.PNG}
    \label{fig:Los3G2elpca0}
\end{figure}

\begin{figure}[H]
    \caption{Esta grafica muestra el espacio latente en la mejor epoca (100) utilizando pca, donde las señas del lenguaje \enquote{WLSL} estan color verde, las del \enquote{SLOVO} en rojo y las del \enquote{ISL} en azul.}
    \centering
    \includegraphics[width=0.8\textwidth]{Images/Imagenes Cap 2/GraficasExperimentos/Los3/2gloss/elpcaid0.PNG}
    \label{fig:Los3G2elpca0}
\end{figure}

\begin{figure}[H]
    \caption{Esta grafica muestra el espacio latente en la mejor epoca (100) utilizando umap, donde las señas \enquote{brother} estan con color azul y \enquote{cold} en naranja. Las diferentes formas de los puntos representan una variante diferente del video, siendo el circulo el original, la cruz el desplazado, el cuadrado el desordenado, y la equis la invertida.}
    \centering
    \includegraphics[width=0.8\textwidth]{Images/Imagenes Cap 2/GraficasExperimentos/Los3/2gloss/elumap0.PNG}
    \label{fig:Los3G2elumap0}
\end{figure}

\begin{figure}[H]
    \caption{Esta grafica muestra el espacio latente en la mejor epoca (100) utilizando umap, donde las señas del lenguaje \enquote{WLSL} estan color verde, las del \enquote{SLOVO} en rojo y las del \enquote{ISL} en azul.}
    \centering
    \includegraphics[width=0.8\textwidth]{Images/Imagenes Cap 2/GraficasExperimentos/Los3/2gloss/elumapid0.PNG}
    \label{fig:Los3G2elpca0}
\end{figure}

\begin{figure}[H]
    \caption{Esta grafica muestra el espacio latente en la epoca 5 utilizando pca, donde las señas \enquote{brother} estan con color azul y \enquote{cold} en naranja. Las diferentes formas de los puntos representan una variante diferente del video, siendo el circulo el original, la cruz el desplazado, el cuadrado el desordenado, y la equis la invertida.}
    \centering
    \includegraphics[width=0.8\textwidth]{Images/Imagenes Cap 2/GraficasExperimentos/Los3/2gloss/elpca1.PNG}
    \label{fig:Los3G2elpca1}
\end{figure}

\begin{figure}[H]
    \caption{Esta grafica muestra el espacio latente en la epoca 5 utilizando pca, donde las señas del lenguaje \enquote{WLSL} estan color verde, las del \enquote{SLOVO} en rojo y las del \enquote{ISL} en azul.}
    \centering
    \includegraphics[width=0.8\textwidth]{Images/Imagenes Cap 2/GraficasExperimentos/Los3/2gloss/elpcaid1.PNG}
    \label{fig:Los3G2elpca0}
\end{figure}

\begin{figure}[H]
    \caption{Esta grafica muestra el espacio latente en la epoca 5 utilizando umap, donde las señas \enquote{brother} estan con color azul y \enquote{cold} en naranja. Las diferentes formas de los puntos representan una variante diferente del video, siendo el circulo el original, la cruz el desplazado, el cuadrado el desordenado, y la equis la invertida.}
    \centering
    \includegraphics[width=0.8\textwidth]{Images/Imagenes Cap 2/GraficasExperimentos/Los3/2gloss/elumap1.PNG}
    \label{fig:Los3G2elumap1}
\end{figure}

\begin{figure}[H]
    \caption{Esta grafica muestra el espacio latente en la epoca 5 utilizando umap, donde las señas del lenguaje \enquote{WLSL} estan color verde, las del \enquote{SLOVO} en rojo y las del \enquote{ISL} en azul.}
    \centering
    \includegraphics[width=0.8\textwidth]{Images/Imagenes Cap 2/GraficasExperimentos/Los3/2gloss/elumapid1.PNG}
    \label{fig:Los3G2elpca0}
\end{figure}

\begin{figure}[H]
    \caption{Esta grafica muestra el espacio latente en la epoca 15 utilizando pca, donde las señas \enquote{brother} estan con color azul y \enquote{cold} en naranja. Las diferentes formas de los puntos representan una variante diferente del video, siendo el circulo el original, la cruz el desplazado, el cuadrado el desordenado, y la equis la invertida.}
    \centering
    \includegraphics[width=0.8\textwidth]{Images/Imagenes Cap 2/GraficasExperimentos/Los3/2gloss/elpca2.PNG}
    \label{fig:Los3G2elpca1}
\end{figure}

\begin{figure}[H]
    \caption{Esta grafica muestra el espacio latente en la epoca 15 utilizando pca, donde las señas del lenguaje \enquote{WLSL} estan color verde, las del \enquote{SLOVO} en rojo y las del \enquote{ISL} en azul.}
    \centering
    \includegraphics[width=0.8\textwidth]{Images/Imagenes Cap 2/GraficasExperimentos/Los3/2gloss/elpcaid2.PNG}
    \label{fig:Los3G2elpca0}
\end{figure}

\begin{figure}[H]
    \caption{Esta grafica muestra el espacio latente en la epoca 15 utilizando umap, donde las señas \enquote{brother} estan con color azul y \enquote{cold} en naranja. Las diferentes formas de los puntos representan una variante diferente del video, siendo el circulo el original, la cruz el desplazado, el cuadrado el desordenado, y la equis la invertida.}
    \centering
    \includegraphics[width=0.8\textwidth]{Images/Imagenes Cap 2/GraficasExperimentos/Los3/2gloss/elumap2.PNG}
    \label{fig:Los3G2elumap2}
\end{figure}

\begin{figure}[H]
    \caption{Esta grafica muestra el espacio latente en la epoca 15 utilizando umap, donde las señas del lenguaje \enquote{WLSL} estan color verde, las del \enquote{SLOVO} en rojo y las del \enquote{ISL} en azul.}
    \centering
    \includegraphics[width=0.8\textwidth]{Images/Imagenes Cap 2/GraficasExperimentos/Los3/2gloss/elumapid2.PNG}
    \label{fig:Los3G2elpca0}
\end{figure}

\begin{figure}[H]
    \caption{Esta grafica muestra el espacio latente en la epoca 25 utilizando pca, donde las señas \enquote{brother} estan con color azul y \enquote{cold} en naranja. Las diferentes formas de los puntos representan una variante diferente del video, siendo el circulo el original, la cruz el desplazado, el cuadrado el desordenado, y la equis la invertida.}
    \centering
    \includegraphics[width=0.8\textwidth]{Images/Imagenes Cap 2/GraficasExperimentos/Los3/2gloss/elpca3.PNG}
    \label{fig:Los3G2elpca3}
\end{figure}

\begin{figure}[H]
    \caption{Esta grafica muestra el espacio latente en la epoca 25 utilizando pca, donde las señas del lenguaje \enquote{WLSL} estan color verde, las del \enquote{SLOVO} en rojo y las del \enquote{ISL} en azul.}
    \centering
    \includegraphics[width=0.8\textwidth]{Images/Imagenes Cap 2/GraficasExperimentos/Los3/2gloss/elpcaid3.PNG}
    \label{fig:Los3G2elpca0}
\end{figure}

\begin{figure}[H]
    \caption{Esta grafica muestra el espacio latente en la epoca 25 utilizando umap, donde las señas \enquote{brother} estan con color azul y \enquote{cold} en naranja. Las diferentes formas de los puntos representan una variante diferente del video, siendo el circulo el original, la cruz el desplazado, el cuadrado el desordenado, y la equis la invertida.}
    \centering
    \includegraphics[width=0.8\textwidth]{Images/Imagenes Cap 2/GraficasExperimentos/Los3/2gloss/elumap3.PNG}
    \label{fig:Los3G2elumap3}
\end{figure}

\begin{figure}[H]
    \caption{Esta grafica muestra el espacio latente en la epoca 25 utilizando umap, donde las señas del lenguaje \enquote{WLSL} estan color verde, las del \enquote{SLOVO} en rojo y las del \enquote{ISL} en azul.}
    \centering
    \includegraphics[width=0.8\textwidth]{Images/Imagenes Cap 2/GraficasExperimentos/Los3/2gloss/elumapid3.PNG}
    \label{fig:Los3G2elpca0}
\end{figure}

\begin{figure}[H]
    \caption{Esta grafica muestra el espacio latente en la epoca 45 utilizando pca, donde las señas \enquote{brother} estan con color azul y \enquote{cold} en naranja. Las diferentes formas de los puntos representan una variante diferente del video, siendo el circulo el original, la cruz el desplazado, el cuadrado el desordenado, y la equis la invertida.}
    \centering
    \includegraphics[width=0.8\textwidth]{Images/Imagenes Cap 2/GraficasExperimentos/Los3/2gloss/elpca4.PNG}
    \label{fig:Los3G2elpca4}
\end{figure}

\begin{figure}[H]
    \caption{Esta grafica muestra el espacio latente en la epoca 45 utilizando pca, donde las señas del lenguaje \enquote{WLSL} estan color verde, las del \enquote{SLOVO} en rojo y las del \enquote{ISL} en azul.}
    \centering
    \includegraphics[width=0.8\textwidth]{Images/Imagenes Cap 2/GraficasExperimentos/Los3/2gloss/elpcaid4.PNG}
    \label{fig:Los3G2elpca0}
\end{figure}

\begin{figure}[H]
    \caption{Esta grafica muestra el espacio latente en la epoca 45 utilizando umap, donde las señas \enquote{brother} estan con color azul y \enquote{cold} en naranja. Las diferentes formas de los puntos representan una variante diferente del video, siendo el circulo el original, la cruz el desplazado, el cuadrado el desordenado, y la equis la invertida.}
    \centering
    \includegraphics[width=0.8\textwidth]{Images/Imagenes Cap 2/GraficasExperimentos/Los3/2gloss/elumap4.PNG}
    \label{fig:Los3G2elumap4}
\end{figure}

\begin{figure}[H]
    \caption{Esta grafica muestra el espacio latente en la epoca 45 utilizando umap, donde las señas del lenguaje \enquote{WLSL} estan color verde, las del \enquote{SLOVO} en rojo y las del \enquote{ISL} en azul.}
    \centering
    \includegraphics[width=0.8\textwidth]{Images/Imagenes Cap 2/GraficasExperimentos/Los3/2gloss/elumapid4.PNG}
    \label{fig:Los3G2elpca0}
\end{figure}

\begin{figure}[H]
    \caption{Esta grafica muestra el espacio latente en la epoca 65 utilizando pca, donde las señas \enquote{brother} estan con color azul y \enquote{cold} en naranja. Las diferentes formas de los puntos representan una variante diferente del video, siendo el circulo el original, la cruz el desplazado, el cuadrado el desordenado, y la equis la invertida.}
    \centering
    \includegraphics[width=0.8\textwidth]{Images/Imagenes Cap 2/GraficasExperimentos/Los3/2gloss/elpca5.PNG}
    \label{fig:Los3G2elpca5}
\end{figure}

\begin{figure}[H]
    \caption{Esta grafica muestra el espacio latente en la epoca 65 utilizando pca, donde las señas del lenguaje \enquote{WLSL} estan color verde, las del \enquote{SLOVO} en rojo y las del \enquote{ISL} en azul.}
    \centering
    \includegraphics[width=0.8\textwidth]{Images/Imagenes Cap 2/GraficasExperimentos/Los3/2gloss/elpcaid5.PNG}
    \label{fig:Los3G2elpca0}
\end{figure}

\begin{figure}[H]
    \caption{Esta grafica muestra el espacio latente en la epoca 65 utilizando umap, donde las señas \enquote{brother} estan con color azul y \enquote{cold} en naranja. Las diferentes formas de los puntos representan una variante diferente del video, siendo el circulo el original, la cruz el desplazado, el cuadrado el desordenado, y la equis la invertida.}
    \centering
    \includegraphics[width=0.8\textwidth]{Images/Imagenes Cap 2/GraficasExperimentos/Los3/2gloss/elumap5.PNG}
    \label{fig:Los3G2elumap5}
\end{figure}

\begin{figure}[H]
    \caption{Esta grafica muestra el espacio latente en la epoca 65 utilizando umap, donde las señas del lenguaje \enquote{WLSL} estan color verde, las del \enquote{SLOVO} en rojo y las del \enquote{ISL} en azul.}
    \centering
    \includegraphics[width=0.8\textwidth]{Images/Imagenes Cap 2/GraficasExperimentos/Los3/2gloss/elumapid5.PNG}
    \label{fig:Los3G2elpca0}
\end{figure}

\begin{figure}[H]
    \caption{Esta grafica muestra el espacio latente en la epoca 85 utilizando pca, donde las señas \enquote{brother} estan con color azul y \enquote{cold} en naranja. Las diferentes formas de los puntos representan una variante diferente del video, siendo el circulo el original, la cruz el desplazado, el cuadrado el desordenado, y la equis la invertida.}
    \centering
    \includegraphics[width=0.8\textwidth]{Images/Imagenes Cap 2/GraficasExperimentos/Los3/2gloss/elpca6.PNG}
    \label{fig:Los3G2elpca6}
\end{figure}

\begin{figure}[H]
    \caption{Esta grafica muestra el espacio latente en la epoca 85 utilizando pca, donde las señas del lenguaje \enquote{WLSL} estan color verde, las del \enquote{SLOVO} en rojo y las del \enquote{ISL} en azul.}
    \centering
    \includegraphics[width=0.8\textwidth]{Images/Imagenes Cap 2/GraficasExperimentos/Los3/2gloss/elpcaid6.PNG}
    \label{fig:Los3G2elpca0}
\end{figure}

\begin{figure}[H]
    \caption{Esta grafica muestra el espacio latente en la epoca 85 utilizando umap, donde las señas \enquote{brother} estan con color azul y \enquote{cold} en naranja. Las diferentes formas de los puntos representan una variante diferente del video, siendo el circulo el original, la cruz el desplazado, el cuadrado el desordenado, y la equis la invertida.}
    \centering
    \includegraphics[width=0.8\textwidth]{Images/Imagenes Cap 2/GraficasExperimentos/Los3/2gloss/elumap6.PNG}
    \label{fig:Los3G2elumap6}
\end{figure}

\begin{figure}[H]
    \caption{Esta grafica muestra el espacio latente en la epoca 85 utilizando umap, donde las señas del lenguaje \enquote{WLSL} estan color verde, las del \enquote{SLOVO} en rojo y las del \enquote{ISL} en azul.}
    \centering
    \includegraphics[width=0.8\textwidth]{Images/Imagenes Cap 2/GraficasExperimentos/Los3/2gloss/elumapid6.PNG}
    \label{fig:Los3G2elpca0}
\end{figure}

\subsubsection{Con 3 etiquetas}

\begin{figure}[H]
    \caption{Esta gráfica compara el ratio Semántico del modelo comparado con un \enquote{baseline} o punto de referencia. Siendo este el modelo sin entrenar. El eje Y representa el ratio semántico, que mide la capacidad del modelo para diferenciar entre palabras diferentes, mientras que el eje X representa las épocas de entrenamiento.}
    \centering
    \includegraphics[width=0.8\textwidth]{Images/Imagenes Cap 2/GraficasExperimentos/Los3/3gloss/baseline1.png}
    \label{fig:Los3G3baseline1}
\end{figure}

\begin{figure}[H]
    \caption{Esta gráfica evalúa la capacidad del modelo para entender el orden temporal de las secuencias de video comparadas con sus respectivos \enquote{baselines} del modelo no entrenado. Muestra la distancia euclidiana promedio entre la secuencia original y sus versiones alteradas (shifted, inverted, permuted).}
    \centering
    \includegraphics[width=0.8\textwidth]{Images/Imagenes Cap 2/GraficasExperimentos/Los3/3gloss/baseline2.png}
    \label{fig:Los3G3baseline2}
\end{figure}

\begin{figure}[H]
    \caption{Esta gráfica compara el rendimiento del modelo principal con un modelo más simple en términos de la pérdida total de validación. El eje Y representa el valor de la pérdida, una métrica que indica cuán bien el modelo está aprendiendo, donde los valores más bajos son mejores, y el eje X representa las épocas.}
    \centering
    \includegraphics[width=0.8\textwidth]{Images/Imagenes Cap 2/GraficasExperimentos/Los3/3gloss/baseline3.png}
    \label{fig:Los3G3baseline3}
\end{figure}

\begin{figure}[H]
    \caption{Esta gráfica compara el ratio semántico del modelo comparado con dos \enquote{baselines} o puntos de referencia. Siendo estos el modelo sin entrenar y la representación de PCA. El eje Y representa el ratio semántico, que mide la capacidad del modelo para diferenciar entre palabras diferentes, mientras que el eje X representa las épocas de entrenamiento.}
    \centering
    \includegraphics[width=0.8\textwidth]{Images/Imagenes Cap 2/GraficasExperimentos/Los3/3gloss/baseline4.png}
    \label{fig:Los3G3baseline4}
\end{figure}

\begin{figure}[H]
    \caption{Este gráfico mide directamente la calidad de la separación semántica en el espacio latente para palabras con la misma y diferente clase. El eje Y representa la distancia euclidiana promedio y el eje X son las épocas.}
    \centering
    \includegraphics[width=0.8\textwidth]{Images/Imagenes Cap 2/GraficasExperimentos/Los3/3gloss/gen1.PNG}
    \label{fig:Los3G3gen1}
\end{figure}

\begin{figure}[H]
    \caption{Este gráfico muestra la evolución de la \enquote{Pérdida Total} a lo largo de 100 épocas de entrenamiento. El eje Y representa el valor de la pérdida, una métrica que indica cuán bien el modelo está aprendiendo donde valores más bajos son mejores. El eje X representa las épocas, es decir, cada ciclo completo de entrenamiento sobre el conjunto de datos.}
    \centering
    \includegraphics[width=0.8\textwidth]{Images/Imagenes Cap 2/GraficasExperimentos/Los3/3gloss/loss1.PNG}
    \label{fig:Los3G3loss1}
\end{figure}

\begin{figure}[H]
    \caption{Esta gráfica ilustra la \enquote{Pérdida de Reconstrucción}, que mide qué tan bien el autoencoder del modelo puede reconstruir la entrada original después de haberla comprimido en un espacio latente. Al igual que en la gráfica anterior, el eje Y es el valor de la pérdida y el eje X son las épocas.}
    \centering
    \includegraphics[width=0.8\textwidth]{Images/Imagenes Cap 2/GraficasExperimentos/Los3/3gloss/loss2.PNG}
    \label{fig:Los3G3loss2}
\end{figure}

\begin{figure}[H]
    \caption{Este gráfico muestra la \enquote{Pérdida Triplet Semántica}, una métrica clave que evalúa si el modelo puede diferenciar entre distintos glosarios, en este caso, las señas \enquote{brother}, \enquote{cold} y \enquote{man}. El objetivo es que las representaciones de un mismo glosario estén más cerca entre sí que las de glosarios diferentes. Igualmente, el eje Y representa el valor de esta pérdida, mientras que el eje X indica las épocas de entrenamiento.}
    \centering
    \includegraphics[width=0.8\textwidth]{Images/Imagenes Cap 2/GraficasExperimentos/Los3/3gloss/loss3.PNG}
    \label{fig:Los3G3loss3}
\end{figure}

\begin{figure}[H]
    \caption{Esta visualización se enfoca en la sensibilidad temporal del modelo, puesto que, mide la diferencia entre la secuencia original de un video y su versión invertida. Esto quiere decir que el modelo debe aprender que una secuencia invertida es significativamente diferente de la original. El eje Y representa el valor de esta pérdida, mientras que el eje X indica las épocas de entrenamiento.}
    \centering
    \includegraphics[width=0.8\textwidth]{Images/Imagenes Cap 2/GraficasExperimentos/Los3/3gloss/loss4.PNG}
    \label{fig:Los3G3loss4}
\end{figure}

\begin{figure}[H]
    \caption{Similar a la gráfica anterior, esta también evalúa la sensibilidad temporal, pero en este caso, compara la secuencia original con una versión donde los fotogramas han sido desordenados aleatoriamente. Donde el objetivo es que el modelo reconozca que una secuencia permutada es muy diferente de la original. El eje Y representa el valor de esta pérdida, mientras que el eje X indica las épocas de entrenamiento.}
    \centering
    \includegraphics[width=0.8\textwidth]{Images/Imagenes Cap 2/GraficasExperimentos/Los3/3gloss/loss5.PNG}
    \label{fig:Los3G3loss5}
\end{figure}

\begin{figure}[H]
    \caption{Esta grafica muestra el espacio latente en la mejor epoca (100) utilizando pca, donde las señas \enquote{brother} estan con color azul, las de \enquote{cold} en naranja y las de \enquote{man} en verde. Las diferentes formas de los puntos representan una variante diferente del video, siendo el circulo el original, la cruz el desplazado, el cuadrado el desordenado, y la equis la invertida.}
    \centering
    \includegraphics[width=0.8\textwidth]{Images/Imagenes Cap 2/GraficasExperimentos/Los3/3gloss/elpca0.PNG}
    \label{fig:Los3G3elpca0}
\end{figure}

\begin{figure}[H]
    \caption{Esta grafica muestra el espacio latente en la mejor epoca (100) utilizando pca, donde las señas del lenguaje \enquote{WLSL} estan color verde, las del \enquote{SLOVO} en rojo y las del \enquote{ISL} en azul.}
    \centering
    \includegraphics[width=0.8\textwidth]{Images/Imagenes Cap 2/GraficasExperimentos/Los3/3gloss/elpcaid0.PNG}
    \label{fig:Los3G3elpca0}
\end{figure}

\begin{figure}[H]
    \caption{Esta grafica muestra el espacio latente en la mejor epoca (100) utilizando umap, donde las señas \enquote{brother} estan con color azul, las de \enquote{cold} en naranja y las de \enquote{man} en verde. Las diferentes formas de los puntos representan una variante diferente del video, siendo el circulo el original, la cruz el desplazado, el cuadrado el desordenado, y la equis la invertida.}
    \centering
    \includegraphics[width=0.8\textwidth]{Images/Imagenes Cap 2/GraficasExperimentos/Los3/3gloss/elumap0.PNG}
    \label{fig:Los3G3elumap0}
\end{figure}

\begin{figure}[H]
    \caption{Esta grafica muestra el espacio latente en la mejor epoca (100) utilizando umap, donde las señas del lenguaje \enquote{WLSL} estan color verde, las del \enquote{SLOVO} en rojo y las del \enquote{ISL} en azul.}
    \centering
    \includegraphics[width=0.8\textwidth]{Images/Imagenes Cap 2/GraficasExperimentos/Los3/3gloss/elumapid0.PNG}
    \label{fig:Los3G3elpca0}
\end{figure}

\begin{figure}[H]
    \caption{Esta grafica muestra el espacio latente en la epoca 5 utilizando pca, donde las señas \enquote{brother} estan con color azul, las de \enquote{cold} en naranja y las de \enquote{man} en verde. Las diferentes formas de los puntos representan una variante diferente del video, siendo el circulo el original, la cruz el desplazado, el cuadrado el desordenado, y la equis la invertida.}
    \centering
    \includegraphics[width=0.8\textwidth]{Images/Imagenes Cap 2/GraficasExperimentos/Los3/3gloss/elpca1.PNG}
    \label{fig:Los3G3elpca1}
\end{figure}

\begin{figure}[H]
    \caption{Esta grafica muestra el espacio latente en la epoca 5 utilizando pca, donde las señas del lenguaje \enquote{WLSL} estan color verde, las del \enquote{SLOVO} en rojo y las del \enquote{ISL} en azul.}
    \centering
    \includegraphics[width=0.8\textwidth]{Images/Imagenes Cap 2/GraficasExperimentos/Los3/3gloss/elpcaid1.PNG}
    \label{fig:Los3G3elpca0}
\end{figure}

\begin{figure}[H]
    \caption{Esta grafica muestra el espacio latente en la epoca 5 utilizando umap, donde las señas \enquote{brother} estan con color azul, las de \enquote{cold} en naranja y las de \enquote{man} en verde. Las diferentes formas de los puntos representan una variante diferente del video, siendo el circulo el original, la cruz el desplazado, el cuadrado el desordenado, y la equis la invertida.}
    \centering
    \includegraphics[width=0.8\textwidth]{Images/Imagenes Cap 2/GraficasExperimentos/Los3/3gloss/elumap1.PNG}
    \label{fig:Los3G3elumap1}
\end{figure}

\begin{figure}[H]
    \caption{Esta grafica muestra el espacio latente en la epoca 5 utilizando umap, donde las señas del lenguaje \enquote{WLSL} estan color verde, las del \enquote{SLOVO} en rojo y las del \enquote{ISL} en azul.}
    \centering
    \includegraphics[width=0.8\textwidth]{Images/Imagenes Cap 2/GraficasExperimentos/Los3/3gloss/elumapid1.PNG}
    \label{fig:Los3G3elpca0}
\end{figure}

\begin{figure}[H]
    \caption{Esta grafica muestra el espacio latente en la epoca 15 utilizando pca, donde las señas \enquote{brother} estan con color azul, las de \enquote{cold} en naranja y las de \enquote{man} en verde. Las diferentes formas de los puntos representan una variante diferente del video, siendo el circulo el original, la cruz el desplazado, el cuadrado el desordenado, y la equis la invertida.}
    \centering
    \includegraphics[width=0.8\textwidth]{Images/Imagenes Cap 2/GraficasExperimentos/Los3/3gloss/elpca2.PNG}
    \label{fig:Los3G3elpca1}
\end{figure}

\begin{figure}[H]
    \caption{Esta grafica muestra el espacio latente en la epoca 15 utilizando pca, donde las señas del lenguaje \enquote{WLSL} estan color verde, las del \enquote{SLOVO} en rojo y las del \enquote{ISL} en azul.}
    \centering
    \includegraphics[width=0.8\textwidth]{Images/Imagenes Cap 2/GraficasExperimentos/Los3/3gloss/elpcaid2.PNG}
    \label{fig:Los3G3elpca0}
\end{figure}

\begin{figure}[H]
    \caption{Esta grafica muestra el espacio latente en la epoca 15 utilizando umap, donde las señas \enquote{brother} estan con color azul, las de \enquote{cold} en naranja y las de \enquote{man} en verde. Las diferentes formas de los puntos representan una variante diferente del video, siendo el circulo el original, la cruz el desplazado, el cuadrado el desordenado, y la equis la invertida.}
    \centering
    \includegraphics[width=0.8\textwidth]{Images/Imagenes Cap 2/GraficasExperimentos/Los3/3gloss/elumap2.PNG}
    \label{fig:Los3G3elumap2}
\end{figure}

\begin{figure}[H]
    \caption{Esta grafica muestra el espacio latente en la epoca 15 utilizando umap, donde las señas del lenguaje \enquote{WLSL} estan color verde, las del \enquote{SLOVO} en rojo y las del \enquote{ISL} en azul.}
    \centering
    \includegraphics[width=0.8\textwidth]{Images/Imagenes Cap 2/GraficasExperimentos/Los3/3gloss/elumapid2.PNG}
    \label{fig:Los3G3elpca0}
\end{figure}

\begin{figure}[H]
    \caption{Esta grafica muestra el espacio latente en la epoca 25 utilizando pca, donde las señas \enquote{brother} estan con color azul, las de \enquote{cold} en naranja y las de \enquote{man} en verde. Las diferentes formas de los puntos representan una variante diferente del video, siendo el circulo el original, la cruz el desplazado, el cuadrado el desordenado, y la equis la invertida.}
    \centering
    \includegraphics[width=0.8\textwidth]{Images/Imagenes Cap 2/GraficasExperimentos/Los3/3gloss/elpca3.PNG}
    \label{fig:Los3G3elpca3}
\end{figure}

\begin{figure}[H]
    \caption{Esta grafica muestra el espacio latente en la epoca 25 utilizando pca, donde las señas del lenguaje \enquote{WLSL} estan color verde, las del \enquote{SLOVO} en rojo y las del \enquote{ISL} en azul.}
    \centering
    \includegraphics[width=0.8\textwidth]{Images/Imagenes Cap 2/GraficasExperimentos/Los3/3gloss/elpcaid3.PNG}
    \label{fig:Los3G3elpca0}
\end{figure}

\begin{figure}[H]
    \caption{Esta grafica muestra el espacio latente en la epoca 25 utilizando umap, donde las señas \enquote{brother} estan con color azul, las de \enquote{cold} en naranja y las de \enquote{man} en verde. Las diferentes formas de los puntos representan una variante diferente del video, siendo el circulo el original, la cruz el desplazado, el cuadrado el desordenado, y la equis la invertida.}
    \centering
    \includegraphics[width=0.8\textwidth]{Images/Imagenes Cap 2/GraficasExperimentos/Los3/3gloss/elumap3.PNG}
    \label{fig:Los3G3elumap3}
\end{figure}

\begin{figure}[H]
    \caption{Esta grafica muestra el espacio latente en la epoca 25 utilizando umap, donde las señas del lenguaje \enquote{WLSL} estan color verde, las del \enquote{SLOVO} en rojo y las del \enquote{ISL} en azul.}
    \centering
    \includegraphics[width=0.8\textwidth]{Images/Imagenes Cap 2/GraficasExperimentos/Los3/3gloss/elumapid3.PNG}
    \label{fig:Los3G3elpca0}
\end{figure}

\begin{figure}[H]
    \caption{Esta grafica muestra el espacio latente en la epoca 45 utilizando pca, donde las señas \enquote{brother} estan con color azul, las de \enquote{cold} en naranja y las de \enquote{man} en verde. Las diferentes formas de los puntos representan una variante diferente del video, siendo el circulo el original, la cruz el desplazado, el cuadrado el desordenado, y la equis la invertida.}
    \centering
    \includegraphics[width=0.8\textwidth]{Images/Imagenes Cap 2/GraficasExperimentos/Los3/3gloss/elpca4.PNG}
    \label{fig:Los3G3elpca4}
\end{figure}

\begin{figure}[H]
    \caption{Esta grafica muestra el espacio latente en la epoca 45 utilizando pca, donde las señas del lenguaje \enquote{WLSL} estan color verde, las del \enquote{SLOVO} en rojo y las del \enquote{ISL} en azul.}
    \centering
    \includegraphics[width=0.8\textwidth]{Images/Imagenes Cap 2/GraficasExperimentos/Los3/3gloss/elpcaid4.PNG}
    \label{fig:Los3G3elpca0}
\end{figure}

\begin{figure}[H]
    \caption{Esta grafica muestra el espacio latente en la epoca 45 utilizando umap, donde las señas \enquote{brother} estan con color azul, las de \enquote{cold} en naranja y las de \enquote{man} en verde. Las diferentes formas de los puntos representan una variante diferente del video, siendo el circulo el original, la cruz el desplazado, el cuadrado el desordenado, y la equis la invertida.}
    \centering
    \includegraphics[width=0.8\textwidth]{Images/Imagenes Cap 2/GraficasExperimentos/Los3/3gloss/elumap4.PNG}
    \label{fig:Los3G3elumap4}
\end{figure}

\begin{figure}[H]
    \caption{Esta grafica muestra el espacio latente en la epoca 45 utilizando umap, donde las señas del lenguaje \enquote{WLSL} estan color verde, las del \enquote{SLOVO} en rojo y las del \enquote{ISL} en azul.}
    \centering
    \includegraphics[width=0.8\textwidth]{Images/Imagenes Cap 2/GraficasExperimentos/Los3/3gloss/elumapid4.PNG}
    \label{fig:Los3G3elpca0}
\end{figure}

\begin{figure}[H]
    \caption{Esta grafica muestra el espacio latente en la epoca 65 utilizando pca, donde las señas \enquote{brother} estan con color azul, las de \enquote{cold} en naranja y las de \enquote{man} en verde. Las diferentes formas de los puntos representan una variante diferente del video, siendo el circulo el original, la cruz el desplazado, el cuadrado el desordenado, y la equis la invertida.}
    \centering
    \includegraphics[width=0.8\textwidth]{Images/Imagenes Cap 2/GraficasExperimentos/Los3/3gloss/elpca5.PNG}
    \label{fig:Los3G3elpca5}
\end{figure}

\begin{figure}[H]
    \caption{Esta grafica muestra el espacio latente en la epoca 65 utilizando pca, donde las señas del lenguaje \enquote{WLSL} estan color verde, las del \enquote{SLOVO} en rojo y las del \enquote{ISL} en azul.}
    \centering
    \includegraphics[width=0.8\textwidth]{Images/Imagenes Cap 2/GraficasExperimentos/Los3/3gloss/elpcaid5.PNG}
    \label{fig:Los3G3elpca0}
\end{figure}

\begin{figure}[H]
    \caption{Esta grafica muestra el espacio latente en la epoca 65 utilizando umap, donde las señas \enquote{brother} estan con color azul, las de \enquote{cold} en naranja y las de \enquote{man} en verde. Las diferentes formas de los puntos representan una variante diferente del video, siendo el circulo el original, la cruz el desplazado, el cuadrado el desordenado, y la equis la invertida.}
    \centering
    \includegraphics[width=0.8\textwidth]{Images/Imagenes Cap 2/GraficasExperimentos/Los3/3gloss/elumap5.PNG}
    \label{fig:Los3G3elumap5}
\end{figure}

\begin{figure}[H]
    \caption{Esta grafica muestra el espacio latente en la epoca 65 utilizando umap, donde las señas del lenguaje \enquote{WLSL} estan color verde, las del \enquote{SLOVO} en rojo y las del \enquote{ISL} en azul.}
    \centering
    \includegraphics[width=0.8\textwidth]{Images/Imagenes Cap 2/GraficasExperimentos/Los3/3gloss/elumapid5.PNG}
    \label{fig:Los3G3elpcaid5}
\end{figure}

\begin{figure}[H]
    \caption{Esta grafica muestra el espacio latente en la epoca 85 utilizando pca, donde las señas \enquote{brother} estan con color azul, las de \enquote{cold} en naranja y las de \enquote{man} en verde. Las diferentes formas de los puntos representan una variante diferente del video, siendo el circulo el original, la cruz el desplazado, el cuadrado el desordenado, y la equis la invertida.}
    \centering
    \includegraphics[width=0.8\textwidth]{Images/Imagenes Cap 2/GraficasExperimentos/Los3/3gloss/elpca6.PNG}
    \label{fig:Los3G3elpca6}
\end{figure}

\begin{figure}[H]
    \caption{Esta grafica muestra el espacio latente en la epoca 85 utilizando pca, donde las señas del lenguaje \enquote{WLSL} estan color verde, las del \enquote{SLOVO} en rojo y las del \enquote{ISL} en azul.}
    \centering
    \includegraphics[width=0.8\textwidth]{Images/Imagenes Cap 2/GraficasExperimentos/Los3/3gloss/elpcaid6.PNG}
    \label{fig:Los3G3elpcaid6}
\end{figure}

\begin{figure}[H]
    \caption{Esta grafica muestra el espacio latente en la epoca 85 utilizando umap, donde las señas \enquote{brother} estan con color azul, las de \enquote{cold} en naranja y las de \enquote{man} en verde. Las diferentes formas de los puntos representan una variante diferente del video, siendo el circulo el original, la cruz el desplazado, el cuadrado el desordenado, y la equis la invertida.}
    \centering
    \includegraphics[width=0.8\textwidth]{Images/Imagenes Cap 2/GraficasExperimentos/Los3/3gloss/elumap6.PNG}
    \label{fig:Los3G3elumap6}
\end{figure}

\begin{figure}[H]
    \caption{Esta grafica muestra el espacio latente en la epoca 85 utilizando umap, donde las señas del lenguaje \enquote{WLSL} estan color verde, las del \enquote{SLOVO} en rojo y las del \enquote{ISL} en azul.}
    \centering
    \includegraphics[width=0.8\textwidth]{Images/Imagenes Cap 2/GraficasExperimentos/Los3/3gloss/elumapid6.PNG}
    \label{fig:Los3G3elumapid6}
\end{figure}

    
    % Indices (Opcional)
    % \chapter*{\'Indices}
    % \input{BackMatter/B10-Index}
    
    % Incluye las referencias
    %\bibliographystyle{apalike}
    \printbibliography

\end{document}
